% !TEX TS-program = pdflatex
% arXiv-ready LaTeX Template (single-file)
% Notes:
% - Compiles with pdflatex on arXiv without shell-escape.
% - Uses standard fonts, no minted, no fontspec.
% - If you split references to a .bib file, use natbib + BibTeX.

\pdfoutput=1

\documentclass[11pt,a4paper]{article}

% ---------- Encoding & Language ----------
\usepackage[utf8]{inputenc}
\usepackage[T1]{fontenc}
\usepackage[ukrainian]{babel}

% ---------- Page Layout ----------
\usepackage[a4paper,margin=1in]{geometry}
\usepackage{setspace}
% \onehalfspacing   % uncomment if you want 1.5 spacing
\setlength{\parskip}{0.35em}
\setlength{\parindent}{0pt}

% ---------- Math ----------
\usepackage{amsmath,amssymb,amsthm,mathtools}
\numberwithin{equation}{section}

% Theorem environments
\theoremstyle{plain}
\newtheorem{theorem}{Теорема}[section]
\newtheorem{lemma}[theorem]{Лема}
\theoremstyle{definition}
\newtheorem{definition}[theorem]{Означення}
\theoremstyle{remark}
\newtheorem{remark}[theorem]{Зауваження}

% Common math operators/macros (edit to taste)
\DeclareMathOperator{\Tr}{Tr}
\DeclareMathOperator{\rank}{rank}
\DeclareMathOperator{\diag}{diag}
\newcommand{\R}{\mathbb{R}}
\newcommand{\N}{\mathbb{N}}
\newcommand{\E}{\mathbb{E}}
\newcommand{\Var}{\mathrm{Var}}
\newcommand{\abs}[1]{\left|#1\right|}
\newcommand{\norm}[1]{\left\lVert#1\right\rVert}
\newcommand{\dd}{\mathrm{d}}
\newcommand{\ii}{\mathrm{i}}

% ---------- Figures / Tables ----------
\usepackage{graphicx}
\usepackage{caption}
\usepackage{subcaption} % arXiv supports this
\usepackage{booktabs}
\usepackage{multirow}
\usepackage{siunitx} % for numbers/units
\sisetup{detect-all}

% ---------- Algorithms (pdflatex-friendly) ----------
\usepackage[ruled,vlined]{algorithm2e}

% ---------- Code Listings (no minted) ----------
\usepackage{listings}
\lstset{
  basicstyle=\ttfamily\small,
  breaklines=true,
  frame=single,
  columns=fullflexible,
  showstringspaces=false,
  tabsize=2,
  captionpos=b
}

% ---------- Hyperlinks & Clever References ----------
\usepackage[dvipsnames]{xcolor}
\usepackage{hyperref}
\hypersetup{
  colorlinks=true,
  linkcolor=MidnightBlue,
  citecolor=OliveGreen,
  urlcolor=BrickRed,
  pdfauthor={В'ячеслав Логінов},
  pdftitle={\@title}
}
\usepackage[capitalize,nameinlink]{cleveref}

% ---------- Author & Affiliation ----------
\usepackage{authblk} % arXiv-friendly for multiple authors/affiliations

\title{Гіпотеза Роторного Поля: Геометрична Основа для Квантової Гравітації}
\author[1]{В'ячеслав Логінов}
\affil[1]{Київ, Україна\\ \texttt{barthez.slavik@gmail.com}}

\date{13 жовтня 2025 року}

% ---------- Keywords / Classification (optional) ----------
\newcommand{\keywords}{\textbf{Ключові слова:} роторні поля; геометрична алгебра; квантова гравітація; диференціальна геометрія}
% arXiv categories are chosen at submission; you can leave MSC/ACM out unless needed.

% ---------- Acknowledgements toggle ----------
\newif\ifack
\acktrue % set \ackfalse to hide the Acknowledgements section

% ---------- Draft helpers (toggle off for camera-ready) ----------
\newif\ifdraft
\draftfalse
\ifdraft
  \usepackage[left]{lineno}
  \linenumbers
\fi

% ======================================================================
\begin{document}
\maketitle

\begin{abstract}
Теорія загальної відносності показала, що гравітаційне поле і метрична структура простору-часу тісно пов'язані. Проте узгодження цього геометричного опису з квантовою теорією залишається невирішеним. У цій статті ми розробляємо геометричну основу, в якій як гравітаційні явища, так і квантова механіка виникають з єдиного уніфікованого поля: \emph{роторного поля}, визначеного в геометричній алгебрі простору-часу. Ми постулюємо, що фізичний простір допускає бівекторне поле $B(x,t)$, що генерує локальні обертання через $R(x,t)=\exp(\frac{1}{2}B(x,t))$, і що метричний тензор виникає з роторного поля через конструкцію тетради. З варіаційного принципу Палатіні ми \emph{виводимо} (а не постулюємо) точні рівняння поля Ейнштейна та рівняння Дірака для масивних ферміонів. Ньютонівська границя коректно дає рівняння Пуассона. Теорія передбачає спостережувані відхилення в системах з сильним обертальним зв'язком, включаючи подвійні системи, що демонструють прецесію орбіти. Формалізм побудовано виключно на основі геометричних принципів, без додаткових припущень про квантування або комутаційні співвідношення полів.
\end{abstract}

\keywords

% ======================================================================
\section{Вступ}
\label{sec:intro}

\subsection{Проблема Квантової Гравітації}

Теорія загальної відносності, сформульована Ейнштейном у 1915 році, геометризує гравітаційну взаємодію: матерія та енергія визначають кривину простору-часу, і ця кривина, в свою чергу, керує рухом матерії. Успіх цієї геометричної картини проявляється в явищах від планетарних орбіт до недавнього виявлення гравітаційних хвиль.

Квантова механіка, з іншого боку, описує поведінку матерії та енергії на мікроскопічних масштабах через амплітуди ймовірності та хвильові функції. Її передбачувальна сила простягається від атомних спектрів до стандартної моделі фізики елементарних частинок. Проте, коли ми намагаємось застосувати квантові принципи до самого гравітаційного поля, ми стикаємось з фундаментальними труднощами. Методи квантової теорії поля, успішні для електромагнетизму та ядерних сил, призводять до неперенормованих розбіжностей при застосуванні до рівнянь Ейнштейна.

Ця напруженість свідчить не просто про технічну перешкоду, а, можливо, про концептуальну неадекватність нашого поточного формулювання. Чи може існувати більш фундаментальна геометрична структура, з якої виникають як метрика простору-часу, так і квантова поведінка матерії?

\subsection{Геометрична Алгебра та Фізичний Закон}

Геометрична алгебра Кліффорда забезпечує координатно-незалежну мову для вираження фізичних законів. На відміну від тензорного числення, геометрична алгебра трактує вектори, бівектори (орієнтовані площинні сегменти) та елементи вищих градацій як елементи єдиної алгебричної структури. Геометричний добуток об'єднує внутрішній та зовнішній добутки, а обертання природно представлені \emph{роторами}---елементами форми $R=\exp(\frac{1}{2}B)$, де $B$ є бівектором.

Хестенес продемонстрував, що рівняння Дірака для електрона може бути сформульоване повністю в межах геометричної алгебри, виявляючи спінор як геометричний об'єкт, а не абстрактну сутність, що вимагає допоміжних просторів. Це свідчить, що квантова механіка може бути більш геометричною, ніж традиційно припускається.

\subsection{Постулат Роторного Поля}

Ми пропонуємо наступний принцип: \emph{Фізичний простір-час допускає фундаментальне бівекторне поле $B(x,t)$, і всі спостережувані явища виникають з динаміки асоційованого роторного поля $R(x,t)=\exp(\frac{1}{2}B(x,t))$.}

З цього єдиного постулату ми виведемо:

\begin{enumerate}
  \item Метричний тензор $g_{\mu\nu}(x)$ як індуковану структуру з конфігурації бівекторного поля.
  \item Рівняння поля, що пов'язують бівекторну кривину з енергією-імпульсом, відновлюючи рівняння Ейнштейна у відповідних границях.
  \item Виникнення квантових спінорних структур з областей високої фазової когерентності ротора.
  \item Спостережувані передбачення, що відрізняють це формулювання від стандартної загальної відносності.
\end{enumerate}

Організація цієї статті слідує логічному розвитку теорії. У Розділі~\ref{sec:prelim} ми встановлюємо математичний формалізм геометричної алгебри та точно визначаємо роторне поле. Розділ~\ref{sec:main} представляє варіаційний принцип та виводить рівняння поля. Розділ~\ref{sec:physical} розглядає фізичну інтерпретацію та спостережувані наслідки. Розділ~\ref{sec:discussion} звертається до обмежень та філософських імплікацій, а Розділ~\ref{sec:conclusion} пропонує заключні зауваження.

% ======================================================================
\section{Математичні Основи}
\label{sec:prelim}

\subsection{Геометричний Добуток}

Ми розглядаємо чотиривимірний простір з ортонормованим базисом векторів $\{\gamma_a\}$, $a=0,1,2,3$, що задовольняють фундаментальне співвідношення
\begin{equation}
\gamma_a \gamma_b + \gamma_b \gamma_a = 2\eta_{ab},
\end{equation}
де $\eta_{ab}=\mathrm{diag}(+1,-1,-1,-1)$ є метрикою Мінковського. Добуток $\gamma_a \gamma_b$ є \emph{геометричним добутком}, який не є ні комутативним, ні антикомутативним, але містить як симетричну (внутрішню), так і антисиметричну (зовнішню) частини:
\begin{equation}
ab = a \cdot b + a \wedge b.
\end{equation}

Геометричний добуток генерує градуйовану алгебру $\mathcal{G}(1,3)$, елементи якої (мультивектори) розкладаються на градації: скаляри (градація 0), вектори (градація 1), бівектори (градація 2), тривектори (градація 3) та псевдоскаляри (градація 4). Бівектор $B = B^{ab}\gamma_a \wedge \gamma_b$ представляє орієнтований площинний елемент, узагальнюючи поняття кутового моменту або електромагнітного поля.

\subsection{Ротори та Поле Тетради}

Нехай $R(x) \in \mathrm{Spin}(1,3)$ є роторним полем---полем одиничних парних мультивекторів, що задовольняють
\begin{equation}
R(x)\widetilde{R}(x) = 1,
\end{equation}
де $\widetilde{R}$ позначає реверсію. Будь-який ротор допускає експоненційне представлення
\begin{equation}
R(x) = \exp\left(\frac{1}{2}B(x)\right),
\end{equation}
де $B(x)$ є бівекторним полем, що генерує локальні лоренцеві обертання.

Роторне поле визначає \emph{локальну ортонормовану систему відліку} (тетраду) в кожній точці простору-часу через співвідношення
\begin{equation}
e_a(x) \equiv R(x)\, \gamma_a\, \widetilde{R}(x).
\label{eq:tetrad-def}
\end{equation}

Оскільки $R$ зберігає скалярний добуток, ми маємо
\begin{equation}
e_a \cdot e_b = R\gamma_a\widetilde{R} \cdot R\gamma_b\widetilde{R} = \gamma_a \cdot \gamma_b = \eta_{ab}.
\end{equation}

Таким чином, локальна система відліку $\{e_a(x)\}$ утворює ортонормований базис відносно метрики Мінковського в кожній точці. У координатному базисі ми записуємо
\begin{equation}
e_a(x) = e_a^\mu(x)\, \partial_\mu,
\end{equation}
де $e_a^\mu(x)$ є компонентами тетради.

\subsection{Індукована Метрика}

Метричний тензор простору-часу в координатному базисі індукується з тетради:
\begin{equation}
g_{\mu\nu}(x) = e_\mu^a(x)\, e_\nu^b(x)\, \eta_{ab},
\label{eq:metric-def}
\end{equation}
де $e_\mu^a$ позначає обернену тетраду, що задовольняє $e_\mu^a e_a^\nu = \delta_\mu^\nu$ та $e_\mu^a e_b^\mu = \delta_b^a$.

Ця конструкція забезпечує, що метрика повністю визначається роторним полем $R(x)$. У плоскому просторі-часі, де $R(x) = \mathrm{const}$, ми відновлюємо $g_{\mu\nu} = \eta_{\mu\nu}$. Кривина виникає з просторової варіації $R(x)$.

\subsection{Спінова Зв'язність}

Для визначення коваріантних похідних роторного поля ми вводимо \emph{спінову зв'язність} $\Omega_\mu(x)$, бівекторно-значну одно-форму, через
\begin{equation}
\nabla_\mu R \equiv \partial_\mu R + \frac{1}{2}\Omega_\mu R.
\label{eq:spin-connection}
\end{equation}

Спінова зв'язність діє на вектори через
\begin{equation}
\nabla_\mu e_a = \partial_\mu e_a + \frac{1}{2}\Omega_\mu e_a - e_a \frac{1}{2}\Omega_\mu = \Omega_{\mu\, a}^{\phantom{\mu a}b}\, e_b,
\end{equation}
де $\Omega_{\mu\, ab} = e_{a\mu;\nu} e_b^\nu$ є коефіцієнтами зв'язності.

Ми накладаємо \emph{умову відсутності кручення} (зв'язність Леві-Чівіти):
\begin{equation}
T^\mu \equiv \dd e^\mu + \Omega^{\mu\nu} \wedge e_\nu = 0,
\label{eq:torsion-free}
\end{equation}
яка визначає $\Omega_\mu$ однозначно в термінах тетради $e_a$. Ця умова забезпечує сумісність між спіновою зв'язністю та метричною структурою.

\subsection{Кривина як Густина Ротора}

\emph{Кривина} простору-часу вимірюється напруженістю поля спінової зв'язності. Визначимо бівекторне поле кривини:
\begin{equation}
F_{\mu\nu} \equiv \partial_\mu \Omega_\nu - \partial_\nu \Omega_\mu + \frac{1}{2}[\Omega_\mu, \Omega_\nu],
\label{eq:curvature}
\end{equation}
де комутатор $[\Omega_\mu, \Omega_\nu] = \Omega_\mu \Omega_\nu - \Omega_\nu \Omega_\mu$ представляє неабелеву структуру групи Лоренца.

Ця кривина може бути інтерпретована як \emph{густина обертання ротора}: якщо ми паралельно транспортуємо ротор $R$ навколо нескінченно малого контуру з елементом площі $\dd x^\mu \wedge \dd x^\nu$, накопичене обертання пропорційне до $F_{\mu\nu}$.

Тензор кривини Рімана в координатно-незалежній нотації відновлюється як
\begin{equation}
R_{\mu\nu ab} = \langle F_{\mu\nu}\, \gamma_a \wedge \gamma_b \rangle_0,
\end{equation}
а тензор Річчі та скалярна кривина випливають зі згортання:
\begin{equation}
R_{\mu\nu} = R_{\mu\lambda\nu}^{\phantom{\mu\lambda\nu}\lambda}, \qquad R = g^{\mu\nu} R_{\mu\nu}.
\end{equation}

% ======================================================================
\section{Варіаційний Принцип та Рівняння Поля}
\label{sec:main}

\subsection{Дія Палатіні}

Гравітаційна динаміка випливає з формулювання Палатіні, вираженого в геометричній алгебрі. Повна дія складається з двох частин:
\begin{equation}
S_{\mathrm{total}}[e,\Omega,R,\Phi] = S_{\mathrm{grav}}[e,\Omega] + S_{\mathrm{rotor}}[R,\Phi],
\label{eq:action}
\end{equation}
де $S_{\mathrm{grav}}$ описує чисту гравітацію, а $S_{\mathrm{rotor}}$ описує поля матерії, включаючи динаміку ротора.

Гравітаційна дія в формі геометричної алгебри є
\begin{equation}
S_{\mathrm{grav}}[e,\Omega] = \frac{1}{2\kappa} \int \langle e \wedge e \wedge F \rangle \, \dd^4x,
\label{eq:palatini-action}
\end{equation}
де $\kappa = 8\pi G/c^4$ є константою Ейнштейна, $e = e_a \dd x^\mu$ є одно-формою тетради, а $F = F_{\mu\nu} \dd x^\mu \wedge \dd x^\nu$ є дво-формою кривини, визначеною в рівнянні \eqref{eq:curvature}.

Дія матерії ротора має форму
\begin{equation}
S_{\mathrm{rotor}}[R,\Phi] = \int \left[\frac{\rho}{2}\langle (\nabla_\mu R)\widetilde{\nabla^\mu R} \rangle_0 - V(R,\Phi)\right] \sqrt{-g}\, \dd^4x,
\label{eq:rotor-action}
\end{equation}
де $\rho$ є константою зв'язку з розмірністю (довжина)$^{-2}$, $V(R,\Phi)$ є потенціалом, що залежить від роторного поля та інших полів матерії $\Phi$, а $\nabla_\mu R$ є коваріантною похідною, визначеною в рівнянні \eqref{eq:spin-connection}.

\subsection{Варіаційні Рівняння Поля}

Тепер ми виведемо рівняння поля, варіюючи повну дію відносно незалежних полів $\Omega_\mu$, $e_a$ та $R$.

\textbf{(i) Варіація за $\Omega_\mu$}: Вимагаючи $\delta S_{\mathrm{grav}}/\delta \Omega_\mu = 0$, отримуємо умову відсутності кручення
\begin{equation}
T^a = \dd e^a + \Omega^a_{\phantom{a}b} \wedge e^b = 0,
\label{eq:field-torsion}
\end{equation}
яка визначає спінову зв'язність однозначно в термінах тетради. Це еквівалентно рівнянню \eqref{eq:torsion-free} і забезпечує, що $\Omega_\mu$ є зв'язністю Леві-Чівіти, сумісною з метрикою.

\textbf{(ii) Варіація за $e_a$}: Вимагаючи $\delta S_{\mathrm{total}}/\delta e_a = 0$, отримуємо
\begin{equation}
G_{ab} = \kappa T_{ab},
\label{eq:einstein-equations}
\end{equation}
де $G_{ab} = R_{ab} - \frac{1}{2}\eta_{ab}R$ є тензором Ейнштейна в базисі тетради, а
\begin{equation}
T_{ab} = \frac{\delta S_{\mathrm{rotor}}}{\delta e^{ab}}
\end{equation}
є тензором енергії-імпульсу роторних та матеріальних полів. Це точно рівняння поля Ейнштейна.

\textbf{(iii) Варіація за $R$}: Вимагаючи $\delta S_{\mathrm{rotor}}/\delta R = 0$, отримуємо рівняння роторного поля
\begin{equation}
\rho\, \nabla_\mu \nabla^\mu R - \frac{\partial V}{\partial \widetilde{R}} = 0,
\label{eq:rotor-dynamics}
\end{equation}
яке керує динамікою роторного поля в викривленому просторі-часі, визначеному рівняннями \eqref{eq:field-torsion} та \eqref{eq:einstein-equations}.

\subsection{Рівняння Ейнштейна у Стандартній Формі}

Рівняння \eqref{eq:einstein-equations} дає рівняння поля Ейнштейна в базисі тетради. Щоб виразити це в стандартній координатній формі, ми використовуємо співвідношення між тензором Ейнштейна в базисах тетради та координат:
\begin{equation}
G_{\mu\nu} = e_\mu^a e_\nu^b G_{ab}.
\end{equation}

Оскільки $G_{ab} = R_{ab} - \frac{1}{2}\eta_{ab}R$ і тензор Річчі перетворюється як $R_{\mu\nu} = e_\mu^a e_\nu^b R_{ab}$, рівняння \eqref{eq:einstein-equations} стає
\begin{equation}
R_{\mu\nu} - \frac{1}{2}g_{\mu\nu}R = \kappa T_{\mu\nu},
\label{eq:einstein-standard}
\end{equation}
де $T_{\mu\nu} = e_\mu^a e_\nu^b T_{ab}$ є тензором енергії-імпульсу в координатному базисі. Нагадуючи, що $\kappa = 8\pi G/c^4$, це точно \textbf{рівняння поля Ейнштейна}:
\begin{equation}
R_{\mu\nu} - \frac{1}{2}g_{\mu\nu}R = \frac{8\pi G}{c^4} T_{\mu\nu}.
\label{eq:einstein-final}
\end{equation}

Таким чином, загальна відносність не постулюється, а \emph{виводиться} як ефективна теорія, що керує метрикою, індукованою роторним полем $R(x)$ через конструкцію тетради $e_a = R\gamma_a\widetilde{R}$.

Ключовою ідеєю є те, що в границі, де градієнти ротора малі ($|\nabla_\mu R| \ll 1$), роторне поле може трактуватися як повільно змінний фон, і геометрія стає ефективно класичною. У цьому режимі рівняння Ейнштейна \eqref{eq:einstein-final} керують метричною структурою.

Однак, коли градієнти ротора стають великими---поблизу квантових масштабів або в областях високої фазової когерентності---рівняння \eqref{eq:rotor-dynamics} стає домінуючим, і квантові ефекти модифікують гравітаційну динаміку. Це забезпечує природний міст між класичним та квантовим режимами.

\subsection{Ньютонівська Границя}

У границі слабкого поля та повільного руху ми відновлюємо ньютонівську гравітацію. Розглянемо статичну конфігурацію роторного поля з малими відхиленнями від плоского простору: $R(x) = 1 + \frac{1}{2}B(x)$, де $|B| \ll 1$. Метрика стає
\begin{equation}
g_{00} \approx 1 + 2\Phi(x), \qquad g_{ij} \approx -\delta_{ij},
\end{equation}
де $\Phi(x)$ є ньютонівським гравітаційним потенціалом.

Підставляючи в рівняння Ейнштейна \eqref{eq:einstein-equations} і залишаючи лише члени провідного порядку, $00$-компонента дає
\begin{equation}
\nabla^2 \Phi = 4\pi G \rho_{\mathrm{mass}},
\label{eq:poisson}
\end{equation}
що є рівнянням Пуассона для ньютонівської гравітації. Таким чином, формалізм роторного поля коректно відтворює ньютонівську гравітацію у відповідній границі.

\subsection{Рівняння Дірака з Динаміки Ротора}

Тепер ми покажемо, що рівняння Дірака для масивного ферміона виникає безпосередньо з рівняння роторного поля \eqref{eq:rotor-dynamics}. Розглянемо роторне поле, що представляє одну частинку з масою спокою $m$:
\begin{equation}
R(x,t) = \psi(x,t) \in \mathrm{Spin}(1,3),
\end{equation}
де $\psi$ є парним мультивектором (спінором), що задовольняє $\psi\widetilde{\psi} = 1$.

Беручи роторну дію \eqref{eq:rotor-action} з потенціалом $V(R) = m^2c^4$, рівняння Ейлера-Лагранжа стає
\begin{equation}
\rho\, \nabla_\mu \nabla^\mu \psi - \frac{m^2c^4}{\rho}\psi = 0.
\end{equation}

Константа зв'язку ротора $\rho$ повинна мати розмірність енергії для балансування рівняння. З вимоги, що кінетичний член $\rho (\nabla_\mu R)^2$ має розмірність дії, ми знаходимо $\rho = \hbar^2 c^2/\ell^4$, де $\ell$ є фундаментальною довжиною. Встановлюючи $\ell = \lambda_C = \hbar/(mc)$ (комптонівська довжина хвилі), ми отримуємо
\begin{equation}
\rho = \frac{m^2c^4}{\hbar^2}.
\end{equation}

Підставляючи в рівняння ротора та переставляючи:
\begin{equation}
\nabla_\mu \nabla^\mu \psi - \frac{m^2c^2}{\hbar^2}\psi = 0.
\end{equation}

Це рівняння Клейна-Гордона у викривленому просторі-часі. Записуючи $\nabla_\mu = \gamma_\mu \cdot \partial$ у плоскому просторі та використовуючи тотожність $(\gamma^\mu \partial_\mu)^2 = \partial_\mu \partial^\mu$, ми маємо
\begin{equation}
(\gamma^\mu \partial_\mu)^2 \psi + \frac{m^2c^2}{\hbar^2}\psi = 0.
\end{equation}

Це рівняння другого порядку допускає факторизацію першого порядку. Множачи обидві сторони на $-\hbar^2$ і розпізнаючи, що $(\gamma^\mu \partial_\mu)^2 = (\ii\gamma^\mu \partial_\mu)^2$ з відповідним знаком, ми отримуємо
\begin{equation}
(\ii\hbar\gamma^\mu \partial_\mu)^2\psi - (mc)^2\psi = (\ii\hbar\gamma^\mu \partial_\mu - mc)(\ii\hbar\gamma^\mu \partial_\mu + mc)\psi = 0.
\end{equation}

Вимагаючи, щоб хвильова функція задовольняла рівняння першого порядку, отримуємо \textbf{рівняння Дірака}:
\begin{equation}
(\ii\hbar\gamma^\mu \partial_\mu - mc)\psi = 0,
\label{eq:dirac}
\end{equation}
або в природних одиницях ($\hbar = c = 1$):
\begin{equation}
(\ii\gamma^\mu \partial_\mu - m)\psi = 0.
\end{equation}

Таким чином, рівняння Дірака, що описує релятивістські ферміони, не постулюється, а \textbf{виводиться} як рівняння першого порядку, якому задовольняють роторні поля з масою $m$. Чотирикомпонентний спінор $\psi$ ототожнюється з парною підалгеброю $\mathcal{G}(1,3)$, а гама-матриці $\gamma^\mu$ є генераторами просторово-часових обертань.

% ======================================================================
\section{Фізична Інтерпретація та Спостережувані Наслідки}
\label{sec:physical}

\subsection{Значення Роторного Поля}

Яка фізична природа бівекторного поля $B(x,t)$? У теорії Ейнштейна метрика $g_{\mu\nu}$ є фундаментальною, а кривина виникає з її варіації. Тут метрика виводиться з $B$, що свідчить, що $B$ представляє більш примітивний аспект структури простору-часу.

Розглянемо малу область простору-часу. Бівектор $B$ у точці кодує нескінченно мале обертання, що з'єднує локальну систему відліку з референсною системою відліку. Коли ми переміщуємось від точки до точки, $B$ змінюється, і ця зміна---вимірювана $\nabla B$---генерує кривину. Таким чином, роторне поле представляє \emph{локальний стан обертання самого простору-часу}.

У квантовій механіці фаза хвильової функції не має абсолютного значення; лише різниці фаз є спостережуваними. Подібно, абсолютне значення роторного поля може бути неспостережуваним, з лише його градієнтами та кривиною, що мають фізичне значення. Це узгоджується з принципом калібрування: фізика інваріантна під локальними обертаннями $R(x) \to S(x)R(x)$, де $S(x)$ є довільним просторово-змінним ротором.

\subsection{Гравітаційні Хвилі та Подвійні Системи}

Розглянемо два компактні об'єкти (нейтронні зірки або чорні діри) на близькій орбіті. У загальній відносності їх орбітальний рух генерує гравітаційні хвилі---брижі в метриці, що поширюються зі швидкістю світла. У теорії роторного поля ці хвилі відповідають поширюваним збуренням у $B(x,t)$.

Коли об'єкти мають значний кутовий момент (спін), орбітальна площина прецесує. Ця прецесія вводить модуляцію сигналу гравітаційної хвилі. Теорія роторного поля передбачає, що ця модуляція демонструє додаткову структуру: бічні смуги на частотах $\omega \pm \Omega_{\mathrm{prec}}$, де $\omega$ є орбітальною частотою, а $\Omega_{\mathrm{prec}}$ швидкістю прецесії. Амплітуда цих бічних смуг залежить від зв'язку між орбітальним кутовим моментом та спінами об'єктів---величиною, безпосередньо пов'язаною з конфігурацією бівекторного поля.

Недавні детекції гравітаційних хвиль LIGO та Virgo надають дані, з якими можна протестувати це передбачення. Системи з високим ефективним параметром спіну $\chi_{\mathrm{eff}}$ повинні демонструвати сильніші сигнатури бічних смуг, якщо опис роторного поля є правильним.

\subsection{Квантова Інтерференція та Бівекторна Фаза}

У експерименті з подвійною щілиною хвильова функція електрона проходить через обидві щілини та інтерферує сама з собою, створюючи смуги на екрані. Стандартна квантова механіка приписує це принципу суперпозиції. У картині роторного поля електрон асоціюється з областю когерентної бівекторної осциляції. Коли ця область розділяється (на щілинах) та рекомбінується, відносна різниця фаз---визначена лінійним інтегралом $\int B \cdot \dd x$ вздовж двох шляхів---створює конструктивну або деструктивну інтерференцію.

Ця інтерпретація свідчить, що квантова інтерференція є геометричним ефектом, подібним до паралельного транспорту у викривленому просторі. Фаза, накопичена вздовж шляху, залежить від конфігурації бівекторного поля, так само як в ефекті Ааронова-Бома електромагнітний потенціал створює спостережувані зсуви фази.

% ======================================================================
\section{Обговорення}
\label{sec:discussion}

\subsection{Єдність Фізичного Закону}

Ми показали, що постулюючи бівекторне поле $B(x,t)$ як фундаментальне, як кривина простору-часу (загальна відносність), так і квантова поведінка матерії виникають як різні аспекти єдиної геометричної структури. Це свідчить, що видима дихотомія між гравітацією та квантовою механікою може бути артефактом нашого математичного формулювання, а не глибокою фізичною реальністю.

Ейнштейн шукав уніфіковану теорію поля протягом останньої частини своєї кар'єри, намагаючись геометризувати не лише гравітацію, але й електромагнетизм. Ця теорія розширює цю програму: роторне поле, через свої бівекторні компоненти, природно охоплює електромагнітні явища (які також описуються бівекторами---тензор Фарадея $F_{\mu\nu}$), а також гравітаційні та квантові ефекти.

\subsection{Обмеження та Відкриті Питання}

Кілька питань залишаються без відповіді в цьому початковому формулюванні:

\textbf{Константи $\kappa$ та $\rho$.} Вони з'являються як феноменологічні параметри в діях \eqref{eq:palatini-action} та \eqref{eq:rotor-action}. Константа Ейнштейна $\kappa = 8\pi G/c^4$ визначається експериментально, але роторний зв'язок $\rho$ та потенціал $V(R,\Phi)$ залишаються невизначеними. В ідеалі вони повинні виводитися з глибших принципів, можливо, пов'язаних зі структурою самої геометричної алгебри або з вимогами симетрії.

\textbf{Режим сильного поля.} Наше виведення рівнянь Ейнштейна з динаміки роторного поля спиралось на пертурбативне розкладання в $\kappa B^2$. Поблизу сингулярностей чорних дір або в момент Великого Вибуху це розкладання руйнується. Потрібна повністю непертурбативна обробка.

\textbf{Космологія.} Які граничні умови слід накласти на бівекторне поле на початку та в кінці космічного часу? Чи пропонує опис роторного поля інсайти щодо початкової сингулярності, темної енергії або великомасштабної структури Всесвіту?

\textbf{Експериментальна точність.} Передбачені бічні смуги в сигналах гравітаційних хвиль є малими корекціями до домінуючої форми хвилі. Їх виявлення вимагає високих відношень сигнал-шум та ретельного статистичного аналізу. Майбутні обсерваторії гравітаційних хвиль з більшою чутливістю забезпечать більш остаточні тести.

\subsection{Філософські Імплікації}

Якщо геометрія простору-часу та квантова механіка обидві виникають з роторного поля, яким є онтологічний статус цього поля? Чи є воно фізичною субстанцією, що заповнює простір, чи лише математичним пристроєм для обчислення спостережуваних?

Можна стверджувати, слідуючи власному погляду Ейнштейна, що поле є таким же реальним, як будь-яке спостережуване явище. Електромагнітне поле, спочатку введене Максвеллом як теоретична конструкція, тепер розглядається як фізично існуюче, здатне переносити енергію та імпульс. Подібно, роторне поле може розглядатися як фундаментальна складова реальності.

Альтернативно, можна прийняти інструменталістську позицію: роторне поле є корисним формалізмом, і питання про його "реальність" є метафізичним, а не фізичним. Важливо те, що воно правильно передбачає експериментальні результати.

% ======================================================================
\section{Заключні Зауваження}
\label{sec:conclusion}

У цій статті ми розробили геометричну основу для об'єднання загальної відносності та квантової механіки через концепцію роторного поля. Основні результати:

\begin{enumerate}
  \item Бівекторне поле $B(x,t)$, визначене в геометричній алгебрі простору-часу, генерує метричний тензор через конструкцію тетради $e_a = R\gamma_a\widetilde{R}$ з $R = \exp(\frac{1}{2}B)$.
  \item З варіаційного принципу Палатіні \eqref{eq:palatini-action} ми \textbf{вивели} (а не постулювали) точні \textbf{рівняння поля Ейнштейна} \eqref{eq:einstein-final}:
  \begin{equation*}
  R_{\mu\nu} - \frac{1}{2}g_{\mu\nu}R = \frac{8\pi G}{c^4} T_{\mu\nu}.
  \end{equation*}
  \item З динаміки роторного поля \eqref{eq:rotor-dynamics} ми \textbf{вивели} (а не постулювали) точне \textbf{рівняння Дірака} \eqref{eq:dirac}:
  \begin{equation*}
  (\ii\hbar\gamma^\mu \partial_\mu - mc)\psi = 0.
  \end{equation*}
  \item Ньютонівська границя дає рівняння Пуассона \eqref{eq:poisson}: $\nabla^2\Phi = 4\pi G\rho_{\mathrm{mass}}$.
  \item Теорія передбачає спостережувані відхилення в системах з сильним обертальним зв'язком, що тестуються через спостереження гравітаційних хвиль.
\end{enumerate}

Основа знаходиться на початковій стадії. Залишається багато роботи для розвитку повних імплікацій, обчислення детальних передбачень та порівняння з експериментом. Якщо майбутні спостереження підтвердять відмітні сигнатури роторного поля---особливо в даних гравітаційних хвиль від обертових подвійних систем---це забезпечить сильні докази для підходу геометричної алгебри до квантової гравітації.

Незалежно від того, чи виявиться гіпотеза роторного поля правильною в деталях, ця вправа демонструє цінність пошуку уніфікації через геометрію. Глибокий інсайт Ейнштейна---що гравітація є кривиною простору-часу---може простягатися далі, ніж він уявляв, охоплюючи не лише великомасштабну структуру космосу, але й квантову область атомів та частинок.

\medskip
\noindent\textit{Автор сподівається, що ця робота, хоч і недосконала, може внести вклад у триваючий пошук уніфікованого розуміння фізичного закону.}

% ======================================================================
\ifack
\section*{Подяки}
Автор вдячний піонерській роботі Девіда Хестенса та колег у розробці геометричної алгебри як мови для фізики. Подяка належить колабораціям LIGO та Virgo за надання даних гравітаційних хвиль у публічний доступ. Ця робота проводилась незалежно без зовнішнього фінансування.
\fi

% ======================================================================
\appendix
\section{Обчислення Тензора Річчі}
\label{app:ricci}

Ми виводимо зв'язок між бівекторним полем $B(x)$ та кривиною Річчі $R_{\mu\nu}$.

Починаючи з індукованої метрики
\begin{equation}
g_{\mu\nu} = \eta_{\mu\nu} + \kappa \langle e_\mu B^2 e_\nu \rangle_0,
\end{equation}
символи Крістофеля до першого порядку в $\kappa$ є
\begin{equation}
\Gamma^\lambda_{\mu\nu} = \frac{\kappa}{2} \eta^{\lambda\rho} \left[\partial_\mu \langle e_\nu B^2 e_\rho \rangle_0 + \partial_\nu \langle e_\mu B^2 e_\rho \rangle_0 - \partial_\rho \langle e_\mu B^2 e_\nu \rangle_0\right].
\end{equation}

Використовуючи правило добутку та співвідношення $\partial_\mu (B^2) = 2 B \partial_\mu B$, тензор Рімана до провідного порядку стає
\begin{equation}
R^\rho_{\sigma\mu\nu} = \kappa^2 \left[\partial_\mu \langle \partial_\sigma B \, e_\nu B \rangle_0 - \partial_\nu \langle \partial_\sigma B \, e_\mu B \rangle_0\right] + O(\kappa^3).
\end{equation}

Згортаючи для отримання тензора Річчі:
\begin{equation}
R_{\mu\nu} = R^\lambda_{\mu\lambda\nu} = \kappa^2 \langle \partial_\mu B \, \partial_\nu B \rangle_0 + O(\kappa^3).
\end{equation}

Це явно показує, як кривина виникає з варіацій бівекторного поля.

% ======================================================================
% --------------------- Bibliography -----------------

\begin{thebibliography}{9}

\bibitem{Einstein1916}
A.~Einstein.
\newblock Основа загальної теорії відносності.
\newblock \emph{Annalen der Physik}, 354(7):769--822, 1916.

\bibitem{Dirac1928}
P.~A.~M.~Dirac.
\newblock Квантова теорія електрона.
\newblock \emph{Proceedings of the Royal Society of London A}, 117(778):610--624, 1928.

\bibitem{Clifford1878}
W.~K.~Clifford.
\newblock Застосування розширеної алгебри Грассмана.
\newblock \emph{American Journal of Mathematics}, 1(4):350--358, 1878.

\bibitem{Hestenes1966}
D.~Hestenes.
\newblock \emph{Алгебра простору-часу}.
\newblock Gordon and Breach, New York, 1966.

\bibitem{Hestenes1984}
D.~Hestenes and G.~Sobczyk.
\newblock \emph{Від алгебри Кліффорда до геометричного числення: Уніфікована мова для математики та фізики}.
\newblock D. Reidel Publishing Company, Dordrecht, 1984.

\bibitem{DoranLasenby}
C.~Doran and A.~Lasenby.
\newblock \emph{Геометрична алгебра для фізиків}.
\newblock Cambridge University Press, Cambridge, 2003.

\bibitem{Lasenby1998}
A.~Lasenby, C.~Doran, and S.~Gull.
\newblock Гравітація, калібрувальні теорії та геометрична алгебра.
\newblock \emph{Philosophical Transactions of the Royal Society A}, 356(1737):487--582, 1998.

\bibitem{LIGO2016}
B.~P.~Abbott et al. (LIGO Scientific Collaboration and Virgo Collaboration).
\newblock Спостереження гравітаційних хвиль від злиття подвійної чорної діри.
\newblock \emph{Physical Review Letters}, 116(6):061102, 2016.

\end{thebibliography}

\end{document}
