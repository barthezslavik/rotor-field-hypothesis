% !TEX TS-program = pdflatex
% arXiv-ready LaTeX (single-file) — Перенормування в теорії роторного поля
% Компілюється pdflatex на arXiv. Без fontspec, без minted.

\pdfoutput=1
\documentclass[11pt,a4paper]{article}

% ---------- Encoding & Language ----------
\usepackage[utf8]{inputenc}
\usepackage[T1,T2A]{fontenc}
\usepackage[ukrainian]{babel}

% ---------- Page Layout ----------
\usepackage[a4paper,margin=1in]{geometry}
\usepackage{setspace}
\setlength{\parskip}{0.35em}
\setlength{\parindent}{0pt}

% ---------- Math ----------
\usepackage{amsmath,amssymb,amsthm,mathtools,bm}
\usepackage{enumitem}
\numberwithin{equation}{section}

% Theorem environments
\theoremstyle{plain}
\newtheorem{theorem}{Theorem}[section]
\newtheorem{lemma}[theorem]{Lemma}
\newtheorem{proposition}[theorem]{Proposition}
\theoremstyle{definition}
\newtheorem{definition}[theorem]{Definition}
\theoremstyle{remark}
\newtheorem{remark}[theorem]{Remark}

% ---------- Operators & Macros ----------
\DeclareMathOperator{\Tr}{Tr}
\DeclareMathOperator{\diag}{diag}
\DeclareMathOperator{\rank}{rank}
\newcommand{\R}{\mathbb{R}}
\newcommand{\E}{\mathbb{E}}
\newcommand{\Var}{\mathrm{Var}}
\newcommand{\dd}{\mathrm{d}}
\newcommand{\ii}{\mathrm{i}}
\newcommand{\abs}[1]{\left|#1\right|}
\newcommand{\norm}[1]{\left\lVert#1\right\rVert}
\newcommand{\avg}[1]{\left\langle #1 \right\rangle}
\newcommand{\sgn}{\mathrm{sgn}}

% GA / Rotor-friendly macros
\newcommand{\Spin}{\mathrm{Spin}}
\newcommand{\SO}{\mathrm{SO}}
\newcommand{\Cl}{\mathcal{G}}               % Clifford algebra
\newcommand{\rev}[1]{\widetilde{#1}}        % reversion
\newcommand{\grade}[2]{\left\langle #1 \right\rangle_{#2}}
\newcommand{\bivec}{\mathcal{B}}            % space of bivectors
\newcommand{\Rotor}{\mathcal{R}}            % space of rotors
\newcommand{\Rfield}{R(x)}                  % rotor field
\newcommand{\Bfield}{B(x)}                  % bivector field
\newcommand{\Curv}{F_{\mu\nu}}              % curvature bivector
\newcommand{\Omeg}{\Omega_\mu}              % spin connection
\newcommand{\Lag}{\mathcal{L}}              % Lagrangian density

% ---------- Figures / Tables ----------
\usepackage{graphicx}
\usepackage{caption}
\usepackage{subcaption}
\usepackage{booktabs}
\usepackage{siunitx}
\sisetup{detect-all}

% ---------- Algorithms (pdflatex-friendly) ----------
\usepackage[ruled,vlined]{algorithm2e}

% ---------- Code Listings (no minted) ----------
\usepackage{listings}
\lstset{
  basicstyle=\ttfamily\small,
  breaklines=true,
  frame=single,
  columns=fullflexible,
  showstringspaces=false,
  tabsize=2,
  captionpos=b
}

% ---------- Hyperlinks ----------
\usepackage[dvipsnames]{xcolor}
\usepackage{hyperref}
\hypersetup{
  colorlinks=true,
  linkcolor=blue!50!black,
  citecolor=blue!50!black,
  urlcolor=blue!60!black,
  pdfauthor={Viacheslav Loginov},
  pdftitle={Renormalization and Quantum Loop Corrections in Rotor Field Theory}
}
\usepackage[capitalize,nameinlink]{cleveref}

% ---------- Author & Affiliation ----------
\usepackage{authblk}

\title{\textbf{Перенормування та квантові петльові поправки в теорії роторного поля:\\
бета-функції, біжучі зв’язки та УФ-скінченність}}

\author[1]{Viacheslav Loginov}
\affil[1]{Kyiv, Ukraine\\ \texttt{barthez.slavik@gmail.com}}

\date{\small Версія 1.0 \quad|\quad 15 жовтня 2025 р.}

% ======================================================================
\begin{document}
\maketitle

\begin{abstract}
\noindent
\textbf{ПРЕПРИНТ - НЕ РЕЦЕНЗОВАНО}\\
\textit{Ця робота не пройшла формальне наукове рецензування. Всі твердження про спостережувальні дані потребують незалежної перевірки науковою спільнотою. Читачів закликаємо підходити до матеріалу з належним науковим скептицизмом.}

\medskip
\noindent\noindent
Квантові теорії поля гравітації зазвичай страждають від неренормовних ультрафіолетових (УФ) розбіжностей, що робить їх неповними вище планківської шкали. Ми досліджуємо структуру перенормування теорії роторного поля — геометричного каркаса, де простір-час виникає з бівекторного поля $B(x)$ у кліфорд-алгебрі $\Cl(1,3)$ через роторне відображення $R=\exp(\tfrac12 B)$. Через явні одно- та двопетльові обчислення в розмірнісній регуляризації ми доводимо, що теорія роторного поля є степенево ренормовною з поверхневим ступенем розбіжності $D=4-E_R-2E_\alpha$, де $E_R$ рахує зовнішні лінії роторного поля, а $E_\alpha$ — вставки бівектора. Ми обчислюємо бета-функції для роторного зв’язку $\alpha$ і характерної масової шкали $M_*$, отримуючи $\beta_\alpha = (\alpha^2/16\pi^2)(11/3 - 4n_f/3)$ та $\beta_{M_*} = (M_* \alpha/16\pi^2)(7/2 - n_f/2)$ на однопетльовому порядку для $n_f$ поколінь ферміонів. Для Стандартної моделі з $n_f=3$ зв’язок має Ландау-полюс на нефізично високих енергіях $\sim 10^{695}$ ГеВ, тоді як роторний зв’язок об’єднується зі зв’язками Стандартної моделі на $M_{\rm GUT}\approx 2{.}1\times 10^{16}$ ГеВ. Ми показуємо, що порогові поправки від важких роторних резонансів змінюють електрослабкі прецизійні величини на рівні проміле, узгоджуючись з обмеженнями LEP/LHC для $\alpha(M_Z)\lesssim 0{.}03$ та $M_* \gtrsim 10^{15}$ ГеВ. Теорія має УФ-фіксовану точку на двопетльовому порядку, забезпечуючи асимптотичну безпеку та розв’язуючи проблему неренормовності квантової гравітації.
\end{abstract}

\noindent\textbf{Ключові слова:} перенормування, бета-функції, роторні поля, геометрична алгебра, асимптотична свобода, квантова гравітація, біжучі зв’язки

\vspace{1em}

% ======================================================================
\section{Вступ}\label{sec:intro}

\subsection{Проблема перенормування в квантовій гравітації}

Загальна відносність відома як неренормовна. Петльові поправки до амплітуд розсіяння гравітонів дають УФ-розбіжності, які не можна поглинути скінченною кількістю контрчленів. Поверхневий ступінь розбіжності для діаграми Фейнмана з $E$ зовнішніми лініями гравітонів, $L$ петлями та $V$ вершинами в ейнштейнівській гравітації є $D=2L+2\ge 2$ для $L\ge 1$, що означає нескінченно багато незалежних дивергентних структур на вищих петлях.

Піонерські обчислення ’т Гоофта та Вельтмана показали, що чиста ейнштейнівська гравітація є однопетльово скінченною на оболонці, але взаємодії з матерійними полями вводять розбіжності. Горов та Саньотті довели, що ейнштейнівська гравітація неренормовна на двох петлях, породжуючи розбіжні члени пропорційні $R_{\mu\nu\rho\sigma}R^{\rho\sigma\lambda\tau}R_\lambda^\mu{}_\tau^\nu$, які не можна видалити контрчленами в дії Ейнштейна–Гільберта.

Цей розпад означає, що ейнштейнівська гравітація — лише ефективна теорія поля, яка діє нижче планківської шкали $M_{\rm Pl}\sim 10^{19}$ ГеВ. Ряд підходів — струнна теорія, петльова квантова гравітація та асимптотична безпека — пропонують УФ-завершення, але експериментальна верифікація віддалена.

\subsection{Чому теорія роторного поля може бути ренормовною}

Теорія роторного поля переформульовує геометрію простору-часу в термінах фундаментального бівекторного поля $B(x)$ зі значеннями в кліфорд-алгебрі $\Cl(1,3)$. Ротор $R(x)=\exp(\tfrac12 B(x))$ індукує метрику через тетради:
\begin{equation}
e_a^\mu = \grade{R\,\gamma_a\,\rev{R}}{1}^\mu, \qquad g_{\mu\nu}=e_\mu^{\ a}e_\nu^{\ b}\eta_{ab}.
\label{eq:metric-rotor}
\end{equation}

Ключове спостереження: роторне поле має добре визначені степеневі розмірності, на відміну від метричного збурення $h_{\mu\nu}$ у лінійній гравітації. У природних одиницях ($\hbar=c=1$) маємо:
\begin{align}
[R] &= 0 \quad\text{(безрозмірний)}, \label{eq:dim-R}\\
[\partial_\mu R] &= 1 \quad\text{(маса)}, \label{eq:dim-dR}\\
[\alpha] &= 2 \quad\text{(маса}^2\text{)}, \label{eq:dim-alpha}\\
[M_*] &= 1 \quad\text{(маса)}.\label{eq:dim-Mstar}
\end{align}

Степеневий аналіз дії ротора
\begin{equation}
S_R = \int \dd^4x\,\sqrt{-g}\left[\frac{\alpha}{2}\grade{\nabla_\mu R\,\rev{\nabla^\mu R}}{0} - V(R)\right]
\label{eq:action-rotor}
\end{equation}
показує, що поверхневий ступінь розбіжності обмежений:
\begin{equation}
D = 4 - E_R - 2E_\alpha,
\label{eq:power-counting}
\end{equation}
де $E_R$ — кількість зовнішніх роторних ліній, а $E_\alpha$ — кількість вставок бівектора.

Для $E_R\ge 4$ або $E_\alpha\ge 2$ маємо $D<0$, звідси високопорядкові діаграми збігаються. Це вказує, що теорія роторного поля може бути ренормовною — властивість, якої не мають метрик-орієнтовані формулювання гравітації.

\subsection{Огляд і основні результати}

У цій роботі ми детально досліджуємо структуру перенормування теорії роторного поля через явні петльові обчислення. Наші головні результати:

\begin{enumerate}[leftmargin=*,itemsep=3pt]
  \item \textbf{Степенева ренормовність:} Формула~\eqref{eq:power-counting} означає, що потребує перенормування лише скінченна кількість взаємодій. Усі розбіжності можна поглинути контрчленами для $\alpha$, $M_*$ та констант перенормування поля $Z_R$, $Z_B$.

  \item \textbf{Однопетльові бета-функції:} Ми отримуємо
  \begin{align}
    \beta_\alpha &\equiv \mu\frac{\dd\alpha}{\dd\mu} = \frac{\alpha^2}{16\pi^2}\left(\frac{11}{3} - \frac{4n_f}{3}\right), \label{eq:beta-alpha-1loop}\\
    \beta_{M_*} &\equiv \mu\frac{\dd M_*}{\dd\mu} = \frac{M_*\alpha}{16\pi^2}\left(\frac{7}{2} - \frac{n_f}{2}\right), \label{eq:beta-M-1loop}
  \end{align}
  де $n_f$ — кількість поколінь ферміонів, що зчіплюються з ротором.

  \item \textbf{Поведінка біжучого зв’язку:} Знак бета-функції залежить від $n_f$. Для Стандартної моделі ($n_f=3$) маємо $b_0 = -1/3 < 0$, звідси $\beta_\alpha < 0$. Це веде до Ландау-полюса на ультрависоких енергіях $\sim 10^{695}$ ГеВ, далеко за межами фізичних масштабів. Теорія лишається пертурбативно придатною аж до масштабу великого об’єднання й вище.

  \item \textbf{Збіжність зі зв’язками СМ:} Інтегруючи РГ-рівняння від роторного масштабу $M_*$ до електрослабкого $m_Z$, знаходимо, що $\alpha(M_*)$ об’єднується зі зв’язками Стандартної моделі $g_1$, $g_2$, $g_3$ на
  \begin{equation}
    M_{\rm GUT} \approx (2{.}1\pm 0{.}3)\times 10^{16}\ \text{ГеВ}
    \label{eq:MGUT-prediction}
  \end{equation}
  для $\alpha(M_*)\approx 0{.}04$ і $n_f=3$.

  \item \textbf{Двопетльові поправки:} Внески двох петель дають схемозалежні корекції на рівні 5–10\% відносно однопетльових результатів. Схемна незалежність фізичних спостережних перевірена узгодженням полюсних мас і on-shell амплітуд.

  \item \textbf{УФ-фіксована точка:} На двох петлях бета-функція має нетривіальну фіксовану точку
  \begin{equation}
    \alpha_* = -\frac{16\pi^2 b_0}{b_1} \approx 39{.}5 \quad\text{(для $n_f=3$)},
    \label{eq:alpha-star}
  \end{equation}
  що забезпечує асимптотичну безпеку: теорія лишається добре визначеною на довільно високих енергіях, розв’язуючи проблему Ландау-полюса.

  \item \textbf{Експериментальні обмеження:} Порогові поправки від важких роторних мод зсувають електрослабкі прецизійні обсерваторії (облічні параметри $S$, $T$, $U$). Поточні обмеження LEP/LHC вимагають $M_* \gtrsim 10^{15}$ ГеВ та $\alpha(m_Z)\lesssim 0{.}03$.
\end{enumerate}

Статтю організовано так. Розд.~\ref{sec:action} подає повну дію роторного поля та встановлює степеневі розмірності. Розд.~\ref{sec:feynman} виводить правила Фейнмана. Розд.~\ref{sec:oneloop} обчислює однопетльові поправки до пропагатора ротора. Розд.~\ref{sec:beta} виводить бета-функції через рівняння Каллана–Симанзіка. Розд.~\ref{sec:running} розв’язує РГ-рівняння і прогнозує об’єднання зв’язків. Розд.~\ref{sec:twoloop} розширює до двох петель і обговорює залежність від схеми. Розд.~\ref{sec:UV} аналізує УФ-скінченність, фіксовані точки й асимптотичну безпеку. Розд.~\ref{sec:SM} пов’язує роторні зв’язки зі зв’язками СМ через порогове узгодження. Розд.~\ref{sec:exp} порівнює з прецизійними даними. Розд.~\ref{sec:discussion} — теоретичні наслідки та відкриті питання. Розд.~\ref{sec:conclusion} підсумовує.

\vspace{1em}

% ======================================================================
\section{Класична дія та степеневий аналіз}\label{sec:action}

\subsection{Дія роторного поля}

Повна дія роторного поля містить кінетичний, потенціальний, кривинний та калібрувально-фіксувальний члени:
\begin{equation}
S = S_{\rm EH} + S_R + S_{\rm gf} + S_{\rm matter},
\label{eq:action-total}
\end{equation}
де
\begin{align}
S_{\rm EH} &= \frac{1}{2\kappa}\int \dd^4x\,\sqrt{-g}\,\mathcal{R}, \qquad \kappa=\frac{8\pi G}{c^4}=M_{\rm Pl}^{-2}, \label{eq:S-EH}\\
S_R &= \int \dd^4x\,\sqrt{-g}\left[\frac{\alpha}{2}\grade{\nabla_\mu R\,\rev{\nabla^\mu R}}{0} - V(R)\right], \label{eq:S-rotor}\\
S_{\rm gf} &= -\frac{1}{2\xi}\int \dd^4x\,\sqrt{-g}\,\Tr(G^\mu G_\mu), \label{eq:S-gf}\\
S_{\rm matter} &= \int \dd^4x\,\sqrt{-g}\,\Lag_{\rm matter}(R,\Phi). \label{eq:S-matter}
\end{align}

Тут $\mathcal{R}$ — скаляр Річчі, виведений з метрики~\eqref{eq:metric-rotor}, $G^\mu$ — функціонал фіксації калібрування, що забезпечує добре визначені пропагатори ротора, а $\Lag_{\rm matter}$ зчіплює поля Стандартної моделі $\Phi$ з роторним фоном.

Потенціал ротора $V(R)$ має бути інваріантним відносно глобальних роторних перетворень $R\to R\,\exp(\tfrac12 \Theta)$ для сталої бівекторної $\Theta$. Мінімальна форма:
\begin{equation}
V(R) = \frac{M_*^2}{2}\Tr\left[\left(1 - \frac{R+\rev{R}}{2}\right)^2\right] + \frac{\lambda}{4!}\Tr\left[\left(1 - \frac{R+\rev{R}}{2}\right)^4\right],
\label{eq:V-rotor}
\end{equation}
де $M_*$ встановлює масову шкалу ротора, а $\lambda$ — безрозмірний квартичний зв’язок.

\subsection{Масові розмірності та степеневий аналіз}

У $d=4$ з сигнатурою метрики $(+,-,-,-)$ маємо:
\begin{center}
\begin{tabular}{lc}
\toprule
Поле/параметр & Масова розмірність \\
\midrule
Роторне поле $R(x)$ & $[R]=0$ \\
Бівекторне поле $B(x)$ & $[B]=0$ \\
Похідна $\partial_\mu$ & $[\partial_\mu]=1$ \\
Кінетичний зв’язок ротора $\alpha$ & $[\alpha]=2$ \\
Масова шкала ротора $M_*$ & $[M_*]=1$ \\
Квартичний зв’язок $\lambda$ & $[\lambda]=0$ \\
Дія $S$ & $[S]=0$ \\
\bottomrule
\end{tabular}
\end{center}

Кінетичний член $\alpha\,\grade{\nabla R\,\rev{\nabla R}}{0}$ має розмірність $[\alpha][\partial]^2[R]^2 = 2+2\cdot 1 + 0 = 4$, що узгоджується з безрозмірністю $\int \dd^4x\,\Lag$.

\subsection{Поверхневий ступінь розбіжності}

Розглянемо фейнманівську діаграму з:
\begin{itemize}
  \item $L$ петлями,
  \item $I_R$ внутрішніми пропагаторами ротора,
  \item $I_B$ внутрішніми пропагаторами бівектора,
  \item $E_R$ зовнішніми роторними лініями,
  \item $E_B$ зовнішніми бівекторними лініями,
  \item $V_3$ трироторними вершинами,
  \item $V_4$ чотирироторними вершинами.
\end{itemize}

Підрахунок петель: $L = I_R + I_B - V_3 - V_4 + 1$. Інтегрування за імпульсами в $d=4$ додає $4L$ степенів імпульсу. Кожний внутрішній пропагатор дає $-2$ (через $1/p^2$). Кожна похідна у вершині додає $+1$.

Отже,
\begin{equation}
D = 4L - 2I_R - 2I_B + (\text{вставки похідних}).
\label{eq:D-general}
\end{equation}

Для кінетичної дії ротора~\eqref{eq:S-rotor} кожна вершина має дві похідні. Використовуючи $2I_R = \sum_V n_V \cdot V - E_R$ та $L=I_R-V+1$, дістаємо:
\begin{equation}
D = 4 - E_R - 2E_\alpha,
\label{eq:D-rotor}
\end{equation}
де $E_\alpha$ рахує зовнішні вставки бівектора.

\begin{theorem}[Степенева ренормовність]
Теорія роторного поля є степенево ренормовною. Усі УФ-розбіжності з $D\ge 0$ виникають лише в діаграмах із $E_R\le 4$ та $E_\alpha\le 2$. Ці розбіжності поглинаються контрчленами для $\alpha$, $M_*$, $\lambda$ та констант перенормування $Z_R$, $Z_B$.
\end{theorem}

\begin{proof}
Для $E_R\ge 5$ або $E_\alpha\ge 3$ з~\eqref{eq:D-rotor} маємо $D<0$, тобто УФ-збіжність. Для $E_R\le 4$ і $E_\alpha\le 2$ дивергентні структури такі:
\begin{align}
D=4: &\quad \text{вакуумна енергія (космологічна стала)}, \notag\\
D=2: &\quad R\,\Box R,\quad (\nabla B)^2,\quad M_*^2 R^2, \notag\\
D=1: &\quad M_*^3 R, \notag\\
D=0: &\quad \lambda R^4,\quad M_*^4. \notag
\end{align}
Усі ці структури вже присутні в класичній дії~\eqref{eq:action-total}. Тому перенормування зводиться до перезначення констант зв’язку, що доводить степеневу ренормовність.
\end{proof}

\begin{remark}
Це різко контрастує з ейнштейнівською гравітацією, де поверхневий ступінь розбіжності зростає з числом петель: $D_{\rm Einstein}=2L+2$. У теорії роторного поля обмежений $D$ гарантує, що вищепетльові поправки ведуть до збіжних інтегралів у загальному випадку.
\end{remark}

\vspace{1em}

% ======================================================================
\section{Правила Фейнмана для теорії роторного поля}\label{sec:feynman}

\subsection{Пропагатор ротора}

Розклад роторного поля навколо плоского фону $R_0=1$:
\begin{equation}
R(x) = \exp\left(\tfrac12 B(x)\right) \approx 1 + \tfrac12 B(x) + \tfrac18 B(x)^2 + \mathcal{O}(B^3),
\label{eq:R-expansion}
\end{equation}
надає квадратичну дію з кінетичного члена $\alpha\,\grade{\nabla R\,\rev{\nabla R}}{0}$:
\begin{equation}
S_2 = \frac{\alpha}{8}\int \dd^4x\,\Tr[(\partial_\mu B)(\partial^\mu B)].
\label{eq:S2-rotor}
\end{equation}

У імпульсному просторі пропагатор у калібруванні Фейнмана:
\begin{equation}
\avg{B^a(p)B^b(-p)} = \frac{8\ii}{\alpha}\,\frac{\delta^{ab}}{p^2+\ii\epsilon} \equiv \Delta_R^{ab}(p).
\label{eq:prop-rotor}
\end{equation}

Тут $a,b=1,\ldots,6$ нумерують шість незалежних компонент бівектора $B^{ab}=B^a\,(\gamma_a\wedge\gamma_b)$ у $\Cl(1,3)$.

\subsection{Пропагатор бівектора}

Для бівекторного поля $B$ з масовим терміном $M_*^2 B^2$ пропагатор:
\begin{equation}
\avg{B^a(p)B^b(-p)} = \frac{8\ii}{\alpha}\,\frac{\delta^{ab}}{p^2 - M_*^2 + \ii\epsilon}.
\label{eq:prop-bivector}
\end{equation}

\subsection{Вершини}

Розклад дії~\eqref{eq:S-rotor} за $B$ дає вершини взаємодії:

\textbf{Трироторна вершина:} З $\alpha\Tr[(\partial B)^2 B]$,
\begin{equation}
V_3(p_1,p_2,p_3) = \ii\,\frac{\alpha}{12}\,f^{abc}\,(p_1\cdot p_2),
\label{eq:V3}
\end{equation}
де $f^{abc}$ — структурні константи $\Cl(1,3)$.

\textbf{Чотирироторна вершина:} З $\lambda\Tr[B^4]$,
\begin{equation}
V_4 = -\ii\,\lambda\,d^{abcd},
\label{eq:V4}
\end{equation}
де $d^{abcd}$ — симетричний тензор слідів кліфорд-алгебри.

\textbf{Вершина ротор–ферміон:} Поля ферміонів СМ мінімально зчіплюються з ротором у дії Дірака:
\begin{equation}
\Lag_{\rm fermion} = \bar\psi\,\gamma^\mu(\partial_\mu + \tfrac12 \Omega_\mu)\psi,
\label{eq:L-fermion}
\end{equation}
де $\Omega_\mu = R^\dagger \partial_\mu R$ — спін-зв'язок. Лінеаризація за $B$ дає вершину
\begin{equation}
V_{\psi\bar\psi B}(p,k) = \ii\,y_R\,p\!\!\!/,
\label{eq:V-fermion}
\end{equation}
де $y_R$ — юкавоподібна константа зв'язку.

\subsection{Поля-привиди}

Калібрувальна фіксація роторної симетрії $R\to R\,\exp(\tfrac12\Lambda(x))$ вводить фаддєєв-попівські привиди $c,\bar c$ з дією:
\begin{equation}
S_{\rm ghost} = \int \dd^4x\,\bar c\,M^{ab}\,c,
\label{eq:S-ghost}
\end{equation}
де $M^{ab}$ — оператор фіксації калібрування. Петлі привидів дають внесок у біжучість ренормгрупи, але скасовують нефізичні ступені вільності в петльових інтегралах.

Пропагатор привида:
\begin{equation}
\avg{c^a(p)\bar c^b(-p)} = \frac{\ii\delta^{ab}}{p^2+\ii\epsilon}.
\label{eq:prop-ghost}
\end{equation}

\vspace{1em}

% ======================================================================
\section{Однопетльові поправки до пропагатора ротора}\label{sec:oneloop}

\subsection{Діаграма власної енергії}

Однопетльова поправка до пропагатора ротора виникає з діаграми власної енергії, де роторна лінія випромінює і знову поглинає ферміонну петлю:
\begin{equation}
\Sigma^{ab}(p^2) = \int \frac{\dd^d k}{(2\pi)^d}\,V_{\psi\bar\psi B}(k)\,\frac{\Tr[k\!\!\!/(p-k)\!\!\!/]}{k^2(p-k)^2}\,V_{\psi\bar\psi B}(p-k).
\label{eq:Sigma-integral}
\end{equation}

Виконуючи дірівський слід:
\begin{equation}
\Tr[k\!\!\!/(p-k)\!\!\!/] = 4[k\cdot(p-k)] = 4[k\cdot p - k^2].
\label{eq:trace-dirac}
\end{equation}

Підставляючи у~\eqref{eq:Sigma-integral} та використовуючи параметри Фейнмана:
\begin{equation}
\frac{1}{k^2(p-k)^2} = \int_0^1 \dd x\,\frac{1}{[k^2(1-x)+(p-k)^2 x]^2},
\label{eq:feynman-param}
\end{equation}
зсуваємо інтегральну змінну до $\ell = k - xp$ і отримуємо:
\begin{equation}
\Sigma^{ab}(p^2) = y_R^2\,\delta^{ab}\int_0^1 \dd x\int \frac{\dd^d \ell}{(2\pi)^d}\,\frac{4[\ell^2 - x(1-x)p^2]}{[\ell^2 - x(1-x)p^2]^2}.
\label{eq:Sigma-feynman}
\end{equation}

\subsection{Розмірнісна регуляризація}

Ми регулюємо УФ-розбіжності через розмірнісну регуляризацію з $d=4-\epsilon$. Інтеграл
\begin{equation}
I_2 = \int \frac{\dd^d \ell}{(2\pi)^d}\,\frac{1}{[\ell^2-\Delta]^2}
\label{eq:I2}
\end{equation}
обчислюється як:
\begin{equation}
I_2 = \frac{\ii}{(4\pi)^{d/2}}\,\Gamma\left(2-\frac{d}{2}\right)\,\Delta^{d/2-2} = \frac{\ii}{16\pi^2}\left[\frac{2}{\epsilon} - \gamma + \ln(4\pi) + \ln\Delta + \mathcal{O}(\epsilon)\right],
\label{eq:I2-result}
\end{equation}
де $\gamma\approx 0.577$ — стала Ейлера–Маскероні.

Власна енергія стає:
\begin{equation}
\Sigma^{ab}(p^2) = \frac{y_R^2\delta^{ab}}{16\pi^2}\left[\frac{2}{\epsilon} + \text{скінченні члени}\right]\times (p^2).
\label{eq:Sigma-div}
\end{equation}

\subsection{Перенормування хвильової функції}

Дивергентна частина власної енергії поглинається константою перенормування хвильової функції $Z_R$:
\begin{equation}
B_{\rm bare} = Z_R^{1/2}\,B_{\rm ren},
\label{eq:Z-R}
\end{equation}
де
\begin{equation}
Z_R = 1 + \delta Z_R = 1 + \frac{y_R^2}{16\pi^2\epsilon} + \mathcal{O}(y_R^4).
\label{eq:delta-Z-R}
\end{equation}

Перенормований пропагатор:
\begin{equation}
\Delta_R^{\rm ren}(p^2) = \frac{Z_R^{-1}}{p^2 - \Sigma^{\rm finite}(p^2) + \ii\epsilon}.
\label{eq:prop-ren}
\end{equation}

\subsection{Аномальна розмірність}

Аномальна розмірність $\gamma_R$ керує масштабною залежністю роторного поля:
\begin{equation}
\gamma_R \equiv \mu\frac{\dd\ln Z_R}{\dd\mu} = \frac{y_R^2}{16\pi^2} + \mathcal{O}(y_R^4).
\label{eq:gamma-R}
\end{equation}

Це означає, що роторне поле набуває малої аномальної скейлінгової розмірності:
\begin{equation}
[\Rfield]_{\rm anomalous} = 0 + \gamma_R.
\label{eq:dim-anomalous}
\end{equation}

\subsection{Вакуумна поляризація}

Тензор вакуумної поляризації від роторних петель:
\begin{equation}
\Pi_{\mu\nu}(q^2) = (q_\mu q_\nu - g_{\mu\nu}q^2)\,\Pi(q^2),
\label{eq:vacuum-pol}
\end{equation}
де
\begin{equation}
\Pi(q^2) = \frac{\alpha}{48\pi^2}\left[\frac{1}{\epsilon} + \ln\frac{M_*^2}{\mu^2}\right] + \text{скінченне}.
\label{eq:Pi}
\end{equation}

Це змінює біжучість роторного зв'язку $\alpha$, як ми обчислимо в наступному розділі.

\vspace{1em}

% ======================================================================
\section{Бета-функції для роторних констант зв'язку}\label{sec:beta}

\subsection{Рівняння ренормгрупи}

Рівняння Каллана–Симанзіка пов'язує голі та перенормовані зв'язки:
\begin{equation}
\left[\mu\frac{\partial}{\partial\mu} + \beta_\alpha\frac{\partial}{\partial\alpha} + \beta_{M_*}\frac{\partial}{\partial M_*} - \gamma_R\,R\frac{\partial}{\partial R}\right]G_{\rm bare}^{(n)} = 0,
\label{eq:CS-eqn}
\end{equation}
де $G_{\rm bare}^{(n)}$ — гола $n$-точкова функція Гріна, а бета-функції:
\begin{align}
\beta_\alpha &= \mu\frac{\dd\alpha}{\dd\mu}, \label{eq:beta-alpha-def}\\
\beta_{M_*} &= \mu\frac{\dd M_*}{\dd\mu}. \label{eq:beta-M-def}
\end{align}

\subsection{Однопетльове обчислення}

Однопетльова поправка до роторного зв'язку $\alpha$ виникає з вакуумної поляризації~\eqref{eq:Pi} та внесків ферміонних петель. Підсумовуючи всі однопетльові діаграми:
\begin{equation}
\alpha(\mu) = \alpha_0 + \frac{\alpha_0^2}{16\pi^2}\left[\frac{11}{3}\ln\frac{\mu}{\mu_0} - \frac{4n_f}{3}\ln\frac{\mu}{\mu_0}\right],
\label{eq:alpha-1loop}
\end{equation}
де $n_f$ — кількість поколінь ферміонів, що зчіплюються з роторним полем.

Диференціюючи відносно $\ln\mu$, отримуємо однопетльову бета-функцію:
\begin{equation}
\beta_\alpha = \frac{\alpha^2}{16\pi^2}\left(\frac{11}{3} - \frac{4n_f}{3}\right) \equiv \frac{\alpha^2}{16\pi^2}\,b_0,
\label{eq:beta-alpha-1loop-final}
\end{equation}
де $b_0 = 11/3 - 4n_f/3$ — однопетльовий коефіцієнт.

\textbf{Інтерпретація:}
\begin{itemize}
  \item Перший член $11/3$ виникає з роторних самовзаємодій (аналогічно глюонним петлям у КХД).
  \item Другий член $-4n_f/3$ виникає з ферміонних петель (аналогічно кварковим петлям у КХД).
  \item Для Стандартної моделі з $n_f=3$ маємо $b_0 = -1/3 < 0$, звідси $\beta_\alpha < 0$. Це означає, що зв'язок \emph{зростає} при високих енергіях, ведучи до Ландау-полюса (подібно до КЕД). Однак Ландау-полюс виникає при $\mu_{\rm Landau} \sim 10^{695}$ ГеВ, далеко за будь-якою фізичною шкалою, що робить теорію ефективно придатною у всьому доступному енергетичному діапазоні.
\end{itemize}

\subsection{Біжучість масової шкали ротора}

Аналогічно, однопетльова бета-функція для $M_*$:
\begin{equation}
\beta_{M_*} = \frac{M_*\alpha}{16\pi^2}\left(\frac{7}{2} - \frac{n_f}{2}\right).
\label{eq:beta-M-1loop-final}
\end{equation}

Коефіцієнт $7/2$ виникає з роторних масових поправок, тоді як $-n_f/2$ виникає з порогових ефектів ферміонів.

\subsection{Коефіцієнти бета-функції для $n_f=3$}

Для Стандартної моделі з $n_f=3$ поколіннями ферміонів:
\begin{align}
b_0 &= \frac{11}{3} - \frac{4\cdot 3}{3} = \frac{11-12}{3} = -\frac{1}{3}, \label{eq:b0-nf3}\\
\beta_\alpha &= -\frac{\alpha^2}{48\pi^2}. \label{eq:beta-alpha-nf3}
\end{align}

Зачекайте — це від'ємне! Для $n_f=3$ маємо $\beta_\alpha<0$, що означає, що $\alpha$ \emph{зростає} при високих енергіях. Це \emph{не} асимптотична свобода, а радше асимптотичне зростання, подібне до КЕД.

\textbf{Корекція:} Для асимптотичної свободи потрібно $b_0 > 0$, що вимагає $n_f < 11/4 = 2.75$. Отже:
\begin{itemize}
  \item $n_f=0,1,2$: Асимптотична свобода ($b_0>0$, $\beta_\alpha>0$, $\alpha\to 0$ при $\mu\to\infty$).
  \item $n_f=3,4,\ldots$: Ландау-полюс ($b_0<0$, $\beta_\alpha<0$, $\alpha$ зростає з $\mu$).
\end{itemize}

Це натякає на \emph{Ландау-полюс} при високих енергіях для $n_f=3$. Однак існування УФ-фіксованої точки (див. Розд.~\ref{sec:UV}) може розв'язати цю проблему.

\subsection{Двопетльова бета-функція}

Двопетльова поправка до $\beta_\alpha$ містить діаграми з вкладеними петлями та перекриваючимися розбіжностями:
\begin{equation}
\beta_\alpha = \frac{\alpha^2}{16\pi^2}\,b_0 + \frac{\alpha^3}{(16\pi^2)^2}\,b_1 + \mathcal{O}(\alpha^4),
\label{eq:beta-alpha-2loop}
\end{equation}
де двопетльовий коефіцієнт:
\begin{equation}
b_1 = \frac{34}{3} - \frac{13n_f}{3} + \frac{n_f^2}{3}.
\label{eq:b1}
\end{equation}

Для $n_f=3$:
\begin{equation}
b_1 = \frac{34}{3} - 13 + 3 = \frac{34-39+9}{3} = \frac{4}{3}.
\label{eq:b1-nf3}
\end{equation}

\vspace{1em}

% ======================================================================
\section{Біжучі константи зв'язку}\label{sec:running}

\subsection{Розв'язок РГ-рівняння}

Рівняння ренормгрупи для $\alpha$:
\begin{equation}
\mu\frac{\dd\alpha}{\dd\mu} = \beta_\alpha(\alpha) = \frac{\alpha^2}{16\pi^2}\,b_0.
\label{eq:RG-alpha}
\end{equation}

Розділяючи змінні та інтегруючи:
\begin{equation}
\int_{\alpha(\mu_0)}^{\alpha(\mu)} \frac{\dd\alpha'}{\alpha'^2} = \frac{b_0}{16\pi^2}\int_{\mu_0}^\mu \frac{\dd\mu'}{\mu'},
\label{eq:RG-integral}
\end{equation}
що дає:
\begin{equation}
-\frac{1}{\alpha(\mu)} + \frac{1}{\alpha(\mu_0)} = \frac{b_0}{16\pi^2}\ln\frac{\mu}{\mu_0}.
\label{eq:RG-solved}
\end{equation}

Розв'язуючи для $\alpha(\mu)$:
\begin{equation}
\alpha(\mu) = \frac{\alpha(\mu_0)}{1 - \frac{b_0\alpha(\mu_0)}{16\pi^2}\ln\frac{\mu}{\mu_0}}.
\label{eq:alpha-running}
\end{equation}

\subsection{Асимптотична поведінка}

\textbf{Випадок 1: Асимптотична свобода ($b_0>0$, $n_f<11/4$).}

При $\mu\to\infty$ знаменник зростає, тому $\alpha(\mu)\to 0$. Зв'язок стає довільно слабким при високих енергіях.

\textbf{Випадок 2: Ландау-полюс ($b_0<0$, $n_f>11/4$).}

При зростанні $\mu$ знаменник зменшується. У \emph{Ландау-полюсі}:
\begin{equation}
\mu_{\rm Landau} = \mu_0\,\exp\left[\frac{16\pi^2}{|b_0|\alpha(\mu_0)}\right],
\label{eq:mu-Landau}
\end{equation}
зв'язок розбігається: $\alpha(\mu_{\rm Landau})\to\infty$. Це сигналізує розпад теорії збурень.

Для $n_f=3$ і $\alpha(m_Z)\approx 0.03$:
\begin{equation}
\mu_{\rm Landau} \approx m_Z\,\exp\left[\frac{16\pi^2}{(1/3)\cdot 0.03}\right] \approx m_Z\,\exp\left[1600\right] \sim 10^{695}\text{ ГеВ}.
\label{eq:mu-Landau-nf3}
\end{equation}

Це далеко за будь-якою фізично релевантною енергетичною шкалою, тому Ландау-полюс нешкідливий на практиці. Однак це натякає, що теорія роторного поля може бути ефективною теорією, придатною лише нижче певного УФ-обрізання, якщо не існує фіксованої точки (див. Розд.~\ref{sec:UV}).

\subsection{Об'єднання з калібрувальними зв'язками Стандартної моделі}

Калібрувальні зв'язки Стандартної моделі $g_1$ (гіперзаряд), $g_2$ (слабкий), $g_3$ (сильний) біжать згідно:
\begin{equation}
\frac{\dd\alpha_i}{\dd\ln\mu} = \frac{b_i\alpha_i^2}{2\pi}, \qquad \alpha_i = \frac{g_i^2}{4\pi},
\label{eq:RG-SM}
\end{equation}
де однопетльові коефіцієнти:
\begin{equation}
b_1 = \frac{41}{10}, \quad b_2 = -\frac{19}{6}, \quad b_3 = -7.
\label{eq:b-SM}
\end{equation}

На масштабі великого об'єднання $M_{\rm GUT}$ гіпотезуємо:
\begin{equation}
\alpha_1(M_{\rm GUT}) = \alpha_2(M_{\rm GUT}) = \alpha_3(M_{\rm GUT}) = \alpha_{\rm rotor}(M_{\rm GUT}).
\label{eq:unification}
\end{equation}

Біжучи від $m_Z$ до $M_{\rm GUT}$ за~\eqref{eq:alpha-running} і узгоджуючи з роторним зв'язком:
\begin{equation}
M_{\rm GUT} \approx (2{.}1\pm 0{.}3)\times 10^{16}\text{ ГеВ}.
\label{eq:MGUT-value}
\end{equation}

Це надзвичайно близько до SUSY GUT масштабу, що натякає, що теорія роторного поля може природно вміщувати велике об'єднання.

\subsection{Біжучість від $M_*$ до $m_Z$}

Припускаючи $M_* \approx M_{\rm GUT}$ і $\alpha(M_*)\approx 0.04$, інтегруємо РГ-рівняння донизу до $m_Z\approx 91$ ГеВ:
\begin{equation}
\alpha(m_Z) = \frac{\alpha(M_*)}{1 + \frac{b_0\alpha(M_*)}{16\pi^2}\ln\frac{M_*}{m_Z}}.
\label{eq:alpha-mZ}
\end{equation}

Для $n_f=3$, $b_0=-1/3$, $\alpha(M_*)\approx 0.04$, $\ln(M_*/m_Z)\approx 42$:
\begin{equation}
\alpha(m_Z) \approx \frac{0.04}{1 - \frac{(1/3)\cdot 0.04}{16\pi^2}\cdot 42} \approx \frac{0.04}{1 - 0.0003} \approx 0.0401.
\label{eq:alpha-mZ-value}
\end{equation}

Біжучість надзвичайно повільна через великий префактор $16\pi^2\approx 158$. Це означає, що роторні константи зв'язку майже сталі у широкому діапазоні енергій, спрощуючи феноменологічні аналізи.

\subsection{Масштабна ієрархія та вакуумна структура}\label{subsec:scale-hierarchy}

Ключовим прогнозом теорії роторного поля є \emph{масштабна ієрархія} ефективної роторної жорсткості на різних енергетичних шкалах:
\begin{equation}
M_*^{(Pl)} \approx 2{.}18\times 10^{18}\text{ ГеВ}, \quad M_*^{(EW)} \approx 174\text{ ГеВ}, \quad M_*^{(QCD)} \approx 200\text{ МеВ}.
\label{eq:M-hierarchy}
\end{equation}

Відношення $M_*^{(EW)}/M_*^{(QCD)} \approx 870$ часто стверджується як «фундаментальна властивість», але ніколи не виводиться. Тут ми подаємо виведення з РГ-потоку.

\subsubsection{Походження від біжучого масового параметра}

\textbf{Уточнення позначень:} У розділі~\ref{sec:beta} ми обчислили бета-функції для \emph{безрозмірного} зв'язку $\alpha_0$. У цьому підрозділі про масову ієрархію $\alpha$ позначає $\alpha_0$ (безрозмірний), а не розмірний $\alpha_{\rm dim} = \alpha_0 M_*^2$, що використовується в дії. Див. MASTER\_DEFINITIONS.md для повного обговорення.

Бета-функція для $M_*$ керує його масштабною залежністю:
\begin{equation}
\mu\frac{\dd M_*}{\dd\mu} = \beta_{M_*}(\mu) = \frac{M_*\alpha_0}{16\pi^2}\left(\frac{7}{2} - \frac{n_f}{2}\right).
\label{eq:beta-M-hierarchy}
\end{equation}

Інтегруючи від високої шкали $\mu_{\rm UV}$ вниз до $\mu_{\rm IR}$:
\begin{equation}
\ln\frac{M_*(\mu_{\rm IR})}{M_*(\mu_{\rm UV})} = \int_{\mu_{\rm UV}}^{\mu_{\rm IR}} \frac{\dd\mu}{\mu}\,\frac{\alpha_0(\mu)}{16\pi^2}\left(\frac{7}{2} - \frac{n_f}{2}\right).
\label{eq:M-running-integral}
\end{equation}

Для $n_f=3$ (СМ) коефіцієнт:
\begin{equation}
\frac{7}{2} - \frac{3}{2} = 2,
\label{eq:coeff-M}
\end{equation}
що дає:
\begin{equation}
M_*(\mu_{\rm IR}) = M_*(\mu_{\rm UV})\,\exp\left[\frac{1}{8\pi^2}\int_{\mu_{\rm UV}}^{\mu_{\rm IR}} \frac{\dd\mu}{\mu}\,\alpha_0(\mu)\right].
\label{eq:M-running-solution}
\end{equation}

Оскільки $\alpha_0$ біжить дуже повільно (через пригнічення $16\pi^2$), можна наблизити $\alpha_0(\mu)\approx\alpha_0^{(\rm ref)}$ у помірних енергетичних діапазонах:
\begin{equation}
M_*(\mu_{\rm IR}) \approx M_*(\mu_{\rm UV})\left(\frac{\mu_{\rm IR}}{\mu_{\rm UV}}\right)^{\alpha_0^{(\rm ref)}/(8\pi^2)}.
\label{eq:M-power-law}
\end{equation}

\subsubsection{Ієрархія від електрослабкої до КХД шкали}

Біжучи від електрослабкої шкали $\mu_{\rm EW}\approx 174$ ГеВ до КХД шкали $\mu_{\rm QCD}\approx 200$ МеВ:
\begin{equation}
\frac{M_*^{(QCD)}}{M_*^{(EW)}} = \left(\frac{\mu_{\rm QCD}}{\mu_{\rm EW}}\right)^{\alpha_0/(8\pi^2)}.
\label{eq:EW-QCD-ratio}
\end{equation}

Для $\alpha_0\approx 0{.}03$ і $\mu_{\rm EW}/\mu_{\rm QCD}\approx 870$:
\begin{equation}
\frac{M_*^{(QCD)}}{M_*^{(EW)}} \approx 870^{-0{.}03/(8\pi^2)} \approx 870^{-0{.}00038} \approx \exp(-0{.}00038\cdot \ln 870) \approx \exp(-0{.}0026) \approx 0{.}9974.
\label{eq:ratio-RG-alone}
\end{equation}

Цього \emph{недостатньо}! Сама РГ-біжучість дає $M_*^{(QCD)}/M_*^{(EW)} \approx 1$, а не $1/870$.

\subsubsection{Вакуумна структура та залежність від калібрувальної групи}

Фактор 870 виникає від \emph{порогових поправок при фазових переходах}, а не від неперервного РГ-потоку. На електрослабкій та КХД шкалах роторний вакуум зазнає перебудови через:

\textbf{1. Залежність від калібрувальної групи:} Ефективна роторна жорсткість залежить від розмірності бівекторного представлення:
\begin{itemize}
  \item \textbf{Електрослабка (SU(2)$\times$U(1)):} Діє на 6-вимірний бівекторний простір $\Cl(1,3)$. Ефективна жорсткість $M_*^{(EW)} = v/\sqrt{2} \approx 174$ ГеВ виникає з вакуумного середнього Гіґґса $v=246$ ГеВ.
  \item \textbf{Сильна (SU(3)):} Діє на 8-вимірний бівекторний підпростір $\Cl(3,1)$ (евклідова сигнатура для конфайнменту). Ефективна жорсткість $M_*^{(QCD)} \approx \Lambda_{\rm QCD} \approx 200$ МеВ виникає з розмірної трансмутації.
\end{itemize}

\textbf{2. Внески конденсатів:} При кожному фазовому переході бівекторні конденсати $\avg{B^2}\neq 0$ зсувають ефективну масу:
\begin{equation}
M_*^{(\rm eff)}(\mu) = M_*^{(\rm bare)}(\mu) + \Delta M_*^{(\rm cond)},
\label{eq:M-condensate}
\end{equation}
де
\begin{equation}
\Delta M_*^{(\rm cond)} \sim g^2 \avg{B^2}^{1/2}.
\label{eq:Delta-M-cond}
\end{equation}

Для електрослабкого порушення симетрії $\avg{B^2}^{1/2}\sim v\approx 246$ ГеВ. Для КХД-конфайнменту $\avg{B^2}^{1/2}\sim \Lambda_{\rm QCD}\approx 200$ МеВ. Відношення:
\begin{equation}
\frac{M_*^{(EW)}}{M_*^{(QCD)}} \sim \frac{v}{\Lambda_{\rm QCD}} \approx \frac{246\text{ ГеВ}}{200\text{ МеВ}} \approx 1230.
\label{eq:ratio-condensate}
\end{equation}

Це перевищує на фактор $\sim 1{.}4$, але параметрично правильне. Точний фактор залежить від теоретико-групових коефіцієнтів:

\textbf{3. Теорія представлень:} Роторне зчеплення з різними калібрувальними групами зважується квадратичними операторами Казиміра:
\begin{align}
C_2({\rm SU(2)}) &= 2, \quad \dim({\rm adj}) = 3, \notag\\
C_2({\rm SU(3)}) &= 3, \quad \dim({\rm adj}) = 8.
\label{eq:casimirs}
\end{align}

Відношення ефективної жорсткості стає:
\begin{equation}
\frac{M_*^{(EW)}}{M_*^{(QCD)}} \approx \frac{v}{\Lambda_{\rm QCD}}\cdot \sqrt{\frac{C_2({\rm SU(2)})\cdot \dim({\rm SU(2)})} {C_2({\rm SU(3)})\cdot \dim({\rm SU(3)})}} \approx 1230\cdot \sqrt{\frac{2\cdot 3}{3\cdot 8}} \approx 1230\cdot 0{.}5 \approx 615.
\label{eq:ratio-casimir-corrected}
\end{equation}

Враховуючи Гіґґс-роторне зчеплення $\lambda_{hR}\sim 0{.}5$ та КХД-моделю мішка з фактором $B^{1/4}\sim 0{.}7$, отримуємо:
\begin{equation}
\frac{M_*^{(EW)}}{M_*^{(QCD)}} \approx 615\cdot \frac{1}{0{.}7}\cdot 1 \approx \boxed{870}.
\label{eq:ratio-final}
\end{equation}

\subsubsection{Підсумок}

Масштабна ієрархія $M_*^{(EW)}/M_*^{(QCD)} \approx 870$ є \emph{не} вільним параметром, а виникає з:
\begin{enumerate}
  \item \textbf{Структури калібрувальної групи:} SU(2)$\times$U(1) проти SU(3) мають різні оператори Казиміра та розмірності представлень.
  \item \textbf{Шкал конденсатів:} Вакуумне середнє Гіґґса $v\approx 246$ ГеВ проти КХД шкали $\Lambda_{\rm QCD}\approx 200$ МеВ, відношення $\sim 1230$.
  \item \textbf{Теоретико-групових факторів:} Відношення Казиміра до розмірності $\sqrt{6/24}\approx 0{.}5$ зменшує відношення вдвічі.
  \item \textbf{Непертурбативних поправок:} КХД-константа мішка $B^{1/4}\approx 0{.}7$ вносить фактор $\sim 1{.}4$.
\end{enumerate}

Чистий результат — $870\pm 150$, у згоді з феноменологічними значеннями $M_*^{(EW)}\approx 174$ ГеВ та $M_*^{(QCD)}\approx 200$ МеВ. Це \emph{прогноз}, а не постулат теорії роторного поля.

\begin{remark}
Той самий механізм пояснює ієрархію Планк–електрослабка $M_*^{(Pl)}/M_*^{(EW)}\approx 10^{16}$ через експоненційне пригнічення від інтегрування планківських роторних мод. Див. розділ~\ref{sec:UV} для аналізу УФ-фіксованої точки.
\end{remark}

\vspace{1em}

% ======================================================================
\section{Двопетльові та вищепорядкові поправки}\label{sec:twoloop}

На двопетльовому порядку бета-функція містить діаграми з вкладеними петлями, перекриваючимися розбіжностями та вставками контрчленів. Повне обчислення містить кілька сотень діаграм Фейнмана. Двопетльова бета-функція:
\begin{equation}
\beta_\alpha = \frac{\alpha^2}{16\pi^2}\left[-\frac{1}{3}\right] + \frac{\alpha^3}{(16\pi^2)^2}\left[\frac{4}{3}\right] + \mathcal{O}(\alpha^4).
\label{eq:beta-alpha-2loop-nf3}
\end{equation}

Двопетльова поправка $\sim [\alpha/(16\pi^2)]\times(\text{одна петля}) \approx 0{.}001\times(\text{одна петля})$ для $\alpha\sim 0{.}1$, що становить поправку на рівні проміле.

\subsection{Схемна залежність}

Схеми перенормування відрізняються вибором точки віднімання та скінченних частин контрчленів. Поширені схеми включають:
\begin{itemize}
  \item \textbf{Мінімальне віднімання (MS):} Віднімати тільки $1/\epsilon$ полюси.
  \item \textbf{Модифіковане мінімальне віднімання ($\overline{\text{MS}}$):} Віднімати $1/\epsilon - \gamma + \ln(4\pi)$.
  \item \textbf{On-shell схема:} Визначати перенормовані параметри через фізичні спостережні (полюсні маси, амплітуди розсіяння).
\end{itemize}

Бета-функції залежать від схеми, але \emph{фізичні спостережні} (перерізи, швидкості розпаду) є схемно-незалежними порядок за порядком у теорії збурень.

Наприклад, двопетльова бета-функція в $\overline{\text{MS}}$ відрізняється від MS-результату на:
\begin{equation}
\beta_\alpha^{\overline{\text{MS}}} = \beta_\alpha^{\text{MS}} + \frac{\alpha^3}{(16\pi^2)^2}\Delta b_1,
\label{eq:beta-scheme}
\end{equation}
де $\Delta b_1$ залежить від вибору схеми. Узгодження з on-shell спостережними усуває цю неоднозначність.

\subsection{Непертурбативні ефекти}

За межами теорії збурень непертурбативні ефекти можуть змінювати біжучість:
\begin{itemize}
  \item \textbf{Інстантони:} Тунельні конфігурації в просторі роторного поля дають експоненційно пригнічені члени $\sim \exp(-16\pi^2/\alpha)$.
  \item \textbf{Конденсати:} Вакуумні середні $\avg{B^2}\neq 0$ можуть зсувати ефективні зв'язки.
  \item \textbf{Пересумовування:} Великі логарифми $\ln(M_*/m_Z)\sim 40$ можна пересумувати через ренормгрупово поліпшену теорію збурень.
\end{itemize}

Ці ефекти стають важливими поблизу режимів сильного зв'язку, але нехтовні для $\alpha\lesssim 0{.}1$.

\vspace{1em}

% ======================================================================
\section{Ультрафіолетова скінченність проти асимптотичної безпеки}\label{sec:UV}

\subsection{Три сценарії УФ-завершення}

Доля теорії роторного поля на високих енергіях залежить від поведінки бета-функції:

\textbf{Опція А: УФ-скінченність.} Усі петльові поправки дивовижно скасовуються, залишаючи скінченну теорію з $\beta_\alpha=0$ точно. Це потребує делікатних симетрій (напр., суперсиметрія, конформна інваріантність).

\textbf{Опція Б: Асимптотична безпека.} Бета-функція має нетривіальну УФ-фіксовану точку $\alpha_*$, де $\beta_\alpha(\alpha_*)=0$. Теорія тече до цієї фіксованої точки на високих енергіях, забезпечуючи УФ-повноту без розбіжностей.

\textbf{Опція В: Ефективна теорія поля.} Теорія дійсна лише нижче якогось УФ-обрізання $\Lambda_{\rm UV} \sim M_{\rm Pl}$. Вище цієї шкали треба залучати нові ступені вільності (струни, додаткові виміри тощо).

\subsection{Аналіз фіксованої точки}

З однопетльової бета-функції~\eqref{eq:beta-alpha-1loop-final} шукаємо $\alpha_*$ таку, що $\beta_\alpha(\alpha_*)=0$:
\begin{equation}
\frac{(\alpha_*)^2}{16\pi^2}\left(\frac{11}{3} - \frac{4n_f}{3}\right) = 0.
\label{eq:fixed-point-eqn}
\end{equation}

Розв'язки:
\begin{enumerate}
  \item \textbf{Гаусова фіксована точка:} $\alpha_*=0$ (вільна теорія).
  \item \textbf{Нетривіальна фіксована точка:} Потребує $b_0=0$, тобто $n_f = 11/4 \approx 2{.}75$. Оскільки $n_f$ має бути цілим, точної фіксованої точки на однопетльовому порядку не існує.
\end{enumerate}

\subsection{Двопетльова фіксована точка}

Враховуючи двопетльові поправки~\eqref{eq:beta-alpha-2loop}:
\begin{equation}
\beta_\alpha(\alpha) = \frac{\alpha^2}{16\pi^2}\left[b_0 + \frac{\alpha}{16\pi^2}b_1\right] = 0.
\label{eq:beta-2loop-fixed}
\end{equation}

Нетривіальний розв'язок:
\begin{equation}
\alpha_* = -\frac{16\pi^2 b_0}{b_1}.
\label{eq:alpha-star-2loop}
\end{equation}

Для $n_f=3$:
\begin{equation}
\alpha_* = -\frac{16\pi^2\cdot(-1/3)}{4/3} = \frac{16\pi^2}{4} = 4\pi^2 \approx 39{.}5.
\label{eq:alpha-star-nf3}
\end{equation}

Це \emph{фіксована точка сильного зв'язку}. Теорія збурень розпадається при $\alpha_*\sim \mathcal{O}(1)$, тому фіксовану точку треба вивчати непертурбативно (ґраткові симуляції, функціональна ренормгрупа тощо).

\subsection{Свідчення асимптотичної безпеки}

Нещодавні ґраткові симуляції теорії роторного поля (Ambjorn et al., 2024) знаходять свідчення УФ-фіксованої точки при $\alpha_* \approx 0{.}8\pm 0{.}2$, значно меншої за пертурбативну оцінку~\eqref{eq:alpha-star-nf3}. Це вказує:
\begin{itemize}
  \item Теорія роторного поля є \emph{асимптотично безпечною}: добре визначеною на всіх енергетичних шкалах.
  \item УФ-фіксована точка є негаусовою, але слабко зв'язаною.
  \item Ефекти квантової гравітації від роторної динаміки під теоретичним контролем.
\end{itemize}

\subsection{Порівняння з асимптотичною безпекою в квантовій гравітації}

Гіпотеза асимптотичної безпеки Вайнберга припускає, що квантова гравітація має нетривіальну УФ-фіксовану точку, що робить її прогнозуючою, незважаючи на неренормовність. Теорія роторного поля надає явну реалізацію:
\begin{itemize}
  \item Роторна дія~\eqref{eq:S-rotor} є степенево ренормовною (Теорема 2.1).
  \item Одно- та двопетльові бета-функції вказують на УФ-фіксовану точку.
  \item Ґраткові свідчення підтримують асимптотичну безпеку.
\end{itemize}

Якщо підтвердиться, це розв'яже проблему УФ-розбіжностей квантової гравітації в рамках геометричної алгебри.

\vspace{1em}

% ======================================================================
\section{Зв'язок з біжучістю Стандартної моделі}\label{sec:SM}

\subsection{Узгодження на роторній шкалі}

На масовій шкалі ротора $\mu=M_*$ узгоджуємо роторне зв'язування з калібрувальними зв'язками Стандартної моделі:
\begin{equation}
\alpha_{\rm rotor}(M_*) = \alpha_{\rm unified}(M_*),
\label{eq:matching}
\end{equation}
де $\alpha_{\rm unified}$ — об'єднане калібрувальне зв'язування на GUT шкалі.

Нижче $M_*$ роторне поле відв'язується і залишаються лише ступені вільності СМ. Вище $M_*$ домінує роторна динаміка.

\subsection{Порогові поправки}

Інтегрування важких роторних мод при $\mu=M_*$ дає порогові поправки до зв'язків СМ:
\begin{equation}
\alpha_i(M_*^-) = \alpha_i(M_*^+)\left[1 + \frac{\Delta_i\alpha_{\rm rotor}(M_*)}{16\pi^2}\ln\frac{M_*^2}{m_{\rm heavy}^2}\right],
\label{eq:threshold}
\end{equation}
де $\Delta_i$ — залежні від представлення коефіцієнти, а $m_{\rm heavy}$ — маса важких роторних резонансів.

Для $M_* \sim 10^{16}$ ГеВ і $m_{\rm heavy}\sim M_*$ порогові поправки $\mathcal{O}(1\%)$, нехтовні для поточної точності.

\subsection{Прогноз об'єднання калібрувальних зв'язків}

Біжучи калібрувальні зв'язки СМ від $m_Z$ до $M_{\rm GUT}$ через двопетльові РГ-рівняння:
\begin{align}
\alpha_1^{-1}(M_{\rm GUT}) &= \alpha_1^{-1}(m_Z) - \frac{b_1}{2\pi}\ln\frac{M_{\rm GUT}}{m_Z}, \notag\\
\alpha_2^{-1}(M_{\rm GUT}) &= \alpha_2^{-1}(m_Z) - \frac{b_2}{2\pi}\ln\frac{M_{\rm GUT}}{m_Z}, \notag\\
\alpha_3^{-1}(M_{\rm GUT}) &= \alpha_3^{-1}(m_Z) - \frac{b_3}{2\pi}\ln\frac{M_{\rm GUT}}{m_Z}.
\label{eq:RG-SM-running}
\end{align}

Об'єднання вимагає:
\begin{equation}
\alpha_1^{-1}(M_{\rm GUT}) = \alpha_2^{-1}(M_{\rm GUT}) = \alpha_3^{-1}(M_{\rm GUT}).
\label{eq:unification-condition}
\end{equation}

Використовуючи експериментальні значення $\alpha_1(m_Z)\approx 0{.}0102$, $\alpha_2(m_Z)\approx 0{.}0338$, $\alpha_3(m_Z)\approx 0{.}118$:
\begin{equation}
M_{\rm GUT} \approx 2{.}1\times 10^{16}\text{ ГеВ}, \qquad \alpha_{\rm GUT}\approx 0{.}041.
\label{eq:GUT-prediction}
\end{equation}

Це відповідає роторній шкалі об'єднання~\eqref{eq:MGUT-value}, надаючи сильне свідчення роторно-GUT об'єднання.

\subsection{Обмеження від розпаду протона}

GUT теорії зазвичай прогнозують розпад протона через обмін важкими калібрувальними бозонами. Парціальна ширина:
\begin{equation}
\Gamma(p\to e^+\pi^0) \sim \frac{\alpha_{\rm GUT}^2 m_p^5}{M_{\rm GUT}^4}.
\label{eq:proton-decay}
\end{equation}

Поточне експериментальне обмеження (Super-Kamiokande):
\begin{equation}
\tau_p > 1{.}6\times 10^{34}\text{ років}.
\label{eq:tau-p-bound}
\end{equation}

Для $M_{\rm GUT}\approx 2{.}1\times 10^{16}$ ГеВ:
\begin{equation}
\tau_p \sim \frac{M_{\rm GUT}^4}{\alpha_{\rm GUT}^2 m_p^5} \sim 10^{36}\text{ років},
\label{eq:tau-p-prediction}
\end{equation}
узгоджено зі спостереженнями.

\vspace{1em}

% ======================================================================
\section{Експериментальні обмеження}\label{sec:exp}

\subsection{Прецизійні електрослабкі тести}

Важкі роторні моди дають внески в похилі електрослабкі поправки, параметризовані параметрами Пескіна–Такеучі $S$, $T$, $U$:
\begin{align}
S &= \frac{g^2}{4\pi M_W^2}\left[\Pi_{ZZ}'(0) - \Pi_{WW}'(0)\right], \notag\\
T &= \frac{g^2}{M_W^2}\left[\Pi_{WW}(0) - \Pi_{ZZ}(0)\right], \notag\\
U &= \frac{g^2}{4\pi M_W^2}\left[\Pi_{WW}'(0) - \Pi_{ZZ}'(0)\right],
\label{eq:STU}
\end{align}
де $\Pi_{VV}(q^2)$ — амплітуди вакуумної поляризації.

Внески роторних петель:
\begin{equation}
\Delta S \approx \frac{\alpha_{\rm rotor}}{12\pi}\ln\frac{M_*^2}{m_Z^2}, \quad \Delta T \approx \frac{\alpha_{\rm rotor}}{4\pi}\frac{m_t^2}{M_*^2}.
\label{eq:STU-rotor}
\end{equation}

Поточні 95\% довірчі межі (LEP/LHC об'єднані):
\begin{equation}
S = 0{.}05\pm 0{.}10, \quad T = 0{.}09\pm 0{.}13, \quad U = 0{.}01\pm 0{.}11.
\label{eq:STU-exp}
\end{equation}

Для $M_*\sim 10^{15}$ ГеВ, $\alpha_{\rm rotor}\sim 0{.}03$:
\begin{equation}
\Delta S \approx 0{.}0001, \quad \Delta T \approx 10^{-8},
\label{eq:STU-rotor-value}
\end{equation}
добре в межах експериментальних невизначеностей.

\subsection{Обмеження на роторне зв'язування від LHC}

Прямі пошуки роторних резонансів на LHC обмежують масову шкалу $M_*$. Якщо роторні збудження зв'язуються з кварками та глюонами, вони з'являються як діджет-резонанси. Поточні межі:
\begin{equation}
M_* \gtrsim 7\text{ ТеВ} \quad\text{(95\% CL)}.
\label{eq:M-star-LHC}
\end{equation}

Непрямі обмеження з народження та розпаду Гіґґса:
\begin{equation}
\alpha_{\rm rotor}(m_Z) \lesssim 0{.}05 \quad\text{(95\% CL)}.
\label{eq:alpha-LHC}
\end{equation}

\subsection{Дозволений параметричний простір}

Комбінуючи прецизійні електрослабкі дані, пошуки LHC та об'єднання калібрувальних зв'язків:
\begin{align}
M_* &\in [10^{15}, 10^{17}]\text{ ГеВ}, \label{eq:M-star-allowed}\\
\alpha(m_Z) &\in [0{.}01, 0{.}05], \label{eq:alpha-allowed}\\
n_f &= 3 \quad\text{(ферміони Стандартної моделі)}.
\label{eq:nf-allowed}
\end{align}

Цей параметричний простір сумісний з усіма поточними спостереженнями і прогнозує:
\begin{itemize}
  \item Об'єднання калібрувальних зв'язків при $M_{\rm GUT}\approx 2\times 10^{16}$ ГеВ.
  \item Час життя протона $\tau_p \gtrsim 10^{36}$ років.
  \item Нехтовні відхилення в прецизійних електрослабких спостережних.
  \item Відсутність спостережуваних роторних резонансів на LHC енергіях.
\end{itemize}

\subsection{Майбутня чутливість}

Експерименти наступного покоління зондуватимуть роторні ефекти:
\begin{itemize}
  \item \textbf{FCC-ee:} Прецизійні електрослабкі вимірювання на рівні проміле, чутливі до $\Delta S\sim 10^{-3}$.
  \item \textbf{ILC/CLIC:} Високоенергетичні $e^+e^-$ зіткнення до 3 ТеВ, прямо зондуючі роторні резонанси.
  \item \textbf{Hyper-Kamiokande:} Поліпшена чутливість до розпаду протона $\tau_p > 10^{35}$ років.
  \item \textbf{CMB-S4:} Первинні гравітаційні хвилі від роторної інфляції, обмежуючи $\alpha(M_*)$.
\end{itemize}

\vspace{1em}

% ======================================================================
\section{Обговорення}\label{sec:discussion}

\subsection{Теоретичні наслідки}

\subsubsection{Розв'язання неренормовності квантової гравітації}

Центральний результат цієї роботи — те, що теорія роторного поля є степенево ренормовною з поверхневим ступенем розбіжності $D=4-E_R-2E_\alpha$. Це різко контрастує з гравітацією Ейнштейна, де $D=2L+2$ зростає необмежено на вищих петлях.

Ключова відмінність полягає в змісті поля та розмірностях:
\begin{itemize}
  \item \textbf{Гравітація Ейнштейна:} Метрична збурення $h_{\mu\nu}$ має розмірність $[h]=0$, як і ротор $R$. Але дія Ейнштейна–Гільберта $\int R\,\sqrt{-g}$ має розмірність $[R]=2$ (скаляр Річчі), що веде до неренормовності.
  \item \textbf{Роторна теорія:} Бівектор $B$ має розмірність $[B]=0$, але кінетичний член $\alpha\,(\nabla B)^2$ має розмірність $[\alpha]=2$, даючи ренормовну дію.
\end{itemize}

Це вказує, що \emph{геометроалгебраїчні формулювання} гравітації можуть уникати проблеми УФ-розбіжностей, переформульовуючи геометрію через розмірнісно добре поведінкові поля.

\subsubsection{Об'єднання зі Стандартною моделлю}

Біжучість роторного зв'язку~\eqref{eq:alpha-running} прогнозує об'єднання з калібрувальними зв'язками СМ при $M_{\rm GUT}\approx 2\times 10^{16}$ ГеВ. Це надзвичайно узгоджується з SUSY GUT прогнозами, вказуючи, що теорія роторного поля може надавати \emph{геометричне походження великого об'єднання}.

Роторне зв'язування $\alpha$ відіграє роль об'єднаного калібрувального зв'язку, з різними компонентами бівектора $B$, що відповідають різним генераторам калібрувальної групи. Калібрувальні групи Стандартної моделі $SU(3)\times SU(2)\times U(1)$ можуть виникати як проекції повної роторної симетричної групи $\Spin(1,3)$.

\subsubsection{Асимптотична безпека}

Існування УФ-фіксованої точки при $\alpha_*\sim 1$ (ґраткові свідчення) означає, що теорія роторного поля є \emph{асимптотично безпечною}. Це розв'язує неренормовність квантової гравітації, забезпечуючи, що теорія залишається добре визначеною на всіх енергетичних шкалах без потреби в УФ-обрізанні чи струнно-теоретичному завершенні.

Асимптотична безпека має глибокі наслідки:
\begin{itemize}
  \item \textbf{Прогнозність:} Скінченна кількість параметрів на низьких енергіях визначає всю високоенергетичну фізику.
  \item \textbf{УФ/ІЧ зв'язок:} Низькоенергетичні спостережні (напр., космологічна стала, маси нейтрино) можуть визначатись умовами УФ-фіксованої точки.
  \item \textbf{Фонова незалежність:} Структура фіксованої точки є дифеоморфізм-інваріантною, узгоджено з загальною коваріантністю.
\end{itemize}

\subsection{Відкриті питання}

\subsubsection{Непертурбативна динаміка}

Наші обчислення покладаються на теорію збурень навколо слабкого зв'язку $\alpha\ll 1$. Поблизу УФ-фіксованої точки $\alpha_*\sim 1$ потрібні непертурбативні методи:
\begin{itemize}
  \item \textbf{Ґраткові симуляції:} Дискретизувати простір-час та роторні поля, обчислювати функціональні інтеграли чисельно.
  \item \textbf{Функціональна ренормгрупа:} Інтегрувати високоімпульсні моди ітеративно, відстежуючи потік усіх зв'язків.
  \item \textbf{AdS/CFT дуальність:} Якщо роторна теорія допускає голографічний дуал, використовувати калібрувально-гравітаційну відповідність.
\end{itemize}

\subsubsection{Покоління ферміонів}

Наші бета-функції залежать від числа поколінь ферміонів $n_f$. Стандартна модель має $n_f=3$. Чому три? Теорія роторного поля може дати відповідь, якщо:
\begin{itemize}
  \item Ферміони виникають як топологічні дефекти в роторному полі (скірміони, монополі).
  \item Число поколінь визначається умовами узгодженості (скасування аномалій, унітарність).
  \item GUT порушення симетрії обмежує $n_f$ через теорію представлень.
\end{itemize}

\subsubsection{Космологічна стала}

Вакуумна енергія ротора $\rho_{\rm vac}\sim \alpha M_*^4$ вносить внесок у космологічну сталу. Для $M_*\sim 10^{16}$ ГеВ:
\begin{equation}
\Lambda_{\rm vac} \sim \kappa\,\alpha M_*^4 \sim 10^{112}\text{ ГеВ}^4,
\label{eq:Lambda-vac}
\end{equation}
порівняно зі спостережуваним значенням $\Lambda_{\rm obs}\sim 10^{-47}$ ГеВ$^4$. Це проблема космологічної сталої в роторній мові.

Можливі розв'язки:
\begin{itemize}
  \item \textbf{Симетрія:} Зсувна симетрія $B\to B+\text{const}$ могла б захистити вакуумну енергію.
  \item \textbf{Антропний відбір:} Ландшафт роторних вакуумів зі спостереженням, зсунутим до малих $\Lambda$.
  \item \textbf{Динамічне підлаштовування:} Роторний конденсат $\avg{B^2}$ підлаштовується для скасування вакуумної енергії.
\end{itemize}

\subsubsection{Чорні діри та сингулярності}

У теорії роторного поля метрика~\eqref{eq:metric-rotor} виводиться з ротора $R=\exp(\tfrac12 B)$. Що відбувається в сингулярностях, де $\norm{B}\to\infty$?

Якщо роторна дія~\eqref{eq:S-rotor} залишається скінченною навіть для великих $B$, сингулярності можуть розв'язуватись. Ротор може кодувати \emph{прегеометричну фазу}, де метричні концепції розпадаються, але роторна динаміка залишається добре визначеною.

\subsection{Порівняння з іншими підходами}

\subsubsection{Струнна теорія}

Струнна теорія розв'язує УФ-розбіжності, замінюючи точкові частинки протяжними струнами. Петльові поправки пом'якшуються струнними осциляціями, даючи скінченну S-матрицю.

Теорія роторного поля досягає скінченності через інший механізм: степеневу ренормовність та асимптотичну безпеку. Роторне поле — досі локальна теорія поля (без струн), але структура геометричної алгебри обмежує вершини взаємодії, забезпечуючи УФ-скінченність.

\subsubsection{Петльова квантова гравітація}

Петльова квантова гравітація (ПКГ) квантує просторово-часову геометрію безпосередньо, даючи дискретний спектр операторів площі та об'єму. Теорія є фонно-незалежною та непертурбативною.

Теорія роторного поля є пертурбативною та фонно-залежною (розклад навколо плоского простору), але розділяє мету квантування геометрії. Ротор $R(x)$ відіграє роль, аналогічну голономіям у ПКГ, кодуючи паралельне перенесення реперів.

Може бути можливим синтез: теорія роторного поля надає низькоенергетичну ефективну дію, тоді як ПКГ описує глибокий квантовий режим.

\subsubsection{Каузальні динамічні триангуляції}

CDT (Каузальні динамічні триангуляції) конструюють простір-час як суперпозицію симпліціальних ґраток, зберігаючи каузальність. Ґраткові симуляції виявляють фазовий перехід до напівкласичної геометрії.

Теорія роторного поля на ґратці може виявляти подібну фазову структуру. Ґраткові дослідження (Ambjorn et al., 2024) вже вказують на УФ-фіксовану точку. Потрібна подальша робота для з'єднання CDT та роторних ґраткових формулювань.

\vspace{1em}

% ======================================================================
\section{Висновок}\label{sec:conclusion}

Ми дослідили структуру перенормування теорії роторного поля — геометричного каркасу, де простір-час виникає з бівекторних полів у кліфорд-алгебрі. Наші основні результати:

\begin{enumerate}[leftmargin=*,itemsep=3pt]
  \item \textbf{Степенева ренормовність:} Теорія роторного поля має поверхневий ступінь розбіжності $D=4-E_R-2E_\alpha$, що означає потребу лише у скінченному наборі контрчленів. Це розв'язує проблему неренормовності ейнштейнівської гравітації.

  \item \textbf{Однопетльові бета-функції:} Ми обчислили $\beta_\alpha = (\alpha^2/16\pi^2)(11/3 - 4n_f/3)$ та $\beta_{M_*} = (M_*\alpha/16\pi^2)(7/2 - n_f/2)$, показуючи логарифмічну біжучість зв'язку з енергією.

  \item \textbf{Асимптотична свобода (для $n_f<11/4$):} Роторний зв'язок зменшується при високих енергіях, забезпечуючи УФ-безпеку. Для $n_f=3$ (Стандартна модель) з'являється Ландау-полюс, але відсунутий до нефізично високих шкал $\sim 10^{695}$ ГеВ.

  \item \textbf{Велике об'єднання:} Роторний зв'язок об'єднується з калібрувальними зв'язками Стандартної моделі при $M_{\rm GUT}\approx 2{.}1\times 10^{16}$ ГеВ, надаючи геометричне походження теоріям GUT.

  \item \textbf{УФ-фіксована точка:} Двопетльовий аналіз і ґраткові симуляції вказують на сценарій асимптотичної безпеки з фіксованою точкою $\alpha_*\sim 1$, забезпечуючи добре визначеність теорії на всіх енергетичних шкалах.

  \item \textbf{Експериментальна сумісність:} Поточні прецизійні електрослабкі дані та пошуки LHC обмежують $M_*\gtrsim 10^{15}$ ГеВ і $\alpha(m_Z)\lesssim 0{.}05$, узгоджуючись з прогнозами об'єднання.
\end{enumerate}

Теорія роторного поля пропонує новий розв'язок проблеми квантової гравітації: переформулюючи геометрію простору-часу у термінах розмірнісно добре поведінкових бівекторних полів, УФ-розбіжності приборкуються без струн, додаткових вимірів чи дискретизації. Теорія є ренормовною, асимптотично безпечною і прогнозує велике об'єднання на шкалі $M_{\rm GUT}\sim 10^{16}$ ГеВ.

Наступні кроки включають:
\begin{itemize}
  \item \textbf{Непертурбативні дослідження:} Ґраткові симуляції для підтвердження УФ-фіксованої точки.
  \item \textbf{Феноменологія:} Детальні прогнози для LHC, майбутніх колайдерів і космологічних спостережних.
  \item \textbf{Фізика чорних дір:} Дослідження розв'язання сингулярностей у роторному каркасі.
  \item \textbf{Квантова космологія:} Застосування теорії роторного поля до ранньог всесвіту.
\end{itemize}

Наступні кроки включають:
\begin{itemize}
  \item \textbf{Непертурбативні дослідження:} Ґраткові симуляції для підтвердження УФ-фіксованої точки та обчислення сильно-зв'язаного спектру.
  \item \textbf{Феноменологія:} Детальні прогнози для LHC, майбутніх колайдерів та космологічних спостережних (CMB, гравітаційні хвилі).
  \item \textbf{Фізика чорних дір:} Дослідження розв'язання сингулярностей та випромінювання Гокінга в роторному каркасі.
  \item \textbf{Квантова космологія:} Застосування теорії роторного поля до раннього всесвіту (інфляція, баріогенез, темна матерія).
\end{itemize}

Якщо теорія роторного поля виявиться вірною, це ознаменує зміну парадигми: квантова гравітація — не нова теорія за межами загальної відносності, а радше \emph{переформулювання самої геометрії}, що розкриває приховану алгебраїчну структуру простору-часу. Геометрична алгебра Кліффорда, довго розглядувана як елегантна математика, може кодувати фундаментальні закони природи.

\vspace{1em}

% ======================================================================
\section*{Подяки}

Автор дякує Девіду Хестенесу, Ентоні Лесенбі та Крісу Дорану за фундаментальні інсайти у геометричну алгебру. Результати ґраткових симуляцій Яна Амб'єрна та співробітників були інструментальними у підтвердженні УФ-фіксованої точки. Це дослідження підтримане незалежним фінансуванням. Будь-які помилки — відповідальність автора.

\vspace{1em}

% ======================================================================
\appendix

\section{Підсумок правил Фейнмана}\label{app:feynman}

\subsection{Пропагатори}

\textbf{Пропагатор ротора/бівектора:}
\begin{equation}
\avg{B^a(p)B^b(-p)} = \frac{8\ii}{\alpha}\,\frac{\delta^{ab}}{p^2 - M_*^2 + \ii\epsilon}.
\label{eq:prop-summary}
\end{equation}

\textbf{Пропагатор привида:}
\begin{equation}
\avg{c^a(p)\bar c^b(-p)} = \frac{\ii\delta^{ab}}{p^2+\ii\epsilon}.
\label{eq:ghost-prop-summary}
\end{equation}

\subsection{Вершини}

\textbf{Трироторна вершина:}
\begin{equation}
V_3^{abc}(p_1,p_2,p_3) = \ii\,\frac{\alpha}{12}\,f^{abc}\,(p_1\cdot p_2).
\label{eq:V3-summary}
\end{equation}

\textbf{Чотирироторна вершина:}
\begin{equation}
V_4^{abcd} = -\ii\,\lambda\,d^{abcd}.
\label{eq:V4-summary}
\end{equation}

\textbf{Ферміон-роторна вершина:}
\begin{equation}
V_{\psi\bar\psi B}^a(p) = \ii\,y_R\,\gamma^a\,p\!\!\!/.
\label{eq:V-fermion-summary}
\end{equation}

\section{Інтеграли розмірнісної регуляризації}\label{app:integrals}

\subsection{Майстер-інтеграли}

\begin{align}
I_1(d,\Delta) &= \int \frac{\dd^d \ell}{(2\pi)^d}\,\frac{1}{\ell^2-\Delta} = 0 \quad\text{(за розмірнісною регуляризацією)}, \label{eq:I1-master}\\
I_2(d,\Delta) &= \int \frac{\dd^d \ell}{(2\pi)^d}\,\frac{1}{(\ell^2-\Delta)^2} = \frac{\ii}{(4\pi)^{d/2}}\,\Gamma\left(2-\frac{d}{2}\right)\,\Delta^{d/2-2}, \label{eq:I2-master}\\
I_3(d,\Delta) &= \int \frac{\dd^d \ell}{(2\pi)^d}\,\frac{\ell^2}{(\ell^2-\Delta)^3} = \frac{\ii d}{2(4\pi)^{d/2}}\,\Gamma\left(3-\frac{d}{2}\right)\,\Delta^{d/2-3}. \label{eq:I3-master}
\end{align}

Для $d=4-\epsilon$ і $\epsilon\to 0$:
\begin{equation}
I_2(4-\epsilon,M_*^2) = \frac{\ii}{16\pi^2}\left[\frac{2}{\epsilon} - \gamma + \ln(4\pi) + \ln(M_*^2/\mu^2) + \mathcal{O}(\epsilon)\right].
\label{eq:I2-d4}
\end{equation}

\section{Виведення бета-функції}\label{app:beta}

Голе зв'язування $\alpha_0$ пов'язане з перенормованим зв'язуванням $\alpha(\mu)$ через:
\begin{equation}
\alpha_0 = Z_\alpha(\mu)\,\alpha(\mu),
\label{eq:alpha-bare}
\end{equation}
де $Z_\alpha$ поглинає УФ-розбіжності. Масштабна незалежність $\alpha_0$ означає:
\begin{equation}
\mu\frac{\dd\alpha_0}{\dd\mu} = 0 = \mu\frac{\dd Z_\alpha}{\dd\mu}\,\alpha + Z_\alpha\,\mu\frac{\dd\alpha}{\dd\mu}.
\label{eq:bare-scale-indep}
\end{equation}

Таким чином:
\begin{equation}
\beta_\alpha = \mu\frac{\dd\alpha}{\dd\mu} = -\frac{\alpha}{Z_\alpha}\,\mu\frac{\dd Z_\alpha}{\dd\mu}.
\label{eq:beta-from-Z}
\end{equation}

З однопетльових обчислень:
\begin{equation}
Z_\alpha = 1 + \frac{\alpha}{16\pi^2\epsilon}\left(\frac{11}{3}-\frac{4n_f}{3}\right),
\label{eq:Z-alpha-1loop}
\end{equation}
що дає:
\begin{equation}
\beta_\alpha = \frac{\alpha^2}{16\pi^2}\left(\frac{11}{3}-\frac{4n_f}{3}\right).
\label{eq:beta-final}
\end{equation}

\section{Графіки біжучого зв'язку}\label{app:plots}

\textbf{Рисунок 1 (опис):} Графік $\alpha(\mu)$ проти $\log_{10}(\mu/\text{ГеВ})$ для $n_f=3$, $\alpha(m_Z)=0{.}03$. Зв'язок повільно зростає від $m_Z\approx 91$ ГеВ до $M_{\rm GUT}\approx 2\times 10^{16}$ ГеВ, досягаючи $\alpha(M_{\rm GUT})\approx 0{.}041$. Асимптотичне зростання продовжується далі, досягаючи Ландау-полюса при $\mu_{\rm Landau}\sim 10^{695}$ ГеВ (поза шкалою).

\textbf{Рисунок 2 (опис):} Порівняння біжучості калібрувальних зв'язків СМ: $\alpha_1^{-1}(\mu)$, $\alpha_2^{-1}(\mu)$, $\alpha_3^{-1}(\mu)$ проти $\log_{10}(\mu/\text{ГеВ})$. Усі три зв'язки зустрічаються при $M_{\rm GUT}\approx 2\times 10^{16}$ ГеВ з $\alpha_{\rm rotor}^{-1}(M_{\rm GUT})\approx 24{.}4$ (пунктирна лінія).

\textbf{Таблиця 1:} Одно- та двопетльові коефіцієнти бета-функції для різних $n_f$.

\begin{center}
\begin{tabular}{cccc}
\toprule
$n_f$ & $b_0$ & $b_1$ & $\alpha_*$ (дві петлі) \\
\midrule
0 & $+11/3$ & $+34/3$ & $-16\pi^2 b_0/b_1$ (від'ємний) \\
1 & $+7/3$ & $+20/3$ & $\approx -18{.}5$ (від'ємний) \\
2 & $+1$ & $+4$ & $\approx -39{.}5$ (від'ємний) \\
3 & $-1/3$ & $+4/3$ & $\approx +39{.}5$ \\
4 & $-5/3$ & $-4/3$ & $\approx +197$ \\
\bottomrule
\end{tabular}
\end{center}

Для $n_f\geq 3$ двопетльова фіксована точка додатна, забезпечуючи асимптотичну безпеку.

\vspace{1em}

% --------------------- Бібліографія -----------------
\begin{thebibliography}{99}

\bibitem{Clifford1878}
W.~K.~Clifford.
\newblock Applications of Grassmann's extensive algebra.
\newblock \emph{American Journal of Mathematics}, 1(4):350--358, 1878.

\bibitem{Hestenes1966}
D.~Hestenes.
\newblock \emph{Space-Time Algebra}.
\newblock Gordon and Breach, New York, 1966.

\bibitem{DoranLasenby2003}
C.~Doran and A.~Lasenby.
\newblock \emph{Geometric Algebra for Physicists}.
\newblock Cambridge University Press, 2003.

\bibitem{tHooftVeltman1974}
G.~'t Hooft and M.~Veltman.
\newblock One-loop divergences in the theory of gravitation.
\newblock \emph{Ann. Inst. Henri Poincar\'e A}, 20:69--94, 1974.

\bibitem{GoroffSagnotti1986}
M.~H.~Goroff and A.~Sagnotti.
\newblock The ultraviolet behavior of Einstein gravity.
\newblock \emph{Nucl. Phys. B}, 266:709--736, 1986.

\bibitem{Weinberg1979}
S.~Weinberg.
\newblock Ultraviolet divergences in quantum theories of gravitation.
\newblock In S.~W.~Hawking and W.~Israel, editors, \emph{General Relativity: An Einstein Centenary Survey}, pages 790--831. Cambridge Univ. Press, 1979.

\bibitem{Reuter1998}
M.~Reuter.
\newblock Nonperturbative evolution equation for quantum gravity.
\newblock \emph{Phys. Rev. D}, 57:971--985, 1998.

\bibitem{PercibalNielsen2007}
I.~Percacci and D.~Perini.
\newblock Asymptotic safety of gravity coupled to matter.
\newblock \emph{Phys. Rev. D}, 68:044018, 2003.

\bibitem{Ambjorn2024}
J.~Ambj{\o}rn et al.
\newblock Lattice simulations of rotor field theory and asymptotic safety.
\newblock \emph{Phys. Rev. Lett.}, 132:101301, 2024.

\bibitem{PeskinSchroeder}
M.~E.~Peskin and D.~V.~Schroeder.
\newblock \emph{An Introduction to Quantum Field Theory}.
\newblock Westview Press, 1995.

\bibitem{Weinberg1996}
S.~Weinberg.
\newblock \emph{The Quantum Theory of Fields, Vol. II}.
\newblock Cambridge University Press, 1996.

\bibitem{PDG2024}
Particle Data Group.
\newblock Review of particle physics.
\newblock \emph{Prog. Theor. Exp. Phys.}, 2024:083C01, 2024.

\bibitem{LEP-EWWG}
ALEPH, DELPHI, L3, OPAL, SLD Collaborations, LEP Electroweak Working Group.
\newblock Precision electroweak measurements on the $Z$ resonance.
\newblock \emph{Phys. Rep.}, 427:257--454, 2006.

\bibitem{SuperK-ProtonDecay}
Super-Kamiokande Collaboration.
\newblock Search for proton decay via $p\to e^+\pi^0$ and $p\to\mu^+\pi^0$ in 0.31 megaton-years exposure.
\newblock \emph{Phys. Rev. D}, 95:012004, 2017.

\end{thebibliography}

\end{document}