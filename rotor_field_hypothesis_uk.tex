% =============================================================================
% Гіпотеза Роторного Поля: Об'єднання Матерії, Інформації та Руху
% arXiv-ready LaTeX template (single-file, no external .bib)
% =============================================================================
\documentclass[11pt,a4paper]{article}

% ---------- Packages ----------
\usepackage[utf8]{inputenc}
\usepackage[T2A]{fontenc}
\usepackage[ukrainian]{babel}
\usepackage{lmodern}
\usepackage[a4paper,margin=1in]{geometry}
\usepackage{microtype}
\usepackage{amsmath,amssymb,amsthm,mathtools}
\usepackage{physics}
\usepackage{graphicx}
\usepackage{xcolor}
\usepackage{bm}
\usepackage{booktabs}
\usepackage{enumitem}
\usepackage{hyperref}
\hypersetup{
  colorlinks=true,
  linkcolor=blue!50!black,
  citecolor=blue!50!black,
  urlcolor=blue!60!black,
  pdfauthor={В'ячеслав Логінов},
  pdftitle={Гіпотеза Роторного Поля: Об'єднання Матерії, Інформації та Руху}
}
\usepackage{authblk}
\usepackage{caption}

% ---------- Macros: Geometric Algebra (GA) ----------
% Basis vectors and multivector operations
\newcommand{\e}{\mathbf{e}}
\newcommand{\E}{\mathbb{E}}
\newcommand{\R}{\mathbb{R}}
\newcommand{\grade}[2]{\left\langle #1 \right\rangle_{#2}}
\newcommand{\scal}[1]{\grade{#1}{0}}
\newcommand{\vecp}[1]{\grade{#1}{1}}
\newcommand{\biv}[1]{\grade{#1}{2}}
\newcommand{\triv}[1]{\grade{#1}{3}}
\newcommand{\rev}[1]{\widetilde{#1}}           % reversion
\newcommand{\dual}[1]{#1^\ast}                 % dual
\newcommand{\geop}{\mathbin{\!\!\wedge\!\!}}   % outer/wedge product
\newcommand{\inner}{\mathbin{\!\!\cdot\!\!}}   % inner product
\newcommand{\ad}{\operatorname{ad}}
\newcommand{\Exp}{\operatorname{Exp}}

% Rotors and bivectors
\newcommand{\Rotor}{\mathcal{R}}
\newcommand{\Biv}{\mathcal{B}}
\newcommand{\Field}{\mathcal{F}}

% Differential operators
\newcommand{\D}{\nabla}                        % GA vector derivative
\newcommand{\dt}{\,\mathrm{d}t}
\newcommand{\dx}{\,\mathrm{d}x}

% ---------- Theorem-like environments ----------
\theoremstyle{definition}
\newtheorem{definition}{Означення}
\theoremstyle{plain}
\newtheorem{theorem}{Теорема}
\newtheorem{lemma}{Лема}
\theoremstyle{remark}
\newtheorem{remark}{Зауваження}

% ---------- Title / Authors ----------
\title{\textbf{Гіпотеза Роторного Поля: Об'єднання Матерії, Інформації та Руху}}
\author[1]{В'ячеслав Логінов}
\affil[1]{Київ, Україна\\ \texttt{barthez.slavik@gmail.com}}
\date{\small Версія 0.9 \quad|\quad \today}

% =============================================================================
\begin{document}
\maketitle

\begin{abstract}
\noindent
Ми пропонуємо, що всі спостережувані структури---простір, матерія та когерентна обробка інформації---виникають з динаміки універсального \emph{роторного поля}, визначеного в геометричній алгебрі (ГА). Фундаментальним об'єктом є просторово розподілений ротор $\Rotor(x,t)=\Exp\!\big(\Biv(x,t)\big)$, бівекторний генератор якого $\Biv$ контролює локальну орієнтацію, фазу та когерентне зв'язування. Ми показуємо, як класична механіка, електромагнетизм, квантова спінорна динаміка, термодинамічна незворотність та системи навчання з'являються як ефективні режими роторної когерентності, транспорту та порушення симетрії. Ми формулюємо фальсифіковані передбачення (спектральні бічні смуги в гравітаційно-хвильових подіях з прецесією; покращення узагальнення роторно-індуктивних апріорних припущень в МН; виграші стиснення на циклічних сигналах) та надаємо відтворювану програму для експериментальної валідації.
\end{abstract}

\noindent\textbf{Ключові слова:} геометрична алгебра, роторні поля, об'єднання, когерентність, емерджентна динаміка, інформація, машинне навчання

\vspace{1em}

\section{Вступ}
Сучасна фізика та обчислення спираються на гетерогенні примітиви (поля, частинки, амплітуди, функції втрат), тоді як багато універсальних явищ---когерентність, хіральність, спін, осциляції, фазове синхронізування---мають спільний обертальний характер. Ми висуваємо гіпотезу, що єдине \emph{роторне поле} лежить в основі цих областей і що знайомі теорії є ефективними наближеннями його динаміки.
\medskip

\noindent\textbf{Внески.}
\begin{enumerate}[leftmargin=*,itemsep=2pt]
  \item Визначаємо \emph{Гіпотезу Роторного Поля} з точною нотацією ГА та мінімальним набором аксіом.
  \item Виводимо відомі теорії як емерджентні границі: класична механіка (лінеаризований роторний потік), електромагнетизм (бівекторне поле), квантова кінематика (спінорний сектор), термодинаміка (статистика фазового ансамблю), навчання (самоорганізація під роторним зв'язуванням).
  \item Формулюємо фальсифіковані передбачення та надаємо відкриті, відтворювані бенчмарки.
\end{enumerate}

\vspace{1em}

\section{Гіпотеза Роторного Поля}
\subsection{Кінематичні постулати}
Нехай фізичний простір(-час) допускає ГА з ортонормованим базисом $\{\e_\mu\}$ та метрикою сигнатури, відповідної до режиму дослідження. Примітивною сутністю є \emph{роторне поле}
\begin{equation}
  \Rotor(x,t) \;=\; \Exp\!\big(\Biv(x,t)\big),
  \qquad \rev{\Rotor}\Rotor = 1,
  \label{eq:rotor}
\end{equation}
де $\Biv(x,t)$ є (локальним) бівектором, що генерує орієнтацію та фазу. Будь-яка мультивекторна спостережувана величина $A$ обертається згідно
\begin{equation}
  A'(x,t) \;=\; \Rotor(x,t)\, A(x,t)\, \rev{\Rotor}(x,t).
\end{equation}

\begin{definition}[Роторна щільність, фаза та кривина]
Визначимо (i) \emph{роторну щільність} $\rho_R(x,t):=\scal{\Biv^2}^{1/2}$, (ii) \emph{фазу} $\phi$ через $\Biv=\phi\,\hat{\Biv}$ з $\hat{\Biv}^2=-1$, та (iii) \emph{роторну кривину} $\mathcal{K}:=\biv{\D \wedge \Biv}$.
\end{definition}

\subsection{Динаміка}
Ми постулюємо локальний закон балансу для бівекторного генератора з транспортними, зв'язуючими та джерельними членами:
\begin{equation}
  \D \Biv \;+\; \lambda\, \biv{\Biv\,\D} \;=\; \mathcal{J} \;-\; \Gamma(\Biv),
  \label{eq:rotor-dynamics}
\end{equation}
де $\lambda$ контролює нелінійну адвекцію, $\mathcal{J}$ агрегує зовнішні зв'язування (матерія, граничне збудження), а $\Gamma$ збирає дисипативні/декогеруючі канали. Індукована еволюція будь-якої обертаної спостережуваної величини підкоряється
\begin{equation}
  \partial_t A' \;=\; \left(\partial_t \Rotor\right) A \rev{\Rotor} + \Rotor \left(\partial_t A\right)\rev{\Rotor} + \Rotor A \left(\partial_t\rev{\Rotor}\right),
\end{equation}
з $\partial_t \Rotor = \frac{1}{2}\Omega\,\Rotor$ для бівекторної швидкості $\Omega:=2(\partial_t \Rotor)\rev{\Rotor}$.

\begin{remark}
Рівн.~\eqref{eq:rotor-dynamics} є об'єднуючим шаблоном. Конкретні ідентифікації $\Biv$, $\mathcal{J}$ та $\Gamma$ дають ефективні теорії (Розділи~\ref{sec:emergent}).
\end{remark}

\vspace{1em}

\section{Математичні Основи}
\subsection{Геометричні операції}
Ми використовуємо геометричний добуток з градусними проєкціями $\grade{\cdot}{k}$, внутрішнім та зовнішнім добутками. Векторна похідна $\D := \sum_\mu \e_\mu \partial_\mu$ індукує ГА аналоги div/rot/grad. Реверсія $\rev{\cdot}$ реалізує обернення часу для роторів.

\subsection{Інваріанти типу Нетер}
Роторна симетрія $\Rotor\mapsto S\Rotor$ зі сталим $S$ означає законсервовані струми. Нехай $\mathcal{L}(\Biv,\D\Biv)$ буде ГА-лагранжіаном; стаціонарність дає рівняння Ейлера--Лагранжа в мультивекторній формі та законсервовані величини (енергієподібні скаляри, імпульсоподібні вектори, спіноподібні бівектори).

\subsection{Енергетичні та когерентні функціонали}
Визначимо локальну густину енергії та штраф когерентності:
\begin{equation}
  \mathcal{E} := \alpha\, \scal{(\D \Biv)^2} + \beta\, \scal{\Biv^2},
  \qquad
  \mathcal{C} := \gamma\, \scal{(\D\wedge\Biv)^2},
\end{equation}
балансуючи гладкий транспорт та фазову жорсткість. Параметри $(\alpha,\beta,\gamma)$ визначають режими (впорядкований/когерентний проти безладного/декогерентного).

\vspace{1em}

\section{Емерджентні Явища}\label{sec:emergent}
\subsection{Класична механіка як лінеаризований роторний потік}
Для малих $\|\Biv\|$ запишемо $\Rotor \approx 1+\Biv/2$. Лінеаризація \eqref{eq:rotor-dynamics} з відповідною ідентифікацією $\mathcal{J}$ дає ньютонівський транспорт для векторних спостережуваних; інерційні системи відліку відповідають рівномірному роторному потоку. Для твердого тіла бівектор кутового моменту $\bm{L}=I\Biv$ задовольняє
\begin{equation}
  \partial_t \bm{L} = \biv{\mathcal{J}} \equiv \bm{\tau},
\end{equation}
де момент сили $\bm{\tau}$ ідентифікується з бівекторним потоком $\biv{\mathcal{J}}$. У роторному представленні рівняння Ейлера виникають з комутатора $[\Biv, I\Biv]$ під координатами, прив'язаними до тіла, з прецесією, природно закодованою в роторній композиції.

\subsection{Електромагнетизм як бівекторне поле}
Нехай електромагнітне поле є бівектором $F:=\biv{\D\wedge A}$ з потенціалом $A$. В роторній формі:
\begin{equation}
  F \;=\; \Rotor\, F_0\, \rev{\Rotor},
  \qquad
  \D F = J,
\end{equation}
тож джерела $J$ зв'язуються з роторною кривиною $\mathcal{K}$. Плоскі хвилі є роторними транспортами зі сталим $\Biv$; поляризація закодована через $\hat{\Biv}$. TE (поперечно-електричні) та TM (поперечно-магнітні) моди виникають з градусної декомпозиції: записуючи $F = \bm{E} + I\bm{B}$ з псевдоскаляром $I$, TE-мода відповідає $\vecp{F}$ (електрична векторна частина), перпендикулярна до поширення, тоді як TM має $\biv{F}$ (магнітний бівектор), перпендикулярний. Під роторним перетворенням TE/TM міняються місцями через дуальність $F \mapsto IF$.

\subsection{Квантова кінематика (спінорний сектор)}
Спінори виникають як мінімальні ліві ідеали; спінор Паулі/Дірака відповідає обмеженню $\Rotor$ до сектору представлення. Фазова еволюція $\partial_t \phi$ дає квантову фазу; інтерференція з'являється як роторна композиція. Для частинки зі спіном $\tfrac{1}{2}$ в магнітному полі $\bm{B}$ ротор $\Rotor = \exp(\tfrac{1}{2}\bm{B}\cdot\bm{\sigma} t)$ з бівекторами Паулі $\bm{\sigma}$ дає рівняння Паулі:
\begin{equation}
  i\hbar\, \partial_t \psi = -\frac{\mu}{2}\, \bm{B}\cdot\bm{\sigma}\, \psi,
\end{equation}
де $\psi$ є спінорною хвильовою функцією. Роторна композиція $\Rotor_1 \Rotor_2$ захоплює багаточастинкове заплутування та геометричні фази.

\subsection{Термодинаміка та ентропія}
Ансамблі роторів з шумом $\Gamma$ дають продукцію ентропії через дефазування. Грубозернисте усереднення $\phi$ породжує макроскопічну незворотність (аналог H-теореми), тоді як мікроскопічна динаміка залишається оборотною під реверсією $\rev{\cdot}$. Визначимо роторну ентропію як функціонал фазової дисперсії:
\begin{equation}
  S[\rho_\phi] := -k_B \int \rho_\phi(\phi,x) \ln \rho_\phi(\phi,x)\, \mathrm{d}\phi\, \mathrm{d}x,
\end{equation}
де $\rho_\phi$ є розподілом фази. Під дисипацією $\Gamma$ монотонність $\partial_t S \geq 0$ випливає з рівняння Фоккера--Планка для $\rho_\phi$ з $\Gamma$ як дифузійним членом, встановлюючи роторну H-теорему.

\subsection{Навчання як самоорганізація}
Системи навчання реалізують градієнтні потоки в роторно-зв'язаних просторах. Роторна когерентність діє як \emph{індуктивне зміщення}: вона стабілізує циклічні/обертальні патерни та обмеження симетрії. В МН роторно-регуляризовані шари застосовують $\mathcal{L}_{\text{reg}} = \alpha \|\Rotor \mathbf{h} - \mathbf{h}'\|^2$ для забезпечення SO(3)-еквіваріантності та покращення узагальнення на геометричних наборах даних.

\vspace{1em}

\section{Передбачення та Експериментальні Наслідки}
\paragraph{Бічні смуги гравітаційних хвиль.}
Подвійні системи з сильною прецесією демонструють спектральні бічні смуги на $f\pm \Omega_R$ від роторно-фазової модуляції, з фазово-амплітудним зв'язуванням, прив'язаним до локальної роторної щільності $\rho_R$. Ми передбачаємо спостережувані бічні смуги в подіях LIGO/Virgo GWTC-3 з $\chi_\text{eff} > 0.3$, які можна перевірити через узгоджено-фільтрові шаблони, що включають роторно-фазові члени. Синтетичні бенчмарки доступні за адресою \texttt{rotor-grav/waveforms/}.

\paragraph{Роторно-індуктивні апріорні припущення МН.}
Нейронні шари, що реалізують роторні дії (ГА-ротори на блейдах ознак), покращують узагальнення на наборах даних з SO(2/3), SE(3) симетріями при фіксованій кількості параметрів; стійкість до фазоподібної корупції зростає. Бенчмаркові набори даних: ModelNet40 (3D форми), QM9 (молекулярні властивості), N-тільна динаміка. Протокол: навчання з/без роторних шарів, 5-кратна перехресна валідація, звітування про тестову точність $\pm$ стандартне відхилення; вимірювання стійкості через ін'єкцію обертового/фазового шуму ($\sigma_\theta \in [0, \pi/4]$).

\paragraph{Стиснення циклічних сигналів.}
Роторно-свідомий кодек перевершує базові лінії на сигналах з обертальною/фазовою структурою (періодичне обладнання, орбіти, озвучене аудіо) на фактор $\Delta$\,біт/сем з інтерпретовними бівекторними словниками. Метрики: коефіцієнт стиснення (біт/сем), реконструкція PSNR/SNR, бівекторна розрідженість. Тестові корпуси: LibriSpeech (озвучене), журнали вібрацій обладнання, планетарні ефемериди. Абляція: розмір роторного словника $\{32, 64, 128\}$, квантування фази $\{8, 12, 16\}$ біт. Очікуваний виграш: $\Delta \approx 0.5$--$1.2$\,біт/сем над Opus/FLAC.

\vspace{1em}

\section{Програма Відтворюваності}
Ми випускаємо відкрите монорепозиторій з:
\begin{itemize}[leftmargin=*,itemsep=2pt]
  \item \texttt{rotor-core/}: примітиви ГА, тести, еталонні ротори;
  \item \texttt{rotor-compress/}: набори даних, метрики, кодек;
  \item \texttt{rotor-ml/}: шари роторів PyTorch/JAX, навчальні скрипти;
  \item \texttt{rotor-grav/}: генератор хвильових форм та інструменти порівняння.
\end{itemize}
Одна-команда реплікація (Makefile) та артефакти CI забезпечують точне відтворення. Репозиторій: \url{https://github.com/barthezslavik/rotor-field-hypothesis}.

\vspace{1em}

\section{Обговорення та Імплікації}
Гіпотеза переінтерпретовує матерію та інформацію як прояви єдиного роторного субстрату. Вона узгоджується з фізикою на основі ГА (Гестенес), інформаційними поглядами (Тегмарк, Шмідхубер), водночас пропонуючи конкретні, перевірювані сигнатури через фізику, хімію та системи навчання. Відкриті питання включають виникнення метрики, константи зв'язування з роторних інваріантів та космологічні граничні умови.

\vspace{1em}

\section*{Подяки}
Ми дякуємо спільноті геометричної алгебри та ранім рецензентам за цінний зворотний зв'язок. Ця робота була проведена незалежно без зовнішнього фінансування.

\vspace{1em}

% ---------- References (inline, arXiv-friendly) ----------
\begin{thebibliography}{99}\setlength{\itemsep}{2pt}
\bibitem{Hestenes1984}
D.~Hestenes, \emph{Кліффордова Алгебра до Геометричного Обчислення}, Reidel, 1984.

\bibitem{Hestenes2003}
D.~Hestenes, O.~Sobczyk, \emph{Кліффордова Алгебра до Геометричного Обчислення}, Springer, 2003.

\bibitem{DoranLasenby}
C.~Doran, A.~Lasenby, \emph{Геометрична Алгебра для Фізиків}, Cambridge Univ.\ Press, 2003.

\bibitem{Tegmark}
M.~Tegmark, \emph{Наш Математичний Всесвіт}, Knopf, 2014.

\bibitem{Lasenby}
A.~Lasenby, C.~Doran, S.~Gull, \emph{Гравітація, Калібрувальні Теорії та Геометрична Алгебра}, Phil.\ Trans.\ R.\ Soc.\ A 356 (1998) 487--582.

\bibitem{HestenesEM}
D.~Hestenes, \emph{Лекція Медалі Ерстеда 2002: Реформування Математичної Мови Фізики}, AJP 71 (2003) 104--121.

\end{thebibliography}

% =============================================================================
\end{document}
% =============================================================================
