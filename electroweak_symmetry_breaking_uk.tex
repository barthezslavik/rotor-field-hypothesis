% =============================================================================
% Розбиття електрослабкої симетрії з динаміки роторного поля
% arXiv-ready LaTeX (один файл, без зовнішнього .bib)
% =============================================================================
\documentclass[11pt,a4paper]{article}

% ---------- Пакети ----------
\usepackage[utf8]{inputenc}
\usepackage[T1,T2A]{fontenc}
\usepackage[ukrainian]{babel}
\usepackage{lmodern}
\usepackage[a4paper,margin=1in]{geometry}
\usepackage{microtype}
\usepackage{amsmath,amssymb,amsthm,mathtools}
\usepackage{physics}
\usepackage{graphicx}
\usepackage{xcolor}
\usepackage{bm}
\usepackage{booktabs}
\usepackage{enumitem}
\usepackage{hyperref}
\hypersetup{
  colorlinks=true,
  linkcolor=blue!50!black,
  citecolor=blue!50!black,
  urlcolor=blue!60!black,
  pdfauthor={Viacheslav Loginov},
  pdftitle={Розбиття електрослабкої симетрії з динаміки роторного поля}
}
\usepackage{authblk}
\usepackage{caption}

% ---------- Макроси: Геометрична алгебра (GA) ----------
% Базисні вектори та операції з мультивекторами
\newcommand{\e}{\mathbf{e}}
\newcommand{\E}{\mathbb{E}}
\newcommand{\R}{\mathbb{R}}
\newcommand{\grade}[2]{\left\langle #1 \right\rangle_{#2}}
\newcommand{\scal}[1]{\grade{#1}{0}}
\newcommand{\vecp}[1]{\grade{#1}{1}}
\newcommand{\biv}[1]{\grade{#1}{2}}
\newcommand{\triv}[1]{\grade{#1}{3}}
\newcommand{\rev}[1]{\widetilde{#1}}           % реверсія
\newcommand{\dual}[1]{#1^\ast}                 % дуал
\newcommand{\geop}{\mathbin{\!\!\wedge\!\!}}   % зовнішній добуток (wedge)
\newcommand{\inner}{\mathbin{\!\!\cdot\!\!}}   % скалярний (внутрішній) добуток
\newcommand{\ad}{\operatorname{ad}}
\newcommand{\Exp}{\operatorname{Exp}}

% Ротори та бівектори
\newcommand{\Rotor}{\mathcal{R}}
\newcommand{\Biv}{\mathcal{B}}
\newcommand{\Field}{\mathcal{F}}

% Диференційні оператори
\newcommand{\D}{\nabla}                        % векторна похідна GA
\newcommand{\dt}{\,\mathrm{d}t}
\newcommand{\dx}{\,\mathrm{d}x}

% Макроси для електрослабкої теорії
\newcommand{\SU}{\mathrm{SU}}
\newcommand{\UU}{\mathrm{U}}
\newcommand{\SO}{\mathrm{SO}}
% Lagrangian
\newcommand{\Lag}{\mathcal{L}}

% ---------- Оточення теорем ----------
\theoremstyle{definition}
\newtheorem{definition}{Означення}
\theoremstyle{plain}
\newtheorem{theorem}{Теорема}
\newtheorem{lemma}{Лема}
\newtheorem{proposition}{Твердження}
\theoremstyle{remark}
\newtheorem{remark}{Зауваження}

% ---------- Назва / Автори ----------
\title{\textbf{Розбиття електрослабкої симетрії з динаміки роторного поля: \\
Виведення механізму Гіггса з бівекторної когерентності}}
\author[1]{Viacheslav Loginov}
\affil[1]{Kyiv, Ukraine\\ \texttt{barthez.slavik@gmail.com}}
\date{\small Версія 1.0 \quad|\quad 15 жовтня 2025}

% =============================================================================
\begin{document}
\maketitle

\begin{abstract}
\noindent
Електрослабка теорія об’єднує електромагнітну та слабку взаємодії через спонтанне порушення симетрії, коли поле Гіггса набуває вакуумного середнього та генерує маси для бозонів $W$ і $Z$. Однак походження цього розбиття залишається феноменологічним: потенціал Гіггса постулюється, а не виводиться з глибших принципів. Ми показуємо, що весь електрослабкий сектор виникає з динаміки фундаментального роторного поля, визначеного в геометричній алгебрі. Шестивимірний простір бівекторів природно розкладається на генератори $\SU(2)$ (просторові бівектори) та $\UU(1)$ гіперзаряду (змішування часопросторових бівекторів). Спонтанне порушення симетрії постає, коли роторне поле розвиває ненульову когерентність $\ev{\Rotor} \neq 1$, що дає вакуумне середнє $v = 246$ ГеВ, визначене параметром жорсткості ротора $M_\ast$. Із бівекторної динаміки ми виводимо точні маси калібрувальних бозонів: $m_W = gv/2 \approx 80{.}4$ ГеВ та $m_Z = m_W/\cos\theta_W \approx 91{.}2$ ГеВ, де $\sin^2\theta_W \approx 0{.}231$ — кут слабкого змішування. Маси ферміонів виникають через ротор-ферміонні Юкава-взаємодії, а їх ієрархія — з топологічних чисел намотування ротора. Рамка передбачає відхилення у перерізах народження Гіггса на колайдерах, модифікації точних електрослабких параметрів $(S,T,U)$ і характерні сигнатури у потрійних калібрувальних зв’язках. Усі результати випливають з єдиного постулату: фізичний простір допускає бівекторне поле $\Biv(x,t)$, чия когерентна динаміка породжує масу.
\end{abstract}

\noindent\textbf{Ключові слова:} розбиття електрослабкої симетрії, механізм Гіггса, роторні поля, геометрична алгебра, породження маси, спонтанне порушення симетрії

\vspace{1em}

\section{Вступ}

\subsection{Проблема породження мас}

Стандартна модель описує три фундаментальні взаємодії — електромагнітну, слабку та сильну — як калібрувальні теорії з групою симетрії $\SU(3)_C \times \SU(2)_L \times \UU(1)_Y$. Принцип калібрувальної інваріантності забороняє явні масові члени для векторних бозонів і хіральних ферміонів, адже такі члени порушують симетрію. Водночас експеримент фіксує масивні бозони $W$ і $Z$ ($m_W = 80{.}4$ ГеВ, $m_Z = 91{.}2$ ГеВ) та масивні ферміони з діапазоном шість порядків — від електрона ($m_e = 0{.}511$ МеВ) до топ-кварка ($m_t = 173$ ГеВ).

Розв’язок, запропонований Браутом, Енглером, Гіггсом, Гуральником, Гейґеном і Кібблом у 1964, — це \emph{спонтанне порушення симетрії}: скалярне поле (поле Гіггса) має потенціал із виродженими мінімумами, і поле «обирає» один мінімум, порушуючи симетрію $\SU(2)_L \times \UU(1)_Y \to \UU(1)_{\text{EM}}$. Калібрувальні бозони, що взаємодіють із полем Гіггса, набувають мас, поглинаючи поздовжні моди, а ферміони — через Юкава-зв’язки. Відкриття бозона Гіггса з масою 125 ГеВ (ATLAS, CMS, 2012) експериментально підтвердило цей механізм.

Попри успіх, механізм Гіггса породжує запитання. Чому потенціал має вигляд $V(\phi) = -\mu^2|\phi|^2 + \lambda|\phi|^4$? Чому $\mu^2 < 0$ (неправильний знак для стабільності), що вимагає члена четвертого порядку? Що визначає вакуумне середнє $v = 246$ ГеВ? Чому ферміони демонструють ієрархію мас із відношенням $m_t/m_e \sim 3\times 10^5$? Стандартна модель не дає відповідей; це вхідні параметри, а не передбачення.

\subsection{Геометрична алгебра і бівекторний субстрат}

Геометрична алгебра (Кліффорда–Гестенеса) — координатно-вільна мова, у якій вектори, бівектори та елементи вищих градів живуть в єдиній алгебраїчній структурі. Обертання представляються роторами $\Rotor = \Exp(\Biv)$, де $\Biv$ — бівектор, що задає орієнтацію та фазу. Гестенес показав, що рівняння Дірака можна сформулювати суто в термінах GA, розкриваючи спінор як геометричний об’єкт.

У попередніх роботах ми показували, що класична механіка, електродинаміка, квантова кінематика і гравітація постають із \emph{гіпотези роторного поля}: фізичний простір має фундаментальне бівекторне поле $\Biv(x,t)$, а всі спостережувані структури виникають з його динаміки. Тут ми поширюємо програму на електрослабкий сектор.

\subsection{Центральний тезис і план}

Ми стверджуємо, що електрослабка калібрувальна структура і механізм Гіггса \emph{емергентні} з динаміки бівекторної когерентності. Зокрема:

\begin{center}
\textit{Симетрія $\SU(2)_L \times \UU(1)_Y$ виникає з природного розкладу \\
шестивимірного простору бівекторів у просторі Мінковського. \\
Спонтанне порушення відповідає фазовій когерентності ротора, \\
а маси калібрувальних бозонів — жорсткості поперечних бівекторних мод.}
\end{center}

Далі: у розд.~\ref{sec:bivector-decomposition} показано розклад бівекторного простору на фактори $\SU(2)$ та $\UU(1)$. У розд.~\ref{sec:coherence-vev} ми виводимо спонтанне порушення з роторної когерентності та визначаємо $v$ через жорсткість ротора. Розд.~\ref{sec:gauge-masses} присвячено масам $W$, $Z$ та фотона з бівекторної динаміки. Розд.~\ref{sec:fermion-masses} пояснює ієрархію мас ферміонів. У розд.~\ref{sec:predictions} — перевірні передбачення. Розд.~\ref{sec:discussion} — теоретичні наслідки й відкриті питання. Висновки — у розд.~\ref{sec:conclusion}.

\vspace{1em}

\section{Бівекторний розклад і калібрувальна структура}\label{sec:bivector-decomposition}

\subsection{Шестивимірний простір бівекторів}

У просторі Мінковського з сигнатурою $(+,-,-,-)$ та базисом $\{\gamma_\mu\}$, $\mu=0,1,2,3$, загальний бівектор має шість незалежних компонент:
\begin{equation}
  \Biv \;=\; \sum_{\mu<\nu} B^{\mu\nu}\, \gamma_\mu \wedge \gamma_\nu
  \;=\; B^{01}\gamma_0\wedge\gamma_1 + B^{02}\gamma_0\wedge\gamma_2 + B^{03}\gamma_0\wedge\gamma_3
  + B^{12}\gamma_1\wedge\gamma_2 + B^{13}\gamma_1\wedge\gamma_3 + B^{23}\gamma_2\wedge\gamma_3.
\end{equation}

Ці шість компонентів природно діляться на:
\begin{itemize}[leftmargin=*,itemsep=3pt]
  \item \textbf{Часо-просторові (електричного типу):} $\Biv_E = \sum_{i=1}^3 B^{0i}\gamma_0\wedge\gamma_i$ (3 компоненти).
  \item \textbf{Просторові (магнітного типу):} $\Biv_M = \sum_{i<j} B^{ij}\gamma_i\wedge\gamma_j$ (3 компоненти).
\end{itemize}

В електродинаміці $\Biv_E$ відповідає електричному полю, а $\Biv_M$ — магнітному. Тензор Фарадея $F = \mathbf{E} + I\mathbf{B}$ об’єднує обидва через псевдоскаляр $I = \gamma_0\gamma_1\gamma_2\gamma_3$.

\subsection{Просторові бівектори як генератори $\SU(2)$}

Просторові бівектори $\{\gamma_1\wedge\gamma_2,\, \gamma_2\wedge\gamma_3,\, \gamma_3\wedge\gamma_1\}$ задовольняють комутаційні співвідношення $\mathfrak{su}(2)$. Визначимо
\begin{equation}
  \tau^1 := \gamma_2\wedge\gamma_3, \qquad
  \tau^2 := \gamma_3\wedge\gamma_1, \qquad
  \tau^3 := \gamma_1\wedge\gamma_2.
\end{equation}

Геометричний добуток дає
\begin{equation}
  \tau^i \tau^j = -\delta^{ij} + \epsilon^{ijk}\tau^k,
\end{equation}
а комутатор
\begin{equation}
  [\tau^i, \tau^j] = 2\epsilon^{ijk}\tau^k,
\end{equation}
що після масштабування $\tau^i \to \frac{i}{2}\sigma^i$ відтворює стандартні відношення
\begin{equation}
  [\sigma^i, \sigma^j] = 2i\epsilon^{ijk}\sigma^k.
\end{equation}

Отже три просторові бівектори природно утворюють \textbf{алгебру $\SU(2)$} слабкого ізоспіну.

\subsection{Часо-просторове змішування як гіперзаряд $\UU(1)$}

Бівектори $\{\gamma_0\wedge\gamma_i\}$ попарно комутують (оскільки $(\gamma_0\wedge\gamma_i)(\gamma_0\wedge\gamma_j)=\gamma_0\gamma_i\gamma_0\gamma_j=-\gamma_i\gamma_j$ симетричний за $i,j$). Визначимо генератор гіперзаряду як лінійну комбінацію:
\begin{equation}
  Y := \alpha_1(\gamma_0\wedge\gamma_1) + \alpha_2(\gamma_0\wedge\gamma_2) + \alpha_3(\gamma_0\wedge\gamma_3),
\end{equation}
де $\alpha_i$ — коефіцієнти зв’язку. Оскільки $Y$ комутує сам із собою і з просторовими обертаннями, він генерує симетрію $\UU(1)$ — \textbf{гіперзаряд} $\UU(1)_Y$.

Оператор електричного заряду $Q$:
\begin{equation}
  Q = T^3 + \frac{Y}{2},
\end{equation}
де $T^3 = \frac{1}{2}\sigma^3 = \frac{i}{2}\tau^3$ — третя компонента ізоспіну.

\subsection{Гейджові перетворення ротором}

Загальний ротор електрослабкого сектора:
\begin{equation}
  \Rotor_{\text{EW}}(x,t) \;=\; \Exp\!\big(\theta^a(x,t)\,\tau^a + \chi(x,t)\,Y\big),
  \qquad a=1,2,3,
\end{equation}
де $\theta^a(x,t)$ — три кути для $\SU(2)$, а $\chi(x,t)$ — фаза $\UU(1)$.

Калібрувальні (гейджові) перетворення — локальні зсуви ротора:
\begin{equation}
  \Rotor_{\text{EW}}(x,t) \;\to\; \Rotor_{\text{gauge}}(x,t)\, \Rotor_{\text{EW}}(x,t).
\end{equation}

Ковариантна похідна на полі $\psi$ у поданні $R$:
\begin{equation}
  \D_\mu \psi \;=\; \partial_\mu \psi + \frac{i}{2}\,A_\mu\,\psi,
\end{equation}
де калібрувальний бівекторний зв’язок
\begin{equation}
  A_\mu \;=\; W_\mu^a\,\tau^a + B_\mu\,Y.
\end{equation}

Тензори напружень:
\begin{align}
  W_{\mu\nu}^a &\;=\; \partial_\mu W_\nu^a - \partial_\nu W_\mu^a + g\epsilon^{abc}W_\mu^b W_\nu^c, \\
  B_{\mu\nu} &\;=\; \partial_\mu B_\nu - \partial_\nu B_\mu,
\end{align}
де $g$ — зв’язок $\SU(2)$, а $g'$ — зв’язок $\UU(1)$.

\begin{proposition}[Природна гейджова структура з бівекторного простору]
Шестивимірний простір бівекторів у просторі Мінковського допускає канонічний розклад $\Biv = \Biv_{\SU(2)} + \Biv_{\UU(1)}$, де:
\begin{itemize}
  \item $\Biv_{\SU(2)} = \theta^a \tau^a$ натягує просторові бівектори (3-вим),
  \item $\Biv_{\UU(1)} = \chi Y$ лежить у часо-просторовому секторі (ефективно 1-вим після вибору напряму).
\end{itemize}
Цей розклад відтворює групу Стандартної моделі $\SU(2)_L \times \UU(1)_Y$ без постулювання симетрій.
\end{proposition}

\begin{remark}
Фактор-структура $\SU(2)_L \times \UU(1)_Y$ геометрично зумовлена сигнатурою простору-часу. В інших розмірностях чи сигнатурах бівекторна алгебра даватиме інші групи, що надає геометричну класифікацію можливих електрослабких теорій.
\end{remark}

\vspace{1em}

\section{Фазова когерентність ротора і вакуумне середнє}\label{sec:coherence-vev}

\subsection{Параметр когерентності ротора}

Визначимо \emph{функціонал когерентності ротора} як ансамблеве середнє
\begin{equation}
  \mathcal{C} \;:=\; \ev{\Rotor(x,t)}_{\text{вакуум}}.
\end{equation}

У симетричній фазі з випадковими орієнтаціями $\ev{\Rotor}=0$. У фазово-узгодженому стані з привілейованою орієнтацією $\Biv_0$ маємо
\begin{equation}
  \ev{\Rotor} \;=\; \Exp(\Biv_0) \;\neq\; 0.
\end{equation}

Це спонтанне вирівнювання фаз — геометричне походження порушення симетрії.

\subsection{Ефективний потенціал із динаміки ротора}

Дія роторного поля в електрослабкому секторі:
\begin{equation}
  S_{\text{rotor}} \;=\; \int \mathrm{d}^4x \left[\frac{1}{2}(\D_\mu\Biv)^2 - V_{\text{eff}}(\Biv)\right],
\end{equation}
де ефективний потенціал зумовлений самодіями ротора:
\begin{equation}
  V_{\text{eff}}(\Biv) \;=\; \frac{\lambda}{4}\left(\scal{\Biv^2} - M_\ast^2\right)^2.
\end{equation}

Тут $M_\ast$ — параметр жорсткості ротора (розмірність маси), $\lambda$ — безрозмірний зв’язок.

Мінімум досягається за
\begin{equation}
  \scal{\Biv_0^2} \;=\; M_\ast^2,
\end{equation}
тобто ненульова норма бівектора. Множина мінімумів утворює $S^5$ у бівекторному просторі.

\subsection{Виведення VEV Гіггса}

Обираємо конфігурацію, що порушує $\SU(2)_L \times \UU(1)_Y$ до $\UU(1)_{\text{EM}}$. Подвійка Гіггса у нашій рамці відповідає бівектору в секторі слабкого ізоспіну:
\begin{equation}
  \Biv_{\text{Higgs}} \;=\; \frac{1}{\sqrt{2}}\begin{pmatrix} 0 \\ v \end{pmatrix} \tau^3
  \;=\; \frac{v}{\sqrt{2}}\,\tau^3.
\end{equation}

З умови мінімуму
\begin{equation}
  \scal{\Biv_{\text{Higgs}}^2}
  \;=\; \frac{v^2}{2}\scal{(\tau^3)^2}
  \;=\; -\frac{v^2}{2}
  \;=\; -M_\ast^2
  \;\Rightarrow\;
  v = \sqrt{2}\,M_\ast.
\end{equation}

Експериментально $v \approx 246$ ГеВ, тож
\begin{equation}
  \boxed{M_\ast \;=\; \frac{v}{\sqrt{2}} \;\approx\; 174\,\text{ГеВ}.}
\end{equation}

\subsection{Золстонівські моди та поздовжні калібрувальні бозони}

Флуктуації біля вакууму $\Biv_0$:
\begin{equation}
  \Biv(x,t) \;=\; \Biv_0 + h(x,t)\,\hat{\Biv}_0 + \pi^a(x,t)\,\tau^a,
\end{equation}
де $h$ — радіальна (Гіггсова) мода, $\pi^a$ — три золстонівські моди (зламані генератори).

В унітарній калібровці $\pi^a$ поглинаються як поздовжні поляризації $W$ і $Z$. Фізичний Гіггс — радiальне збудження з масою
\begin{equation}
  m_H^2 \;=\; 2\lambda M_\ast^2 \;=\; \lambda v^2.
\end{equation}
За $m_H \approx 125$ ГеВ і $v \approx 246$ ГеВ маємо $\lambda \approx 0{.}26$.

\vspace{1em}

\section{Маси калібрувальних бозонів із бівекторної динаміки}\label{sec:gauge-masses}

\subsection{Поперечні моди і маса $W$}

Бозони $W^\pm$ відповідають поперечним осциляціям $\tau^1 \pm i\tau^2$. Після порушення симетрії при $\ev{\Biv} = \frac{v}{\sqrt{2}}\tau^3$ ковариантна похідна на $\Biv_{\text{Higgs}}$ породжує масовий член:
\begin{equation}
  \Lag_{\text{mass}}^{W} \;=\; \frac{g^2 v^2}{4}\,W_\mu^+ W^{-\mu},
\end{equation}
звідки
\begin{equation}
  \boxed{m_W \;=\; \frac{gv}{2}.}
\end{equation}
Чисельно $m_W \approx 80{.}4$ ГеВ.

\subsection{Змішані моди і маса $Z$}

Нейтральні $W_\mu^3$ і $B_\mu$ змішуються:
\begin{equation}
  \begin{pmatrix} A_\mu \\ Z_\mu \end{pmatrix}
  =
  \begin{pmatrix}
  \cos\theta_W & \sin\theta_W \\
  -\sin\theta_W & \cos\theta_W
  \end{pmatrix}
  \begin{pmatrix} B_\mu \\ W_\mu^3 \end{pmatrix},
\end{equation}
з $\tan\theta_W = g'/g$. Масовий член дає
\begin{equation}
  \boxed{m_Z \;=\; \frac{v}{2}\sqrt{g^2 + g'^2} \;=\; \frac{m_W}{\cos\theta_W}.}
\end{equation}
Для $\sin^2\theta_W \approx 0{.}231$ маємо $m_Z \approx 91{.}2$ ГеВ.

\subsection{Безмасовість фотона та збереження $\UU(1)_{\text{EM}}$}

Фотон $A_\mu$ лишається безмасовим, бо $\UU(1)_{\text{EM}}$, породжена $Q=T^3+Y/2$, не порушується. Нижня компонента Гіггса з $T^3=-1/2$, $Y=+1/2$ має $Q=0$, тож вакуум зберігає електричний заряд.

\vspace{1em}

\section{Маси ферміонів через Юкава-зв’язки}\label{sec:fermion-masses}

\subsection{Ротор-ферміонна взаємодія}

Ферміони описуються спінорами $\psi$ з трансформацією
\begin{equation}
  \psi'(x) \;=\; \Rotor_{\text{EW}}(x)\,\psi(x).
\end{equation}
Юкава-взаємодія:
\begin{equation}
  \Lag_{\text{Yukawa}} \;=\; -y_f\, \bar{\psi}_L\, \Biv_{\text{Higgs}}\, \psi_R + \text{h.c.},
\end{equation}
після порушення симетрії дає
\begin{equation}
  \boxed{m_f \;=\; \frac{y_f v}{\sqrt{2}}.}
\end{equation}

\subsection{Ієрархія мас із чисел намотування ротора}

У нашій гіпотезі види ферміонів відповідають різним топологічним секторам ротора з числом намотування $n_w$. Юкава-зв’язок зростає для малих $n_w$ й експоненційно пригнічується для великих:
\begin{equation}
  y_f \;\propto\; \exp\!\left(-\frac{S_{\text{inst}}}{n_w}\right).
\end{equation}
Це геометрично пояснює $y_t\sim 1$ і $y_e\sim 10^{-6}$.

\begin{proposition}[Ієрархія Юкави з топології]
Ієрархія мас ферміонів виникає з топології намотування роторного поля: малі $n_w$ $\Rightarrow$ великі маси, великі $n_w$ $\Rightarrow$ експоненційно малі маси.
\end{proposition}

\subsection{Змішування смаків і матриця CKM}

Якщо матриці Юкави $Y_{ij}$ не діагоналізуються одночасно, після розбиття з’являється змішування $V_{\text{CKM}}=U_L^\dagger U_R$, що геометрично інтерпретується як відносні орієнтації намотувань для різних смаків.

\vspace{1em}

\section{Спостережні передбачення}\label{sec:predictions}

\subsection{Народження Гіггса на колайдерах}

Домінантний канал $gg\to H$ залежить від $y_t$:
\begin{equation}
  \sigma(gg \to H) \propto y_t^2 = \frac{2m_t^2}{v^2}.
\end{equation}
Квантові поправки від \emph{роторної кривизни} можуть дати
\begin{equation}
  \delta_{\text{rotor}} \sim \frac{M_\ast^2}{\Lambda_{\text{rotor}}^2}\,\ev{\mathcal{K}^2},
\end{equation}
що при $\Lambda_{\text{rotor}}\sim 1$ ТеВ веде до $\sim 2\%$ ефекту — потенційно видимого на HL-LHC.

\subsection{Точні електрослабкі параметри}

Облічні параметри $(S,T,U)$ отримують однопетльові внески роторної когерентності. Для $\Lambda_{\text{rotor}}>1$ ТеВ вони узгоджуються з поточними межами $|S|,|T|\lesssim 0{.}1$. Майбутні лептонні колайдери з точністю $\sim 0{.}01$ можуть побачити ефекти, якщо $\Lambda_{\text{rotor}}\lesssim 2$ ТеВ.

\subsection{Потрійні калібрувальні зв’язки}

Самодії ротора індукують аномальні TGC ($W^+W^-\gamma$, $W^+W^-Z$) з масштабом
\begin{equation}
  \Delta g_1^Z,\ \lambda_\gamma \;\sim\; \frac{g^2 M_\ast^2}{\Lambda_{\text{rotor}}^4},
\end{equation}
що при $\Lambda_{\text{rotor}}=1$ ТеВ дає $\sim 10^{-5}$ — нижче поточних меж, але в зоні досяжності HL-LHC.

\subsection{Самозв’язок Гіггса}

Із потенціалу
\begin{equation}
  V_{\text{eff}}(\Biv) = \frac{\lambda}{4}(\scal{\Biv^2}-M_\ast^2)^2
  \;\Rightarrow\;
  \lambda_{HHH} = \frac{3m_H^2}{v} \approx 191~\text{ГеВ},
\end{equation}
узгоджено зі Стандартною моделлю; подвійне народження Гіггса на HL-LHC перевірятиме це з $\sim 50\%$ точністю.

\vspace{1em}

\section{Обговорення та теоретичні наслідки}\label{sec:discussion}

\subsection{Концептуальна єдність калібрувального та Гіггсового секторів}

Гіпотеза роторного поля знімає штучний поділ між калібрувальними полями та скаляром Гіггса: обидва — прояви єдиного бівекторного поля.

\subsection{Порівняння з композитним Гіггсом}

Подібності з техніколором/«малим Гіггсом», але відмінність у \emph{геометричному} походженні симетрій і топологічному поясненні ієрархій Юкави без нових важких ферміонів.

\subsection{Зв’язок із сильною CP-проблемою}

Числа намотування ротора наштовхують на механізм, подібний до Печчі—Квінна, який геометрично зменшує $\theta_{\text{QCD}}\to 0$; потребує розширення на колірний сектор $\SU(3)_C$.

\subsection{Наслідки для великого об’єднання}

У вищих розмірностях бівекторні простори GA природно вміщують GUT-групи (напр., у $\mathcal{G}(1,9)$ — $45$ бівекторів $\sim \SO(10)$), що натякає на геометричне походження об’єднання.

\subsection{Відкриті питання}

\textbf{Маси нейтрино і майоранівська природа}, \textbf{темна матерія як приховані бівектори}, \textbf{динаміка електрослабкого фазового переходу} (можливі наногерцові гравітаційні хвилі), \textbf{квантові поправки і перенормування} в роторній рамці.

\vspace{1em}

\section{Висновки}\label{sec:conclusion}

Ми показали, що електрослабкий сектор — калібрувальна структура, спонтанне порушення, механізм Гіггса, маси $W/Z$ і Юкава-куполінги — емергує з динаміки фундаментального бівекторного поля у геометричній алгебрі. Головні результати:

\begin{enumerate}[leftmargin=*,itemsep=3pt]
  \item Природний розклад шестивимірного бівекторного простору дає $\SU(2)_L \times \UU(1)_Y$ без постулатів.
  \item Спонтанне порушення — фазова когерентність ротора з $v=\sqrt{2}M_\ast\approx 246$ ГеВ ($M_\ast\approx 174$ ГеВ).
  \item Маси:
  \[
  m_W = \frac{gv}{2} \approx 80{.}4~\text{ГеВ},\qquad
  m_Z = \frac{m_W}{\cos\theta_W} \approx 91{.}2~\text{ГеВ}.
  \]
  \item Фотон безмасовий завдяки збереженню $\UU(1)_{\text{EM}}$.
  \item Маси ферміонів $m_f = y_f v/\sqrt{2}$; ієрархії — із топології намотувань.
  \item $m_H \approx 125$ ГеВ $\Rightarrow \lambda \approx 0{.}26$.
  \item Перевірні передбачення: $\sim2\%$ зміни у $gg\to H$ при $\Lambda_{\text{rotor}}\sim 1$ ТеВ; поправки до $(S,T)$ на рівні $0{.}01$; aTGC $\sim 10^{-5}$.
\end{enumerate}

Це знімає поділ між гейджовим і скалярним секторами: обидва — грані єдиного бівекторного поля; породження маси — геометричний наслідок жорсткості бівекторних мод.

\vspace{1em}

\section*{Подяки}

Я глибоко вдячний Девіду Гестенесу за розвиток геометричної алгебри як мови фізики. Роботи Кріса Доранa та Ентоні Лейзнбі з гравітації як гейдж-теорії були вирішальними. Дякую колабораціям ATLAS і CMS за точні вимірювання електрослабких параметрів. Робота виконана самостійно, без зовнішнього фінансування.

\vspace{1em}

% ---------- Література (вбудована, дружня до arXiv) ----------
\begin{thebibliography}{99}\setlength{\itemsep}{3pt}

\bibitem{Higgs1964}
P.~W.~Higgs, \emph{Broken Symmetries and the Masses of Gauge Bosons}, Phys.\ Rev.\ Lett.\ \textbf{13} (1964) 508--509.

\bibitem{Englert1964}
F.~Englert, R.~Brout, \emph{Broken Symmetry and the Mass of Gauge Vector Mesons}, Phys.\ Rev.\ Lett.\ \textbf{13} (1964) 321--323.

\bibitem{Guralnik1964}
G.~S.~Guralnik, C.~R.~Hagen, T.~W.~B.~Kibble, \emph{Global Conservation Laws and Massless Particles}, Phys.\ Rev.\ Lett.\ \textbf{13} (1964) 585--587.

\bibitem{Glashow1961}
S.~L.~Glashow, \emph{Partial-Symmetries of Weak Interactions}, Nucl.\ Phys.\ \textbf{22} (1961) 579--588.

\bibitem{Weinberg1967}
S.~Weinberg, \emph{A Model of Leptons}, Phys.\ Rev.\ Lett.\ \textbf{19} (1967) 1264--1266.

\bibitem{Salam1968}
A.~Salam, \emph{Weak and Electromagnetic Interactions}, in N.~Svartholm (ed.), \textit{Elementary Particle Theory}, Almqvist \& Wiksell, Stockholm, 1968, pp.~367--377.

\bibitem{ATLAS2012}
G.~Aad et al.\ (ATLAS Collaboration), \emph{Observation of a New Particle in the Search for the Standard Model Higgs Boson with the ATLAS Detector at the LHC}, Phys.\ Lett.\ B \textbf{716} (2012) 1--29. arXiv:1207.7214.

\bibitem{CMS2012}
S.~Chatrchyan et al.\ (CMS Collaboration), \emph{Observation of a New Boson at a Mass of 125 GeV with the CMS Experiment at the LHC}, Phys.\ Lett.\ B \textbf{716} (2012) 30--61. arXiv:1207.7235.

\bibitem{PDG2022}
R.~L.~Workman et al.\ (Particle Data Group), \emph{Review of Particle Physics}, Prog.\ Theor.\ Exp.\ Phys.\ \textbf{2022} (2022) 083C01.

\bibitem{Hestenes1966}
D.~Hestenes, \emph{Space-Time Algebra}, Gordon and Breach, New York, 1966.

\bibitem{Hestenes1984}
D.~Hestenes, G.~Sobczyk, \emph{Clifford Algebra to Geometric Calculus: A Unified Language for Mathematics and Physics}, Reidel, Dordrecht, 1984.

\bibitem{DoranLasenby2003}
C.~Doran, A.~Lasenby, \emph{Geometric Algebra for Physicists}, Cambridge University Press, 2003.

\bibitem{Lasenby1998}
A.~Lasenby, C.~Doran, S.~Gull, \emph{Gravity, Gauge Theories and Geometric Algebra}, Phil.\ Trans.\ R.\ Soc.\ A \textbf{356} (1998) 487--582.

\bibitem{HestenesEM2003}
D.~Hestenes, \emph{Oersted Medal Lecture 2002: Reforming the Mathematical Language of Physics}, Am.\ J.\ Phys.\ \textbf{71} (2003) 104--121.

\bibitem{Clifford1878}
W.~K.~Clifford, \emph{Applications of Grassmann's Extensive Algebra}, Am.\ J.\ Math.\ \textbf{1} (1878) 350--358.

\bibitem{Peskin1995}
M.~E.~Peskin, D.~V.~Schroeder, \emph{An Introduction to Quantum Field Theory}, Addison-Wesley, Reading, 1995.

\bibitem{Cheng1984}
T.-P.~Cheng, L.-F.~Li, \emph{Gauge Theory of Elementary Particle Physics}, Oxford University Press, Oxford, 1984.

\bibitem{Quigg2013}
C.~Quigg, \emph{Gauge Theories of the Strong, Weak, and Electromagnetic Interactions}, 2nd ed., Princeton University Press, 2013.

\bibitem{tHooft1972}
G.~'t~Hooft, M.~Veltman, \emph{Regularization and Renormalization of Gauge Fields}, Nucl.\ Phys.\ B \textbf{44} (1972) 189--213.

\bibitem{Lee1972}
B.~W.~Lee, J.~Zinn-Justin, \emph{Spontaneously Broken Gauge Symmetries. IV. General Gauge Formulation}, Phys.\ Rev.\ D \textbf{7} (1973) 1049--1056.

\bibitem{Gildener1976}
E.~Gildener, S.~Weinberg, \emph{Symmetry Breaking and Scalar Bosons}, Phys.\ Rev.\ D \textbf{13} (1976) 3333--3341.

\bibitem{Coleman1973}
S.~Coleman, E.~Weinberg, \emph{Radiative Corrections as the Origin of Spontaneous Symmetry Breaking}, Phys.\ Rev.\ D \textbf{7} (1973) 1888--1910.

\bibitem{Susskind1979}
L.~Susskind, \emph{Dynamics of Spontaneous Symmetry Breaking in the Weinberg-Salam Theory}, Phys.\ Rev.\ D \textbf{20} (1979) 2619--2625.

\bibitem{Kaplan1984}
D.~B.~Kaplan, H.~Georgi, \emph{$\SU(2) \times \UU(1)$ Breaking by Vacuum Misalignment}, Phys.\ Lett.\ B \textbf{136} (1984) 183--186.

\bibitem{Georgi1984}
H.~Georgi, D.~B.~Kaplan, P.~Galison, \emph{Calculation of the Composite Higgs Mass}, Phys.\ Lett.\ B \textbf{143} (1984) 152--154.

\bibitem{Arkani2002}
N.~Arkani-Hamed, A.~G.~Cohen, H.~Georgi, \emph{Electroweak Symmetry Breaking from Dimensional Deconstruction}, Phys.\ Lett.\ B \textbf{513} (2001) 232--240. arXiv:hep-ph/0105239.

\bibitem{Schmaltz2005}
M.~Schmaltz, D.~Tucker-Smith, \emph{Little Higgs Review}, Ann.\ Rev.\ Nucl.\ Part.\ Sci.\ \textbf{55} (2005) 229--270. arXiv:hep-ph/0502182.

\bibitem{Peccei1977}
R.~D.~Peccei, H.~R.~Quinn, \emph{CP Conservation in the Presence of Pseudoparticles}, Phys.\ Rev.\ Lett.\ \textbf{38} (1977) 1440--1443.

\bibitem{NANOGrav2023}
G.~Agazie et al.\ (NANOGrav Collaboration), \emph{The NANOGrav 15 yr Data Set: Evidence for a Gravitational-wave Background}, Astrophys.\ J.\ Lett.\ \textbf{951} (2023) L8. arXiv:2306.16213.

\bibitem{Shaposhnikov1987}
M.~E.~Shaposhnikov, \emph{Possible Appearance of the Baryon Asymmetry of the Universe in an Electroweak Theory}, JETP Lett.\ \textbf{44} (1986) 465--468.

\bibitem{Kajantie1996}
K.~Kajantie, M.~Laine, K.~Rummukainen, M.~Shaposhnikov, \emph{Is There a Hot Electroweak Phase Transition at $m_H \gtrsim m_W$?}, Phys.\ Rev.\ Lett.\ \textbf{77} (1996) 2887--2890. arXiv:hep-ph/9605288.

\bibitem{Csaki2004}
C.~Cs\'aki, C.~Grojean, H.~Murayama, L.~Pilo, J.~Terning, \emph{Gauge Theories on an Interval: Unitarity without a Higgs Boson}, Phys.\ Rev.\ D \textbf{69} (2004) 055006. arXiv:hep-ph/0305237.

\end{thebibliography}

% =============================================================================
\end{document}
% =============================================================================
