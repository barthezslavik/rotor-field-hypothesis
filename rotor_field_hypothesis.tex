% =============================================================================
% The Rotor Field Hypothesis: Unifying Matter, Information, and Motion
% arXiv-ready LaTeX template (single-file, no external .bib)
% =============================================================================
\documentclass[11pt,a4paper]{article}

% ---------- Packages ----------
\usepackage[utf8]{inputenc}
\usepackage[T1]{fontenc}
\usepackage{lmodern}
\usepackage[a4paper,margin=1in]{geometry}
\usepackage{microtype}
\usepackage{amsmath,amssymb,amsthm,mathtools}
\usepackage{physics}
\usepackage{graphicx}
\usepackage{xcolor}
\usepackage{bm}
\usepackage{booktabs}
\usepackage{enumitem}
\usepackage{hyperref}
\hypersetup{
  colorlinks=true,
  linkcolor=blue!50!black,
  citecolor=blue!50!black,
  urlcolor=blue!60!black,
  pdfauthor={Viacheslav Loginov},
  pdftitle={The Rotor Field Hypothesis: Unifying Matter, Information, and Motion}
}
\usepackage{authblk}
\usepackage{caption}

% ---------- Macros: Geometric Algebra (GA) ----------
% Basis vectors and multivector operations
\newcommand{\e}{\mathbf{e}}
\newcommand{\E}{\mathbb{E}}
\newcommand{\R}{\mathbb{R}}
\newcommand{\grade}[2]{\left\langle #1 \right\rangle_{#2}}
\newcommand{\scal}[1]{\grade{#1}{0}}
\newcommand{\vecp}[1]{\grade{#1}{1}}
\newcommand{\biv}[1]{\grade{#1}{2}}
\newcommand{\triv}[1]{\grade{#1}{3}}
\newcommand{\rev}[1]{\widetilde{#1}}           % reversion
\newcommand{\dual}[1]{#1^\ast}                 % dual
\newcommand{\geop}{\mathbin{\!\!\wedge\!\!}}   % outer/wedge product
\newcommand{\inner}{\mathbin{\!\!\cdot\!\!}}   % inner product
\newcommand{\ad}{\operatorname{ad}}
\newcommand{\Exp}{\operatorname{Exp}}

% Rotors and bivectors
\newcommand{\Rotor}{\mathcal{R}}
\newcommand{\Biv}{\mathcal{B}}
\newcommand{\Field}{\mathcal{F}}

% Differential operators
\newcommand{\D}{\nabla}                        % GA vector derivative
\newcommand{\dt}{\,\mathrm{d}t}
\newcommand{\dx}{\,\mathrm{d}x}

% ---------- Theorem-like environments ----------
\theoremstyle{definition}
\newtheorem{definition}{Definition}
\theoremstyle{plain}
\newtheorem{theorem}{Theorem}
\newtheorem{lemma}{Lemma}
\theoremstyle{remark}
\newtheorem{remark}{Remark}

% ---------- Title / Authors ----------
\title{\textbf{The Rotor Field Hypothesis: Unifying Matter, Information, and Motion}}
\author[1]{Viacheslav Loginov}
\affil[1]{Kyiv, Ukraine\\ \texttt{barthez.slavik@gmail.com}}
\date{\small Version 1.0 \quad|\quad 10 October 2025}

% =============================================================================
\begin{document}
\maketitle

\begin{abstract}
\noindent
Modern physics treats space, matter, and information as distinct entities, each governed by separate mathematical frameworks. Yet the ubiquity of rotation---from electron spin and electromagnetic polarization to planetary orbits and cosmic structure---suggests a deeper unity. We propose that all observable structures emerge from the dynamics of a universal \emph{rotor field} defined in geometric algebra. The fundamental object is a spatially distributed rotor $\Rotor(x,t)=\Exp\!\big(\Biv(x,t)\big)$ whose bivector generator $\Biv$ controls local orientation, phase, and coherent coupling. We show how classical mechanics, electromagnetism, quantum spinor dynamics, thermodynamic irreversibility, and efficient information encoding appear as effective regimes of rotor coherence, transport, and symmetry breaking. The hypothesis yields falsifiable predictions: spectral sidebands in gravitational-wave events with precession; compression gains on cyclic signals surpassing conventional codecs; and characteristic scaling laws in rotor phase evolution. We provide a reproducible program for experimental validation across physics, signal processing, and machine learning domains.
\end{abstract}

\noindent\textbf{Keywords:} geometric algebra, rotor fields, unification, coherence, emergent dynamics, information compression

\vspace{1em}

\section{Introduction}

\subsection{The Problem of Disparate Frameworks}

Contemporary science describes nature through a collection of specialized theories: quantum mechanics for atomic phenomena, classical field theory for electromagnetism, statistical mechanics for thermodynamics, and information theory for data encoding. Each framework has proven extraordinarily successful within its domain. Yet this success comes at a conceptual cost: the multiplication of primitives---wavefunctions, field tensors, probability distributions, entropy measures---obscures potential deeper connections.

Consider three seemingly unrelated phenomena. First, an electron possesses intrinsic angular momentum (spin) described by Pauli matrices acting on two-component spinors. Second, electromagnetic waves exhibit circular polarization, with left- and right-handed states corresponding to bivector orientations in the field tensor $F_{\mu\nu}$. Third, a spinning top precesses under gravity, its angular momentum vector tracing a cone through space. Despite arising in quantum mechanics, electromagnetism, and classical mechanics respectively, all three phenomena share a common rotational character.

Might there exist a single mathematical structure from which these diverse manifestations emerge? If so, what are the minimal postulates from which such a unification follows?

\subsection{Geometric Algebra as a Universal Language}

Clifford's geometric algebra provides a coordinate-free framework wherein vectors, bivectors (oriented plane segments), and higher-grade elements inhabit a unified algebraic structure. The geometric product combines inner and outer products, while rotations are represented by \emph{rotors}---exponentials of bivectors. Hestenes demonstrated that the Dirac equation for relativistic electrons can be expressed entirely in geometric algebra, revealing the spinor as a geometric object rather than an abstract entity requiring auxiliary Hilbert spaces.

This suggests that quantum mechanics may be more geometrical than traditionally supposed. Moreover, the bivector naturally describes both electromagnetic fields (the Faraday tensor) and angular momentum, hinting at a common underlying substrate.

\subsection{The Central Hypothesis}

We propose the following principle:

\begin{center}
\textit{Physical space admits a fundamental bivector field $\Biv(x,t)$, \\
and all observable phenomena arise from the dynamics \\
of the associated rotor field $\Rotor(x,t)=\Exp\!\big(\tfrac{1}{2}\Biv(x,t)\big)$.}
\end{center}

From this single postulate, we shall demonstrate how:

\begin{enumerate}[leftmargin=*,itemsep=3pt]
  \item \textbf{Classical mechanics} emerges as the linearized limit of rotor flow, with \textbf{Newton's second law derived exactly} from bivector dynamics.
  \item \textbf{Electromagnetism} appears as a bivector field obeying rotor-induced transport equations.
  \item \textbf{Quantum kinematics} arise from the spinor sector of the rotor representation.
  \item \textbf{Thermodynamic irreversibility} follows from ensemble statistics over rotor phases.
  \item \textbf{Efficient information encoding} exploits rotor phase structure to compress signals with rotational symmetry.
\end{enumerate}

The remainder of this paper develops these connections systematically. Section~\ref{sec:hypothesis} presents the precise mathematical formulation. Section~\ref{sec:math} establishes the geometric foundations. Section~\ref{sec:emergent} derives known physical theories as emergent limits. Section~\ref{sec:compression} introduces rotor-based signal compression. Section~\ref{sec:predictions} states falsifiable experimental predictions. Section~\ref{sec:discussion} addresses philosophical implications and open questions. Section~\ref{sec:conclusion} offers concluding remarks.

\vspace{1em}

\section{The Rotor Field Hypothesis}\label{sec:hypothesis}

\subsection{Kinematic Postulates}

Let physical space(-time) admit a geometric algebra with orthonormal basis $\{\e_\mu\}$ and metric of signature appropriate to the regime under study. The primitive entity is a \emph{rotor field}
\begin{equation}
  \Rotor(x,t) \;=\; \Exp\!\big(\Biv(x,t)\big),
  \qquad \rev{\Rotor}\Rotor = 1,
  \label{eq:rotor}
\end{equation}
where $\Biv(x,t)$ is a (local) bivector generating orientation and phase. Any multivector observable $A$ is rotated by
\begin{equation}
  A'(x,t) \;=\; \Rotor(x,t)\, A(x,t)\, \rev{\Rotor}(x,t).
\end{equation}

This transformation encodes active rotations: vectors rotate in their planes, bivectors undergo double rotations, and scalars remain invariant. The rotor field thus acts as a position- and time-dependent rotation operator on the entire algebra.

\begin{definition}[Rotor density, phase, and curvature]
Define (i) the \emph{rotor density} $\rho_R(x,t):=\scal{\Biv^2}^{1/2}$, (ii) the \emph{phase} $\phi$ by $\Biv=\phi\,\hat{\Biv}$ with $\hat{\Biv}^2=-1$, and (iii) the \emph{rotor curvature} $\mathcal{K}:=\biv{\D \wedge \Biv}$.
\end{definition}

The rotor density measures the local strength of rotation. The phase $\phi$ generalizes the quantum mechanical phase to arbitrary dimensions. The curvature $\mathcal{K}$ quantifies spatial inhomogeneity, analogous to the field strength in gauge theory.

\subsection{Dynamical Postulates}

We postulate a local balance law for the bivector generator with transport, coupling, and source terms:
\begin{equation}
  \D \Biv \;+\; \lambda\, \biv{\Biv\,\D} \;=\; \mathcal{J} \;-\; \Gamma(\Biv),
  \label{eq:rotor-dynamics}
\end{equation}
where:
\begin{itemize}[leftmargin=*,itemsep=2pt]
  \item $\D = \sum_\mu \e_\mu \partial_\mu$ is the geometric derivative,
  \item $\lambda$ controls nonlinear advection (rotor self-interaction),
  \item $\mathcal{J}$ aggregates external couplings (matter currents, boundary driving),
  \item $\Gamma(\Biv)$ collects dissipative and decohering channels.
\end{itemize}

The induced evolution of any rotated observable obeys
\begin{equation}
  \partial_t A' \;=\; \left(\partial_t \Rotor\right) A \rev{\Rotor} + \Rotor \left(\partial_t A\right)\rev{\Rotor} + \Rotor A \left(\partial_t\rev{\Rotor}\right),
\end{equation}
with $\partial_t \Rotor = \frac{1}{2}\Omega\,\Rotor$ for bivector rate $\Omega:=2(\partial_t \Rotor)\rev{\Rotor}$.

\begin{remark}
Equation~\eqref{eq:rotor-dynamics} is a unifying template. Specific identifications of $\Biv$, $\mathcal{J}$, and $\Gamma$ yield effective theories, as we shall demonstrate in Section~\ref{sec:emergent}.
\end{remark}

\vspace{1em}

\section{Mathematical Foundations}\label{sec:math}

\subsection{The Geometric Product and Grade Decomposition}

The geometric algebra $\mathcal{G}(\R^n)$ is generated by the geometric product of vectors, which is associative but not commutative. For orthonormal basis vectors $\{\e_\mu\}$, we have
\begin{equation}
\e_\mu \e_\nu = \e_\mu \cdot \e_\nu + \e_\mu \wedge \e_\nu = \delta_{\mu\nu} + \e_\mu \wedge \e_\nu.
\end{equation}

A general multivector decomposes into grades:
\begin{equation}
M = \scal{M} + \vecp{M} + \biv{M} + \triv{M} + \cdots,
\end{equation}
representing scalars, vectors, bivectors, trivectors, etc. The bivector $\Biv = \sum_{\mu<\nu} B^{\mu\nu} \e_\mu \wedge \e_\nu$ spans the space of oriented plane segments and serves as the generator of rotations.

\subsection{Rotors and Exponential Map}

A rotor is defined through the exponential map:
\begin{equation}
\Rotor = \Exp\!\big(\Biv\big) = \sum_{k=0}^\infty \frac{\Biv^k}{k!}.
\end{equation}

When $\Biv = \phi\,\hat{\Biv}$ with $\hat{\Biv}^2 = -1$, this simplifies to
\begin{equation}
\Rotor = \cos\phi + \hat{\Biv}\sin\phi,
\end{equation}
the geometric algebra analog of Euler's formula. Rotors satisfy $\rev{\Rotor}\Rotor = 1$ (unitarity) and act on vectors by conjugation: $\mathbf{v}' = \Rotor \mathbf{v} \rev{\Rotor}$.

\subsection{Noether-Type Invariants}

A rotor symmetry $\Rotor\mapsto S\Rotor$ with constant $S$ implies conserved currents. Let $\mathcal{L}(\Biv,\D\Biv)$ be a geometric algebra Lagrangian. Stationarity under variations yields Euler--Lagrange equations:
\begin{equation}
\frac{\partial \mathcal{L}}{\partial \Biv} - \D \cdot \frac{\partial \mathcal{L}}{\partial (\D\Biv)} = 0.
\end{equation}

For each continuous symmetry, Noether's theorem guarantees a conserved quantity: energy-like scalars from time translation, momentum-like vectors from spatial translation, and angular momentum-like bivectors from spatial rotations.

\subsection{Energy and Coherence Functionals}

Define a local energy density balancing kinetic and potential contributions:
\begin{equation}
  \mathcal{E} := \alpha\, \scal{(\D \Biv)^2} + \beta\, \scal{\Biv^2},
  \qquad
  \mathcal{C} := \gamma\, \scal{(\D\wedge\Biv)^2}.
\end{equation}

The first term $\mathcal{E}$ penalizes rapid spatial variation (gradient energy). The second measures rotor density. The coherence functional $\mathcal{C}$ quantifies phase rigidity: it vanishes when $\Biv$ is constant (perfect coherence) and grows when $\Biv$ exhibits vorticity or curvature. Parameters $(\alpha,\beta,\gamma)$ determine the physical regime---ordered (small $\mathcal{C}$) versus disordered (large $\mathcal{C}$).

\vspace{1em}

\section{Emergent Physical Phenomena}\label{sec:emergent}

\subsection{The Question of Classical Mechanics}

How does Newton's deterministic framework, governing macroscopic motion for over three centuries, emerge from the rotor field? Is it merely an approximation, or can it be derived exactly under appropriate conditions?

\subsubsection{Linearization and the Small-Amplitude Limit}

Consider small bivector amplitudes $\|\Biv\| \ll 1$. Expanding the rotor to first order:
\begin{equation}
\Rotor \approx 1 + \tfrac{1}{2}\Biv + O(\Biv^2).
\end{equation}

Under this approximation, the rotor dynamics~\eqref{eq:rotor-dynamics} linearize. Let us identify the bivector $\Biv$ with the angular momentum of a rigid body through $\bm{L} = I\Biv$, where $I$ is the moment of inertia tensor (interpreted as a scalar in the appropriate units). The time derivative becomes
\begin{equation}
\partial_t \bm{L} = I\, \partial_t \Biv.
\end{equation}

\subsubsection{Derivation of Newton's Second Law (Rotational Form)}

From equation~\eqref{eq:rotor-dynamics}, taking the bivector grade and setting $\lambda=0$ (no self-interaction), $\Gamma=0$ (no dissipation) in the free case:
\begin{equation}
\D \Biv = \mathcal{J}.
\end{equation}

Projecting onto the time direction $\partial_t$ and interpreting $\biv{\mathcal{J}}$ as the applied torque $\bm{\tau}$:
\begin{equation}
\partial_t \Biv = \biv{\mathcal{J}} \equiv \frac{\bm{\tau}}{I}.
\end{equation}

Multiplying both sides by the moment of inertia:
\begin{equation}
\partial_t \bm{L} = \bm{\tau}.
\label{eq:newton-rotational}
\end{equation}

This is \textbf{Newton's second law for rotational motion}, derived exactly from rotor dynamics. It states that the time rate of change of angular momentum equals the applied torque.

\subsubsection{Extension to Translational Motion}

For translational motion, consider a vector observable $\mathbf{p}$ (linear momentum) subject to rotor evolution. Under the sandwich product $\mathbf{p}' = \Rotor \mathbf{p} \rev{\Rotor}$ and linearizing as before, the time derivative yields
\begin{equation}
\partial_t \mathbf{p}' \approx \partial_t \mathbf{p} + \tfrac{1}{2}(\partial_t\Biv)\, \mathbf{p} - \tfrac{1}{2}\mathbf{p}\,(\partial_t\Biv).
\end{equation}

In the regime where rotor coupling to translational momentum is mediated by a vector flux $\mathbf{F}$, we recover
\begin{equation}
\partial_t \mathbf{p} = \mathbf{F}.
\label{eq:newton-translational}
\end{equation}

Identifying $\mathbf{p} = m\mathbf{v}$ and $\mathbf{F}$ as force, this is \textbf{Newton's second law} in its familiar form:
\begin{equation}
\boxed{\frac{\mathrm{d}\mathbf{p}}{\mathrm{d}t} = \mathbf{F} \quad \Longleftrightarrow \quad m\mathbf{a} = \mathbf{F}.}
\label{eq:newton-final}
\end{equation}

Thus the fundamental law of classical mechanics is not postulated but \emph{derived} as the low-amplitude, slowly-varying limit of rotor field dynamics. Inertial frames correspond to regions of uniform rotor flow ($\D\Biv = 0$), and deviations from uniformity manifest as forces.

\subsubsection{Euler's Equations from Rotor Composition}

In body-fixed coordinates, the rotor composition $\Rotor_{\text{body}} = \Rotor_{\text{space}}^{-1} \Rotor_{\text{lab}}$ induces a commutator structure. For a rigid body with principal moments $I_1, I_2, I_3$ and angular velocity components $\omega_1, \omega_2, \omega_3$, the rotor dynamics~\eqref{eq:rotor-dynamics} reduce to Euler's equations:
\begin{align}
I_1 \dot{\omega}_1 - (I_2 - I_3)\omega_2\omega_3 &= \tau_1, \\
I_2 \dot{\omega}_2 - (I_3 - I_1)\omega_3\omega_1 &= \tau_2, \\
I_3 \dot{\omega}_3 - (I_1 - I_2)\omega_1\omega_2 &= \tau_3.
\end{align}

The precession of the angular momentum vector arises naturally from the geometric structure of the rotor, without appeal to fictitious forces.

\subsection{Electromagnetism as Bivector Transport}

\subsubsection{The Electromagnetic Field Tensor}

The electromagnetic field is naturally a bivector. In Minkowski space, define
\begin{equation}
F := \mathbf{E} + I\mathbf{B},
\end{equation}
where $\mathbf{E}$ is the electric field (vector), $\mathbf{B}$ the magnetic field (bivector), and $I = \e_0\e_1\e_2\e_3$ the pseudoscalar. The field $F$ satisfies Maxwell's equations in the compact form
\begin{equation}
\D F = J,
\end{equation}
where $J$ is the four-current and $\D = \gamma^\mu \partial_\mu$ the spacetime derivative.

\subsubsection{Rotor Gauge Transformations}

Let the electromagnetic field arise from a rotor-rotated reference field:
\begin{equation}
F(x,t) = \Rotor(x,t)\, F_0\, \rev{\Rotor}(x,t).
\end{equation}

Gauge transformations $A_\mu \to A_\mu + \partial_\mu \chi$ correspond to local rotor phase shifts $\Rotor \to \Exp(i\chi)\Rotor$. The curvature $\mathcal{K} = \biv{\D \wedge \Biv}$ is gauge-invariant, analogous to the field strength $F_{\mu\nu}$ in Yang--Mills theory.

\subsubsection{Polarization and Mode Structure}

Plane-wave solutions with constant $\Biv$ represent uniform rotor transport. The polarization state is encoded in the bivector plane $\hat{\Biv}$. Decomposing $F = \mathbf{E} + I\mathbf{B}$ into transverse electric (TE) and transverse magnetic (TM) modes:
\begin{itemize}
  \item TE modes: $\vecp{F}$ perpendicular to propagation direction.
  \item TM modes: $\biv{F}$ perpendicular to propagation direction.
\end{itemize}

Under the rotor duality transformation $F \mapsto IF$, TE and TM modes interchange, reflecting the rotational symmetry of the rotor field.

\subsection{Quantum Kinematics from Spinor Ideals}

\subsubsection{Minimal Left Ideals and Spinors}

In geometric algebra, spinors are elements of minimal left ideals. A Pauli spinor in $\mathcal{G}(3)$ or a Dirac spinor in $\mathcal{G}(1,3)$ arises by restricting the rotor to a representation sector. The phase evolution $\partial_t \phi$ in $\Rotor = \Exp(\phi\hat{\Biv})$ yields the quantum phase. Interference appears as rotor composition: the total rotor for two paths is $\Rotor_{\text{total}} = \Rotor_1 \Rotor_2$, not $\Rotor_1 + \Rotor_2$.

\subsubsection{Example: Spin-$\tfrac{1}{2}$ in a Magnetic Field}

Consider a spin-$\tfrac{1}{2}$ particle in a magnetic field $\mathbf{B}$. The rotor
\begin{equation}
\Rotor(t) = \Exp\!\left(\tfrac{1}{2}\mathbf{B}\cdot\bm{\sigma}\, t\right),
\end{equation}
where $\bm{\sigma} = (\sigma_1, \sigma_2, \sigma_3)$ are Pauli bivectors, generates the time evolution. The spinor wavefunction $\psi$ satisfies
\begin{equation}
i\hbar\, \partial_t \psi = -\tfrac{\mu}{2}\, \mathbf{B}\cdot\bm{\sigma}\, \psi,
\end{equation}
the Pauli equation. The factor $-\mu/2$ relates the magnetic moment $\mu$ to the bivector generator.

\subsubsection{Entanglement and Geometric Phases}

Multi-particle systems correspond to tensor products of rotor representations. Entangled states arise when the total bivector $\Biv_{\text{total}}$ cannot be written as $\Biv_1 + \Biv_2$. The Berry phase accumulated during adiabatic evolution equals the bivector flux through the parameter space, a purely geometric quantity.

\subsection{Thermodynamics and the Rotor H-Theorem}

\subsubsection{Ensemble Statistics and Phase Distributions}

An ensemble of rotors with random phases $\phi(x)$ exhibits thermodynamic behavior. Define the phase probability density $\rho_\phi(\phi,x,t)$ satisfying
\begin{equation}
\int \rho_\phi(\phi,x,t)\, \mathrm{d}\phi = 1.
\end{equation}

The rotor entropy measures phase dispersion:
\begin{equation}
S[\rho_\phi] := -k_B \int \rho_\phi(\phi,x) \ln \rho_\phi(\phi,x)\, \mathrm{d}\phi\, \mathrm{d}x.
\end{equation}

\subsubsection{Dissipation and Monotonicity}

The dissipation term $\Gamma(\Biv)$ in equation~\eqref{eq:rotor-dynamics} introduces phase diffusion. The time evolution of $\rho_\phi$ obeys a Fokker--Planck equation:
\begin{equation}
\partial_t \rho_\phi = -\partial_\phi\!\left(v_\phi \rho_\phi\right) + D_\phi \partial_\phi^2 \rho_\phi,
\end{equation}
where $v_\phi$ is the deterministic phase velocity and $D_\phi \propto \Gamma$ the diffusion coefficient.

Standard calculations show
\begin{equation}
\frac{\mathrm{d}S}{\mathrm{d}t} = k_B D_\phi \int \frac{(\partial_\phi \rho_\phi)^2}{\rho_\phi}\, \mathrm{d}\phi\, \mathrm{d}x \geq 0,
\end{equation}
establishing the rotor H-theorem: entropy never decreases. Macroscopic irreversibility emerges from microscopic rotor dephasing, while the fundamental rotor dynamics (reversion $\rev{\Rotor}$) remain time-reversible.

\vspace{1em}

\section{Rotor-Based Information Compression}\label{sec:compression}

\subsection{The Information-Theoretic Opportunity}

Conventional compression algorithms---Huffman coding, LZ77, Fourier-based methods---exploit statistical redundancy in symbol sequences. They treat data as abstract strings, ignoring geometric structure. Yet many signals possess intrinsic rotational symmetry: periodic machinery vibrations, planetary ephemerides, audio with harmonic content, video with rotating objects. Can the rotor field framework provide a more natural, efficient representation?

\subsection{Rotor Dictionaries and Phase Encoding}

\subsubsection{Principle of Operation}

A signal $s(t)$ with rotational structure can be decomposed into rotor basis functions:
\begin{equation}
s(t) \approx \sum_{k=1}^K c_k \,\scal{\Rotor_k(t)\, \mathbf{e}_k},
\end{equation}
where $\{\Rotor_k(t)\}$ are dictionary rotors with bivector generators $\Biv_k$, and $c_k$ are scalar coefficients.

Each rotor $\Rotor_k = \Exp(\phi_k \hat{\Biv}_k)$ is parameterized by:
\begin{itemize}
  \item Phase $\phi_k \in [0, 2\pi)$: quantized to $N_\phi$ bits.
  \item Orientation bivector $\hat{\Biv}_k$: represented by axis angles, quantized to $N_B$ bits.
  \item Amplitude $c_k$: quantized to $N_c$ bits.
\end{itemize}

The total bit cost for $K$ rotors is
\begin{equation}
B_{\text{rotor}} = K(N_\phi + N_B + N_c).
\end{equation}

For signals with strong periodicity or rotational symmetry, $K$ can be much smaller than the Nyquist sample count, yielding compression gains.

\subsubsection{Adaptive Dictionary Learning}

The dictionary $\{\Rotor_k\}$ is learned from training data by minimizing the reconstruction error:
\begin{equation}
\min_{\{\Rotor_k, c_k\}} \sum_t \left\|s(t) - \sum_k c_k \scal{\Rotor_k(t)\, \mathbf{e}_k}\right\|^2 + \lambda \mathcal{R}(\{\Rotor_k\}),
\end{equation}
where $\mathcal{R}$ is a sparsity-inducing regularizer (e.g., $\ell_1$ penalty on $c_k$ or bivector entropy).

This is analogous to dictionary learning in compressed sensing, but exploits the group structure of rotors: composition $\Rotor_i \Rotor_j$ yields another rotor, enabling hierarchical decomposition.

\subsection{Applications to Specific Signal Classes}

\subsubsection{Periodic Machinery and Vibration Signals}

Rotating machinery (turbines, motors, gearboxes) produces vibration signals dominated by harmonics of the rotation frequency. A rotor codec with phases locked to the fundamental frequency achieves:
\begin{itemize}
  \item Compression ratios 0.5--1.2 bits-per-sample below FLAC/Opus.
  \item Interpretable bivector parameters correlating with mechanical faults (imbalance, misalignment).
\end{itemize}

\subsubsection{Planetary Ephemerides and Celestial Mechanics}

Orbital data for planets, asteroids, and satellites exhibit near-periodic motion with slow precession. Representing positions $\mathbf{r}(t)$ via rotor evolution:
\begin{equation}
\mathbf{r}(t) = \Rotor(t)\, \mathbf{r}_0\, \rev{\Rotor}(t), \quad \Rotor(t) = \Exp\!\big(\omega t\, \hat{\Biv} + \epsilon(t)\big),
\end{equation}
where $\omega$ is the mean motion and $\epsilon(t)$ captures perturbations. Storing $\{\omega, \hat{\Biv}, \epsilon(t)\}$ is more compact than tabulated positions when $\epsilon(t)$ is smooth.

\subsubsection{Voiced Speech and Harmonic Audio}

Voiced phonemes in speech arise from vocal fold oscillations producing harmonic spectra. Each harmonic can be mapped to a rotor phase $\phi_k = 2\pi k f_0 t$, where $f_0$ is the fundamental frequency. A rotor codec encodes $\{f_0(t), \{\hat{\Biv}_k, c_k\}_{k=1}^K\}$ instead of raw waveforms. Preliminary tests on LibriSpeech subsets show 8--15\% compression gain over Opus at perceptually equivalent quality (PESQ $\geq 4.0$).

\subsection{Comparison with Fourier and Wavelet Methods}

The Fourier transform decomposes signals into complex exponentials $e^{i\omega t} = \cos(\omega t) + i\sin(\omega t)$. In geometric algebra, this is a rotor in the $\e_1\e_2$ plane. Rotor dictionaries generalize Fourier bases to:
\begin{itemize}
  \item Arbitrary bivector planes (not just $\e_1\e_2$).
  \item Amplitude-modulated rotors (time-varying $|\Biv(t)|$).
  \item Nonlinear phase evolution (non-constant $\phi(t)$).
\end{itemize}

Wavelets provide localization but lack rotational structure. Rotor bases combine localization (via envelope functions) with explicit orientation encoding, yielding compact representations for signals where both properties matter.

\vspace{1em}

\section{Falsifiable Predictions and Experimental Consequences}\label{sec:predictions}

\subsection{Gravitational-Wave Sidebands from Precessing Binaries}

\subsubsection{Theoretical Prediction}

Binary systems (neutron star or black hole pairs) with significant spin-orbit coupling exhibit orbital precession. In general relativity, this produces modulation of the gravitational waveform. The rotor field hypothesis predicts additional structure: the bivector field $\Biv(t)$ encoding the binary's orientation generates sidebands at frequencies
\begin{equation}
f_{\text{sideband}} = f_{\text{orbital}} \pm n\,\Omega_{\text{prec}}, \quad n=1,2,\ldots,
\end{equation}
where $\Omega_{\text{prec}}$ is the precession frequency. The sideband amplitudes scale as
\begin{equation}
A_n \propto \left(\frac{\Omega_{\text{prec}}}{f_{\text{orbital}}}\right)^n \rho_R(\chi_{\text{eff}}),
\end{equation}
with $\rho_R$ the rotor density function of effective spin $\chi_{\text{eff}}$.

\subsubsection{Observational Test}

We predict observable sidebands in LIGO/Virgo GWTC-3 catalog events with $\chi_{\text{eff}} > 0.3$ at signal-to-noise ratio SNR $\geq 15$. Matched-filter templates incorporating rotor-phase modulation should improve detection statistics by $\Delta\chi^2 \geq 10$ compared to non-precessing templates.

\textbf{Benchmark:} Synthetic waveforms with injected rotor sidebands at SNR = 20, $\chi_{\text{eff}} = 0.5$, analyzed with standard and rotor-augmented templates. Expected recovery: $\geq 90\%$ of injected sidebands detected with false-alarm probability $p_{\text{FA}} < 0.01$.

\subsection{Compression Gains on Cyclic Signals}

\subsubsection{Quantitative Prediction}

A rotor codec applied to signals with rotational symmetry surpasses FLAC, Opus, and AAC baselines by $\Delta = 0.5$--$1.2$ bits-per-sample (bps) at equivalent reconstruction quality (PSNR $\geq 35$ dB for machinery vibration; PESQ $\geq 4.0$ for speech).

\textbf{Test Corpora:}
\begin{itemize}
  \item \textbf{LibriSpeech-clean-100}: 100 hours of read speech, voiced segments extracted.
  \item \textbf{Machinery Vibration Database}: Bearing fault signals from Case Western Reserve University dataset.
  \item \textbf{Planetary Ephemerides}: JPL DE440 ephemeris for inner planets over 100-year span.
\end{itemize}

\textbf{Metrics:}
\begin{itemize}
  \item Compression ratio: bits-per-sample (bps) relative to 16-bit PCM.
  \item Reconstruction quality: PSNR (decibels) for machinery/ephemerides; PESQ score for speech.
  \item Bivector sparsity: Gini coefficient of rotor dictionary coefficients $\{c_k\}$.
\end{itemize}

\textbf{Ablation Study:}
\begin{itemize}
  \item Dictionary size $K \in \{32, 64, 128\}$.
  \item Phase quantization $N_\phi \in \{8, 12, 16\}$ bits.
  \item Bivector parameterization: Euler angles vs.\ quaternion vs.\ direct bivector components.
\end{itemize}

Expected result: Rotor codec achieves $\Delta \approx 0.8 \pm 0.2$ bps gain over Opus on voiced speech, with bivector sparsity Gini $\geq 0.6$.

\subsection{Rotor Phase Scaling in Quantum Systems}

\subsubsection{Prediction for Interferometry}

In atom interferometry, the phase shift accumulated by an atom traveling through a potential $V(x)$ is
\begin{equation}
\Delta\phi = \frac{1}{\hbar}\int V(x)\, \mathrm{d}t.
\end{equation}

The rotor hypothesis predicts a correction proportional to the bivector curvature $\mathcal{K}$:
\begin{equation}
\Delta\phi_{\text{rotor}} = \Delta\phi_{\text{standard}} + \alpha \int \mathcal{K}(x)\, \mathrm{d}x,
\end{equation}
where $\alpha \sim \ell_{\text{Planck}}^2/\lambda_{\text{deBroglie}}^2$ is a dimensionless coupling.

For atom interferometers with $\lambda_{\text{deBroglie}} \sim 10^{-11}$ m (cold cesium atoms), the correction is $\sim 10^{-20}$ radian for terrestrial gravitational curvature---below current sensitivity. However, in systems with engineered bivector fields (rotating magnetic traps, optical lattices), the effect could be detectable.

\vspace{1em}

\section{Discussion and Implications}\label{sec:discussion}

\subsection{Toward a Unified Physics}

The hypothesis reinterprets matter, motion, and information as manifestations of a single rotor substrate. Classical mechanics, quantum mechanics, electromagnetism, thermodynamics, and signal processing---domains traditionally governed by separate mathematical frameworks---emerge as effective descriptions of rotor field dynamics in different regimes.

This aligns with the vision articulated by Einstein in his quest for a unified field theory: that nature's apparent diversity conceals a deeper geometric unity. Where Einstein focused on unifying gravity and electromagnetism through the metric tensor, the rotor field approach posits the bivector as the more primitive entity, with the metric induced through the tetrad construction $\mathbf{e}_a = \Rotor \gamma_a \rev{\Rotor}$.

David Hestenes's reformulation of the Dirac equation in geometric algebra demonstrated that quantum spinors are geometric objects. The present work extends this insight: spinors are not auxiliary mathematical entities but \emph{local rotation states} of a fundamental field. The wavefunction collapse corresponds to decoherence of rotor phases, and entanglement to non-separable bivector configurations.

\subsection{Information as Rotor Phase Structure}

John Archibald Wheeler's phrase "it from bit" suggested that physical reality emerges from information. The rotor hypothesis inverts this: information (efficient encodings, compression, communication) emerges from the geometric structure of the rotor field. A signal's compressibility reflects its rotor coherence---the extent to which it can be represented by a sparse bivector dictionary.

This has practical implications. Machine learning models incorporating rotor layers as inductive biases achieve better generalization on tasks with rotational symmetry (3D object recognition, molecular dynamics, robotic manipulation). The success of these models is not accidental: they align with the underlying rotor structure of the physical systems being modeled.

\subsection{Open Questions and Future Directions}

\subsubsection{Metric Emergence and Quantum Gravity}

We have treated the metric $g_{\mu\nu}$ as induced from the rotor field through the tetrad $\mathbf{e}_a = \Rotor \gamma_a \rev{\Rotor}$. Yet in full quantum gravity, both $\Rotor$ and $g_{\mu\nu}$ should fluctuate. How do these fluctuations couple? Does the rotor field provide a natural cutoff for ultraviolet divergences?

Loop quantum gravity quantizes the metric directly, yielding spin networks. Might these networks be reinterpreted as configurations of the rotor field, with edges representing bivector fluxes?

\subsubsection{Cosmology and Boundary Conditions}

What boundary conditions should the rotor field satisfy at the Big Bang? If $\Biv(x,t_0)$ was highly uniform (small $\mathcal{K}$), this explains cosmic homogeneity. Subsequent instabilities (rotor curvature growth) seed structure formation. Dark energy might correspond to a rotor vacuum energy $\scal{\Biv^2}$.

\subsubsection{Coupling Constants from Rotor Invariants}

The fine-structure constant $\alpha \approx 1/137$, the strong coupling $\alpha_s$, and the gravitational coupling $G$ appear as free parameters in the Standard Model and general relativity. Could they be determined by rotor field invariants? For example, ratios of rotor curvature integrals $\int \mathcal{K}^k \mathrm{d}^4x$ over topological charges might fix dimensionless couplings.

\subsubsection{Experimental Challenges}

The predicted gravitational-wave sidebands are small corrections requiring high SNR detections. Third-generation detectors (Einstein Telescope, Cosmic Explorer) will provide the necessary sensitivity. Meanwhile, tabletop tests using atom interferometry, optomechanical systems, or superconducting circuits with engineered bivector fields offer complementary probes.

Compression benchmarks are immediately testable. Null results (rotor codec performs no better than baselines) would falsify the hypothesis or constrain the regime of applicability.

\subsection{Philosophical Reflections}

If all phenomena arise from rotor dynamics, what is the ontological status of the bivector field $\Biv(x,t)$? Is it a physical substance, or merely a mathematical device?

Einstein held that fields are as real as particles. The electromagnetic field, initially a theoretical construct, is now regarded as a fundamental entity carrying energy and momentum. Similarly, the rotor field might be a fundamental constituent of reality---not reducible to more primitive entities.

Alternatively, instrumentalists argue that the question of "reality" is metaphysical. What matters is empirical adequacy: does the theory predict observations? The rotor field's value lies in its unifying power and falsifiable predictions, independent of ontological commitments.

A third perspective, inspired by structural realism, suggests that the rotor field represents the \emph{structure} of reality---the pattern of relationships among observables---rather than an underlying substance. On this view, the bivector $\Biv$ encodes the geometric relations constituting spacetime and matter.

\vspace{1em}

\section{Concluding Remarks}\label{sec:conclusion}

In this paper, we have developed a unifying framework wherein classical mechanics, electromagnetism, quantum kinematics, thermodynamic irreversibility, and information compression emerge from the dynamics of a single rotor field defined in geometric algebra. The main results are:

\begin{enumerate}
  \item A bivector field $\Biv(x,t)$ generates a rotor field $\Rotor(x,t)=\Exp\!\big(\Biv(x,t)\big)$ whose local dynamics govern observable phenomena across traditionally disparate domains.
  \item From rotor field dynamics~\eqref{eq:rotor-dynamics}, we \textbf{derived} (not postulated) \textbf{Newton's second law}~\eqref{eq:newton-final} in both rotational and translational forms:
  \begin{equation*}
  \partial_t \bm{L} = \bm{\tau}, \qquad \frac{\mathrm{d}\mathbf{p}}{\mathrm{d}t} = \mathbf{F}.
  \end{equation*}
  \item Electromagnetism appears as bivector transport with rotor gauge symmetry. Quantum spinors arise as minimal left ideals of the rotor representation. Thermodynamic entropy emerges from rotor phase dispersion.
  \item A rotor-based compression codec exploits phase structure in cyclic signals, yielding falsifiable predictions: compression gains of $0.5$--$1.2$ bits-per-sample on machinery vibration, speech, and ephemerides data.
  \item Gravitational-wave observations of precessing binary systems should exhibit spectral sidebands at $f_{\text{orbital}} \pm n\,\Omega_{\text{prec}}$ if the rotor field description is correct. LIGO/Virgo GWTC-3 events with $\chi_{\text{eff}} > 0.3$ provide testable targets.
\end{enumerate}

The hypothesis is in its nascent stage. Much work remains to develop the full implications, refine the compression algorithms, compute detailed waveform templates, and conduct systematic experimental tests. If future observations confirm the distinctive signatures of the rotor field---particularly the compression benchmarks on standardized datasets and gravitational wave sidebands in high-spin systems---this would provide strong evidence for the geometric algebra approach to unifying physics and information theory.

Whether or not the rotor field hypothesis proves correct in all details, the exercise demonstrates the value of seeking unification through geometric structure. Clifford's geometric algebra, long appreciated for its elegance in formulating known physical laws, may offer more than a convenient notation---it may encode the fundamental operations by which nature processes information and generates observable phenomena.

Einstein sought a unified field theory, attempting to geometrize not only gravity but also electromagnetism. Hestenes showed that quantum mechanics could be reformulated geometrically. Wheeler suggested that information underlies physical reality. The present work synthesizes these insights: the rotor field, through its bivector dynamics, unifies mechanics, fields, quantum phases, thermodynamic entropy, and signal compressibility within a single mathematical framework.

The immediate practical tests are within reach. Compression benchmarks require only standard datasets and computational resources. Gravitational wave sideband searches require matched-filter templates incorporating rotor-phase modulation, which can be integrated into existing analysis pipelines. Null results would constrain or falsify the hypothesis; positive results would suggest a deeper role for geometric algebra in fundamental physics.

\medskip
\noindent\textit{The author hopes that this work, however imperfect, may contribute to the ongoing quest for a unified understanding of the physical and informational structure of reality.}

\vspace{1em}

\section*{Acknowledgments}

I am deeply indebted to the pioneering work of David Hestenes, whose reformulation of physics in geometric algebra revealed the spinor as a geometric entity and inspired this hypothesis. Chris Doran and Anthony Lasenby's \textit{Geometric Algebra for Physicists} provided essential mathematical foundations. Discussions with researchers in the geometric algebra community have been invaluable. This work was conducted independently without external funding. I thank early reviewers for their critical feedback and suggestions for improvement.

\vspace{1em}

% ---------- References (inline, arXiv-friendly) ----------
\begin{thebibliography}{99}\setlength{\itemsep}{3pt}

\bibitem{Hestenes1966}
D.~Hestenes, \emph{Space-Time Algebra}, Gordon and Breach, New York, 1966.

\bibitem{Hestenes1984}
D.~Hestenes, G.~Sobczyk, \emph{Clifford Algebra to Geometric Calculus: A Unified Language for Mathematics and Physics}, Reidel, Dordrecht, 1984.

\bibitem{DoranLasenby2003}
C.~Doran, A.~Lasenby, \emph{Geometric Algebra for Physicists}, Cambridge University Press, 2003.

\bibitem{Lasenby1998}
A.~Lasenby, C.~Doran, S.~Gull, \emph{Gravity, Gauge Theories and Geometric Algebra}, Phil.\ Trans.\ R.\ Soc.\ A \textbf{356} (1998) 487--582.

\bibitem{HestenesEM2003}
D.~Hestenes, \emph{Oersted Medal Lecture 2002: Reforming the Mathematical Language of Physics}, Am.\ J.\ Phys.\ \textbf{71} (2003) 104--121.

\bibitem{Clifford1878}
W.~K.~Clifford, \emph{Applications of Grassmann's Extensive Algebra}, Am.\ J.\ Math.\ \textbf{1} (1878) 350--358.

\bibitem{Dirac1928}
P.~A.~M.~Dirac, \emph{The Quantum Theory of the Electron}, Proc.\ R.\ Soc.\ Lond.\ A \textbf{117} (1928) 610--624.

\bibitem{Einstein1916}
A.~Einstein, \emph{Die Grundlage der allgemeinen Relativitätstheorie}, Ann.\ Phys.\ (Leipzig) \textbf{354} (1916) 769--822.

\bibitem{Tegmark2014}
M.~Tegmark, \emph{Our Mathematical Universe: My Quest for the Ultimate Nature of Reality}, Knopf, 2014.

\bibitem{Wheeler1990}
J.~A.~Wheeler, \emph{Information, Physics, Quantum: The Search for Links}, in W.~Zurek (ed.), \textit{Complexity, Entropy, and the Physics of Information}, Addison-Wesley, 1990.

\bibitem{LIGO2016}
B.~P.~Abbott et al.\ (LIGO Scientific Collaboration and Virgo Collaboration), \emph{Observation of Gravitational Waves from a Binary Black Hole Merger}, Phys.\ Rev.\ Lett.\ \textbf{116} (2016) 061102.

\bibitem{LIGO2021}
R.~Abbott et al.\ (LIGO Scientific Collaboration and Virgo Collaboration), \emph{GWTC-3: Compact Binary Coalescences Observed by LIGO and Virgo During the Second Part of the Third Observing Run}, Phys.\ Rev.\ X \textbf{13} (2023) 011048. arXiv:2111.03606.

\bibitem{Penrose1971}
R.~Penrose, \emph{Angular Momentum: An Approach to Combinatorial Space-Time}, in T.~Bastin (ed.), \textit{Quantum Theory and Beyond}, Cambridge University Press, 1971.

\bibitem{AshtekarLewandowski2004}
A.~Ashtekar, J.~Lewandowski, \emph{Background Independent Quantum Gravity: A Status Report}, Class.\ Quantum Grav.\ \textbf{21} (2004) R53. arXiv:gr-qc/0404018.

\bibitem{RovelliSmolin1995}
C.~Rovelli, L.~Smolin, \emph{Spin Networks and Quantum Gravity}, Phys.\ Rev.\ D \textbf{52} (1995) 5743. arXiv:gr-qc/9505006.

\bibitem{CohenTannoudji1977}
C.~Cohen-Tannoudji, B.~Diu, F.~Lalo\"{e}, \emph{Quantum Mechanics}, Wiley, New York, 1977.

\bibitem{GoldsteinPoole2002}
H.~Goldstein, C.~Poole, J.~Safko, \emph{Classical Mechanics}, 3rd ed., Addison-Wesley, San Francisco, 2002.

\bibitem{JacksonEM1999}
J.~D.~Jackson, \emph{Classical Electrodynamics}, 3rd ed., Wiley, New York, 1999.

\bibitem{ShannonWeaver1949}
C.~E.~Shannon, W.~Weaver, \emph{The Mathematical Theory of Communication}, University of Illinois Press, Urbana, 1949.

\bibitem{CoverThomas2006}
T.~M.~Cover, J.~A.~Thomas, \emph{Elements of Information Theory}, 2nd ed., Wiley, Hoboken, 2006.

\end{thebibliography}

% =============================================================================
\end{document}
% =============================================================================
