% =============================================================================
% The Rotor Field Hypothesis: Unifying Matter, Information, and Motion
% arXiv-ready LaTeX template (single-file, no external .bib)
% =============================================================================
\documentclass[11pt,a4paper]{article}

% ---------- Packages ----------
\usepackage[utf8]{inputenc}
\usepackage[T1]{fontenc}
\usepackage{lmodern}
\usepackage[a4paper,margin=1in]{geometry}
\usepackage{microtype}
\usepackage{amsmath,amssymb,amsthm,mathtools}
\usepackage{physics}
\usepackage{graphicx}
\usepackage{xcolor}
\usepackage{bm}
\usepackage{booktabs}
\usepackage{enumitem}
\usepackage{hyperref}
\hypersetup{
  colorlinks=true,
  linkcolor=blue!50!black,
  citecolor=blue!50!black,
  urlcolor=blue!60!black,
  pdfauthor={Viacheslav Loginov},
  pdftitle={The Rotor Field Hypothesis: Unifying Matter, Information, and Motion}
}
\usepackage{authblk}
\usepackage{caption}

% ---------- Macros: Geometric Algebra (GA) ----------
% Basis vectors and multivector operations
\newcommand{\e}{\mathbf{e}}
\newcommand{\E}{\mathbb{E}}
\newcommand{\R}{\mathbb{R}}
\newcommand{\grade}[2]{\left\langle #1 \right\rangle_{#2}}
\newcommand{\scal}[1]{\grade{#1}{0}}
\newcommand{\vecp}[1]{\grade{#1}{1}}
\newcommand{\biv}[1]{\grade{#1}{2}}
\newcommand{\triv}[1]{\grade{#1}{3}}
\newcommand{\rev}[1]{\widetilde{#1}}           % reversion
\newcommand{\dual}[1]{#1^\ast}                 % dual
\newcommand{\geop}{\mathbin{\!\!\wedge\!\!}}   % outer/wedge product
\newcommand{\inner}{\mathbin{\!\!\cdot\!\!}}   % inner product
\newcommand{\ad}{\operatorname{ad}}
\newcommand{\Exp}{\operatorname{Exp}}

% Rotors and bivectors
\newcommand{\Rotor}{\mathcal{R}}
\newcommand{\Biv}{\mathcal{B}}
\newcommand{\Field}{\mathcal{F}}

% Differential operators
\newcommand{\D}{\nabla}                        % GA vector derivative
\newcommand{\dt}{\,\mathrm{d}t}
\newcommand{\dx}{\,\mathrm{d}x}

% ---------- Theorem-like environments ----------
\theoremstyle{definition}
\newtheorem{definition}{Definition}
\theoremstyle{plain}
\newtheorem{theorem}{Theorem}
\newtheorem{lemma}{Lemma}
\theoremstyle{remark}
\newtheorem{remark}{Remark}

% ---------- Title / Authors ----------
\title{\textbf{The Rotor Field Hypothesis: Unifying Matter, Information, and Motion}}
\author[1]{Viacheslav Loginov}
\affil[1]{Kyiv, Ukraine\\ \texttt{barthez.slavik@gmail.com}}
\date{\small Version 0.9 \quad|\quad \today}

% =============================================================================
\begin{document}
\maketitle

\begin{abstract}
\noindent
We propose that all observable structures---space, matter, and coherent information processing---emerge from the dynamics of a universal \emph{rotor field} defined in geometric algebra (GA). The fundamental object is a spatially distributed rotor $\Rotor(x,t)=\Exp\!\big(\Biv(x,t)\big)$ whose bivector generator $\Biv$ controls local orientation, phase, and coherent coupling. We show how classical mechanics, electromagnetism, quantum spinor dynamics, thermodynamic irreversibility, and learning systems appear as effective regimes of rotor coherence, transport, and symmetry breaking. We state falsifiable predictions (spectral sidebands in gravitational-wave events with precession; generalization gains of rotor-inductive priors in ML; compression gains on cyclic signals) and provide a reproducible program for experimental validation. 
\end{abstract}

\noindent\textbf{Keywords:} geometric algebra, rotor fields, unification, coherence, emergent dynamics, information, machine learning

\vspace{1em}

\section{Introduction}
Modern physics and computation rely on heterogeneous primitives (fields, particles, amplitudes, loss functions), while many universal phenomena---coherence, chirality, spin, oscillations, phase locking---share a common rotational character. We hypothesize that a single \emph{rotor field} underlies these domains and that familiar theories are effective approximations of its dynamics.
\medskip

\noindent\textbf{Contributions.}
\begin{enumerate}[leftmargin=*,itemsep=2pt]
  \item Define the \emph{Rotor Field Hypothesis} with precise GA notation and a minimal axiom set.
  \item Derive known theories as emergent limits: classical mechanics (linearized rotor flow), electromagnetism (bivector field), quantum kinematics (spinor sector), thermodynamics (phase ensemble statistics), learning (self-organization under rotor coupling).
  \item State falsifiable predictions and provide open, reproducible benchmarks.
\end{enumerate}

\vspace{1em}

\section{The Rotor Field Hypothesis}
\subsection{Kinematic postulates}
Let physical space(-time) admit a GA with orthonormal basis $\{\e_\mu\}$ and metric of signature appropriate to the regime under study. The primitive entity is a \emph{rotor field}
\begin{equation}
  \Rotor(x,t) \;=\; \Exp\!\big(\Biv(x,t)\big), 
  \qquad \rev{\Rotor}\Rotor = 1,
  \label{eq:rotor}
\end{equation}
where $\Biv(x,t)$ is a (local) bivector generating orientation and phase. Any multivector observable $A$ is rotated by
\begin{equation}
  A'(x,t) \;=\; \Rotor(x,t)\, A(x,t)\, \rev{\Rotor}(x,t).
\end{equation}

\begin{definition}[Rotor density, phase, and curvature]
Define (i) the \emph{rotor density} $\rho_R(x,t):=\scal{\Biv^2}^{1/2}$, (ii) the \emph{phase} $\phi$ by $\Biv=\phi\,\hat{\Biv}$ with $\hat{\Biv}^2=-1$, and (iii) the \emph{rotor curvature} $\mathcal{K}:=\biv{\D \wedge \Biv}$.
\end{definition}

\subsection{Dynamics}
We postulate a local balance law for the bivector generator with transport, coupling, and source terms:
\begin{equation}
  \D \Biv \;+\; \lambda\, \biv{\Biv\,\D} \;=\; \mathcal{J} \;-\; \Gamma(\Biv),
  \label{eq:rotor-dynamics}
\end{equation}
where $\lambda$ controls nonlinear advection, $\mathcal{J}$ aggregates external couplings (matter, boundary driving), and $\Gamma$ collects dissipative/decohering channels. The induced evolution of any rotated observable obeys
\begin{equation}
  \partial_t A' \;=\; \left(\partial_t \Rotor\right) A \rev{\Rotor} + \Rotor \left(\partial_t A\right)\rev{\Rotor} + \Rotor A \left(\partial_t\rev{\Rotor}\right),
\end{equation}
with $\partial_t \Rotor = \frac{1}{2}\Omega\,\Rotor$ for bivector rate $\Omega:=2(\partial_t \Rotor)\rev{\Rotor}$.

\begin{remark}
Eq.~\eqref{eq:rotor-dynamics} is a unifying template. Specific identifications of $\Biv$, $\mathcal{J}$, and $\Gamma$ yield effective theories (Sections~\ref{sec:emergent}).
\end{remark}

\vspace{1em}

\section{Mathematical Foundations}
\subsection{Geometric operations}
We use the geometric product, with grade projections $\grade{\cdot}{k}$, inner and outer products. Vector derivative $\D := \sum_\mu \e_\mu \partial_\mu$ induces GA analogs of div/rot/grad. Reversion $\rev{\cdot}$ implements time-reversal for rotors.

\subsection{Noether-type invariants}
A rotor symmetry $\Rotor\mapsto S\Rotor$ with constant $S$ implies conserved currents. Let $\mathcal{L}(\Biv,\D\Biv)$ be a GA Lagrangian; stationarity yields Euler--Lagrange equations in multivector form and conserved quantities (energy-like scalars, momentum-like vectors, spin-like bivectors).

\subsection{Energy and coherence functionals}
Define a local energy density and a coherence penalty,
\begin{equation}
  \mathcal{E} := \alpha\, \scal{(\D \Biv)^2} + \beta\, \scal{\Biv^2}, 
  \qquad 
  \mathcal{C} := \gamma\, \scal{(\D\wedge\Biv)^2},
\end{equation}
balancing smooth transport and phase rigidity. Parameters $(\alpha,\beta,\gamma)$ determine regimes (ordered/coherent vs.\ disordered/decoherent).

\vspace{1em}

\section{Emergent Phenomena}\label{sec:emergent}
\subsection{Classical mechanics as linearized rotor flow}
For small $\|\Biv\|$, write $\Rotor \approx 1+\Biv/2$. Linearizing \eqref{eq:rotor-dynamics} with suitable identification of $\mathcal{J}$ yields Newtonian transport for vector observables; inertial frames correspond to uniform rotor flow. For a rigid body, the angular momentum bivector $\bm{L}=I\Biv$ satisfies
\begin{equation}
  \partial_t \bm{L} = \biv{\mathcal{J}} \equiv \bm{\tau},
\end{equation}
where the torque $\bm{\tau}$ is identified with the bivector flux $\biv{\mathcal{J}}$. In the rotor representation, Euler's equations emerge from the commutator $[\Biv, I\Biv]$ under body-fixed coordinates, with precession naturally encoded in rotor composition.

\subsection{Electromagnetism as a bivector field}
Let the electromagnetic field be the bivector $F:=\biv{\D\wedge A}$ with potential $A$. In rotor form,
\begin{equation}
  F \;=\; \Rotor\, F_0\, \rev{\Rotor}, 
  \qquad
  \D F = J,
\end{equation}
so sources $J$ couple to rotor curvature $\mathcal{K}$. Plane waves are constant-$\Biv$ rotor transports; polarization is encoded by $\hat{\Biv}$. The TE (transverse electric) and TM (transverse magnetic) modes emerge from grade decomposition: writing $F = \bm{E} + I\bm{B}$ with pseudoscalar $I$, the TE mode corresponds to $\vecp{F}$ (electric vector part) perpendicular to propagation, while TM has $\biv{F}$ (magnetic bivector) perpendicular. Under rotor transformation, TE/TM swap via duality $F \mapsto IF$.

\subsection{Quantum kinematics (spinor sector)}
Spinors arise as minimal left ideals; a Pauli/Dirac spinor corresponds to restricting $\Rotor$ to a representation sector. The phase evolution $\partial_t \phi$ gives the quantum phase; interference appears as rotor composition. For a spin-$\tfrac{1}{2}$ particle in a magnetic field $\bm{B}$, the rotor $\Rotor = \exp(\tfrac{1}{2}\bm{B}\cdot\bm{\sigma} t)$ with Pauli bivectors $\bm{\sigma}$ yields the Pauli equation:
\begin{equation}
  i\hbar\, \partial_t \psi = -\frac{\mu}{2}\, \bm{B}\cdot\bm{\sigma}\, \psi,
\end{equation}
where $\psi$ is the spinor wavefunction. Rotor composition $\Rotor_1 \Rotor_2$ captures multi-particle entanglement and geometric phases.

\subsection{Thermodynamics and entropy}
Ensembles of rotors with noise $\Gamma$ yield entropy production via dephasing. Coarse-graining of $\phi$ produces macroscopic irreversibility (H-theorem analogue) while microscopic dynamics stay reversible under reversion $\rev{\cdot}$. Define the rotor entropy as the phase dispersion functional:
\begin{equation}
  S[\rho_\phi] := -k_B \int \rho_\phi(\phi,x) \ln \rho_\phi(\phi,x)\, \mathrm{d}\phi\, \mathrm{d}x,
\end{equation}
where $\rho_\phi$ is the phase distribution. Under dissipation $\Gamma$, monotonicity $\partial_t S \geq 0$ follows from the Fokker--Planck equation for $\rho_\phi$ with $\Gamma$ as a diffusion term, establishing the rotor H-theorem.

\subsection{Learning as self-organization}
Learning systems implement gradient flows in rotor-coupled spaces. Rotor coherence acts as an \emph{inductive bias}: it stabilizes cyclic/rotational patterns and symmetry constraints. In ML, rotor-regularized layers apply $\mathcal{L}_{\text{reg}} = \alpha \|\Rotor \mathbf{h} - \mathbf{h}'\|^2$ to enforce SO(3) equivariance and improve generalization on geometric datasets.

\vspace{1em}

\section{Predictions and Experimental Consequences}
\paragraph{Gravitational-wave sidebands.}
Binary systems with strong precession exhibit spectral sidebands at $f\pm \Omega_R$ from rotor-phase modulation, with phase--amplitude coupling tied to local rotor density $\rho_R$. We predict observable sidebands in LIGO/Virgo GWTC-3 events with $\chi_\text{eff} > 0.3$, testable via matched-filter templates including rotor-phase terms. Synthetic benchmarks available at \texttt{rotor-grav/waveforms/}.

\paragraph{Rotor-inductive ML priors.}
Neural layers implementing rotor actions (GA rotors on feature blades) improve generalization on datasets with SO(2/3), SE(3) symmetries at fixed parameter counts; robustness to phase-like corruption increases. Benchmark datasets: ModelNet40 (3D shapes), QM9 (molecular properties), N-body dynamics. Protocol: train with/without rotor layers, 5-fold CV, report test accuracy $\pm$ std; measure robustness via rotation/phase noise injection ($\sigma_\theta \in [0, \pi/4]$).

\paragraph{Compression of cyclic signals.}
A rotor-aware codec surpasses baselines on signals with rotational/phase structure (periodic machinery, orbits, voiced audio) by factor $\Delta$\,bpb with interpretable bivector dictionaries. Metrics: compression ratio (bpb), reconstruction PSNR/SNR, bivector sparsity. Test corpora: LibriSpeech (voiced), machinery vibration logs, planetary ephemerides. Ablation: rotor dictionary size $\{32, 64, 128\}$, phase quantization $\{8, 12, 16\}$ bits. Expected gain: $\Delta \approx 0.5$--$1.2$\,bpb over Opus/FLAC.

\vspace{1em}

\section{Reproducibility Program}
We release an open monorepo with:
\begin{itemize}[leftmargin=*,itemsep=2pt]
  \item \texttt{rotor-core/}: GA primitives, tests, reference rotors;
  \item \texttt{rotor-compress/}: datasets, metrics, codec;
  \item \texttt{rotor-ml/}: PyTorch/JAX rotor layers, training scripts;
  \item \texttt{rotor-grav/}: waveform generator and comparison tools.
\end{itemize}
One-command replication (Makefile) and CI artifacts ensure exact reproduction. Repository: \url{https://github.com/barthezslavik/rotor-field-hypothesis}.

\vspace{1em}

\section{Discussion and Implications}
The hypothesis reinterprets matter and information as manifestations of a single rotor substrate. It aligns with GA-based physics (Hestenes), informational views (Tegmark, Schmidhuber), while offering concrete, testable signatures across physics, chemistry, and learning systems. Open questions include metric emergence, coupling constants from rotor invariants, and cosmological boundary conditions.

\vspace{1em}

\section*{Acknowledgments}
We thank the geometric algebra community and early reviewers for valuable feedback. This work was conducted independently without external funding.

\vspace{1em}

% ---------- References (inline, arXiv-friendly) ----------
\begin{thebibliography}{99}\setlength{\itemsep}{2pt}
\bibitem{Hestenes1984}
D.~Hestenes, \emph{Clifford Algebra to Geometric Calculus}, Reidel, 1984.

\bibitem{Hestenes2003}
D.~Hestenes, O.~Sobczyk, \emph{Clifford Algebra to Geometric Calculus}, Springer, 2003.

\bibitem{DoranLasenby}
C.~Doran, A.~Lasenby, \emph{Geometric Algebra for Physicists}, Cambridge Univ.\ Press, 2003.

\bibitem{Tegmark}
M.~Tegmark, \emph{Our Mathematical Universe}, Knopf, 2014.

\bibitem{Lasenby}
A.~Lasenby, C.~Doran, S.~Gull, \emph{Gravity, Gauge Theories and Geometric Algebra}, Phil.\ Trans.\ R.\ Soc.\ A 356 (1998) 487--582.

\bibitem{HestenesEM}
D.~Hestenes, \emph{Oersted Medal Lecture 2002: Reforming the Mathematical Language of Physics}, AJP 71 (2003) 104--121.

\end{thebibliography}

% =============================================================================
\end{document}
% =============================================================================
